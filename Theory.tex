% !TEX root = Master_Thesis.tex
\chapter{Theory}
\label{ch:Theo}
	The goal of particle physics is to explain the universe we observe with basic building blocks and fundamental interactions between them. Starting with quantum mechanics in the early 20\textsuperscript{th} century, physicists have developed a theory answering many questions about the development and observable state of the universe. Up to now, the underlying theory, namely the Standard Model of elementary particles, is the most precisely tested theory in physics. In the following, the basics of this theory will be described. A special focus is placed on the elementary particle which plays the most important role in this analysis - the top quark. In addition, basic kinematics and peculiarities of proton-proton collisions and how they are simulated are examined in this chapter. 
\section{Standard Model of Particle Physics}
	The Standard Model (SM) of particle physics is a quantum field theory describing elementary particles and their interactions, providing affiliated equations of motion. All particles contained in this theory have been discovered and various experiments have confirmed predictions of the Standard Model at very high precision. Elementary particles of the Standard Model and their properties - mass, electromagnetic charge and their spin - are displayed in Fig.~\ref{SM}. They are ordered in two fundamental groups by their spin. Spin-$\frac{1}{2}$ particles are called fermions, particles with integer spin are named bosons. 
	\\
	Fermions are further divided into quarks and leptons. While quarks are affected by the strong force, leptons are not. Both subgroups consist of three generations with two particles in each group. Each generation contains an isopin doublet with two particles. A quark generation consists of one up-type and one down-type quark distinguished by their electromagnetic charge of $+\frac{2}{3}e$ and $-\frac{1}{3}e$, respectively. For leptons a generation is built from a particle with charge of $-1e$ and a neutral neutrino. Quarks and leptons from the first generation are the building blocks of atoms and thus form matter we observe on earth. All other particles have a larger mass and therefore a finite lifetime. Additionally, every fermion has an anti-partner which has the same mass but opposite charge. 
	\\	
	Bosons are carriers of the three fundamental forces included in the Standard Model. The massless photons and gluons are the carrier of the electromagnetic and strong force, respectively. $W^\pm$ and $Z^0$ bosons are massive and transfer the weak interaction. While all called bosons are spin-$1$ particle, the Higgs boson is scalar. It is a consequence of the Higgs field, that gives elementary particles their mass. Every interaction of two particles is described by a boson emitted from one and absorbed by the other particle. Thus, a boson changes energy and momentum of an absorbing or emitting particle. For each force a charge is introduced. Only particles carrying the charge connected to a force can interact with the corresponding boson. Hence, the amount of charge carried by a particle is proportional to the probability of emitting or absorbing a boson. 
	\\
	With the discovery of the Higgs boson in 2012, all particles of the Standard Model have been experimentally confirmed. Although the theory is believed to be incomplete, because known phenomena like gravity and dark matter are not included, the Standard Model is very successful describing outcomes of experiments with very high precision. A more detailed look on the three fundamental interactions is given in the sections below. 
	\begin{figure}[tb]
		\centering
		\includegraphics [width=.7\textwidth, trim = {0 0 0 3.5cm}, clip=true]{../Images/Standard_Model_of_Elementary_Particles.pdf}
		\caption{Display of the particle content of the Standard Model divided into groups of bosons and fermions in their three generations. For every particle, mass, electromagnetic charge and spin is shown. Although neutrinos are supposed to be massless in the SM, experiments show that they carry mass. Thus, the upper limit of the neutrino masses are specified. Taken from \cite{SM}.}
		\label{SM}
	\end{figure}
	
	\subsection{Quantum Electro Dynamics (QED)}
	In the history of the Standard Model, QED was the first mechanism to describe known interactions in a quantum field theory. The charge of QED is called electromagnetic charge and usually stated in units of elementary charge $e$. The mechanism to construct a quantum field theory is to take the Lagrange density for fermions, described by Dirac spinors, and require local gauge invariance. The QED requires the Lagrange density to be invariant under transformations of a $U(1)$ symmetry group. Equation~\ref{eq:qedlag} shows the Lagrange density of the Dirac equation. 
	
	\begin{equation}
	\mathcal{L} = \underbrace{\bar{\psi} i \gamma^\mu \partial_\mu \psi}_{\text{kinetic term}} - \underbrace{m \bar{\psi} \psi}_{\text{mass term}}
	\label{eq:qedlag}
	\end{equation}
	Here, $\psi$ denotes the Dirac spinors and $m$ the mass corresponding to the Dirac spinor. A vector of gamma matrices $\gamma^\mu$ is used to write the Lagrangian in a four dimensional spacetime representation. The index $\mu$ indicates one of the four components of a four-vector. While the mass term $m \bar{\psi} \psi$ is invariant under a transformation $\psi \rightarrow \psi'$ the derivative $\partial_\mu \psi$ is not. The transformation reads
	\begin{equation}
	\psi \rightarrow \psi' = e^{i q \alpha} \psi,
	\label{eq:trafo}
	\end{equation}
	where the parameter $q$ will later be identified with a coupling strength and $\alpha = \alpha(x)$ denotes a phase dependent on space-time coordinates. In order to construct an invariant Lagrange density one introduces a new vector field $A_\mu(x)$ and defines a covariant derivative $D_\mu$:
	\begin{equation}
	D_\mu = \partial_\mu + i q A_\mu(x).
	\end{equation}
	With the newly introduced covariant derivative the Lagrange density reads:
	\begin{equation}
	\mathcal{L} = \underbrace{\bar{\psi} i \gamma^\mu \partial_\mu \psi}_{\text{kinetic term}} - \underbrace{m \bar{\psi} \psi}_{\text{mass term}} - \underbrace{q \bar{\psi} \gamma^\mu \psi A_\mu(x)}_{\text{interaction term}}.
	\end{equation}
	Thus, by requiring the Lagrange density to be symmetric under gauge transformations a new vector field is predicted. This vector field is then associated with the photon. Because no mass term for the Photon field occurs, it is expected to be massless. Additionally the coupling between fermions $\psi$ and photon $A_\mu$ with a coupling strength $q$ is included. \cite{ModernParticlePhysics}
	
	\subsection{Quantum Chromo Dynamics (QCD)}
	The theory behind QCD is developed analogously to QED. The main difference between QCD and QED is its charge space. The charge of QCD is called colour and exists in three states for particles: red, green and blue. Additionally every colour has its anti-colour carried by anti-particles. Thus, one has to consider rotations in the three dimensional colour space, represented by an $SU(3)$ group. Generators $T_a$, with $a$ running from $1$ to $8$, are used to express these rotations and have to be considered. Similar to QED a covariant derivative is defined to construct a symmetric Lagrange density. The covariant derivative for QCD reads
	\begin{equation}
	D_\mu = \partial_\mu + i g_s T^a G^a_{\mu},
	\end{equation}
	where $g_s$ denotes the coupling of quarks to the strong force. Secondly eight new vector fields $G_a^\mu$ are introduced, which are associated with gluons, the bosons of the strong force. Because the generators of $SU(3)$ do not commute, not only interaction between quarks and gluons but also gluon-gluon interaction is allowed. In Eq.~\ref{eq:qcd} self interaction of gluons is combined with their kinetic energy term. The complete Lagrange density of QCD reads:
	\begin{equation}
	\mathcal{L} = \underbrace{\bar{\psi}(i \gamma^\mu \partial_\mu -m ) \psi}_{\text{mass and kinetic term quarks}} - \underbrace{g_s \bar{\psi}(\gamma^\mu T^a G^a_\mu) \psi}_{\text{quark-gluon interaction}} - \underbrace{\frac{1}{4} G^a_{\mu \nu}G^{a \mu \nu}}_{\substack{\text{kinetic term gluons} \\ \text{and gluon-gluon interaction}}}
	\label{eq:qcd}
	\end{equation}
	Gluon-gluon interactions lead to a special effect in QCD: the force between two colour charges increases with their distance. Thus, the energy stored in the field between the two charges will increase as well. At some point the energy is large enough to produce new colour charged particles. Because of that, quarks can not exist freely but only in colour neutral bound states, which are called hadrons.
	\subsection{Weak Interaction}
	The third force described in the Standard Model is the weak force. The charge of the weak force is called isospin and is carried by $W^\pm$ and $Z^0$ bosons. Because all weak bosons are massive, the weak interaction has a very limited range. Therefore, weak interaction only takes place on short distances. The corresponding symmetry group is a $SU(2)_L$, where $L$ indicates a preferred coupling to left-handed particles and right-handed anti-particles. . While $Z^0$ bosons also couple to right-handed particles, $W^\pm$ do not. This asymmetry in coupling strength is caused by a violation of parity. Through the $W^\pm$ bosons, fermions can change their flavour. he left-handed quarks and leptons are arranged in isospin doublets, while right-handed particles are represented by singlets. Right-handed neutrinos do not exist,
	\begin{equation}
	\begin{pmatrix}	u \\ d'	\end{pmatrix}_L,
	\begin{pmatrix}	c \\ s'	\end{pmatrix}_L,
	\begin{pmatrix}	t \\ b'	\end{pmatrix}_L,
	\begin{pmatrix}	\nu_e \\ e	\end{pmatrix}_L,
	\begin{pmatrix}	\nu_\mu \\ \mu	\end{pmatrix}_L,
	\begin{pmatrix}	\nu_\tau \\ \tau	\end{pmatrix}_L,
	u_r, d_r, c_r, s_r, b_r, t_r, e_r, \mu_r, \tau_r.
	\end{equation}
	The theory only allows flavour changing processes inside these doublets. Although, decays from one generation to another have been observed experimentally. The answer to this contradiction is that flavour states and mass states are not referring to the same state. The weak interaction changes the flavour but the state one observes, the mass state, is a superposition of different flavour states. By definition, flavour and mass states of up-type quarks are identical. For down-type quarks the Cabibbo-Kobayashi-Maskawa (CKM) matrix is introduced to account for the mixing of flavour states,
	\begin{equation}
	\begin{pmatrix}d'\\s'\\b'\end{pmatrix} = 
	\begin{pmatrix}
	V_{ud}&&V{us}&&V{ub} \\
	V_{cd}&&V{cs}&&V{cb} \\
	V_{td}&&V{ts}&&V{tb}
	\end{pmatrix}
	\begin{pmatrix}d\\s\\b\end{pmatrix}.
	\end{equation}
	The probability that a quark of type $i$ will transform into a quark of type $j$ when it emits a $W^\pm$ is then calculable via $|V_{ij}|^2$. Because of the mixing of states, not only a violation of parity but also CP-violation (charge and parity) is observed.
	\subsection{Electroweak Unification}
	\label{sec:elw}	
	To get a better understanding of the forces in the SM, it would be a large success to lead the different forces back to one fundamental mechanism. A first step towards a unification of all three interactions can be realised by combining QED with the weak force. Electromagnetic charge $Q$ and weak isospin $T_3$ are combined to achieve a charge for the electroweak unification, called hypercharge $Y$:
	\begin{equation}
	Y = 2(Q-I_3).
	\end{equation}
	Again, the Lagrange density is required to be invariant under transformations of the underlying $SU(2)_L \times U(1)_Y$ group. This results into four new massless bosons $W_1$, $W_2$, $W_3$ and $B$. 
	\subsection{Higgs Mechanism and Spontaneous Symmetry Breaking}
	\label{sec:higgs}
	The bosons, $W_1$, $W_2$, $W_3$ and $B$, found above in section~\ref{sec:elw} are massless. In contradiction, the masses of the $W$ and $Z$ bosons are measured to values of $80\;\text{GeV}$ and $91\;\text{GeV}$, respectively~\cite{pdg2016}. Therefore, a theory was developed which solves this problem via spontaneous symmetry breaking. A field is introduced where the Lagrange density for local variations is not symmetric while the Lagrangian of the field itself is. This leads to mass terms for electro-weak bosons and a new boson - the Higgs boson - in the Standard Model Lagrange density. After spontaneous symmetry breaking, $W_1$ and $W_2$ are combined to get the $W^\pm$ bosons of the weak interaction via
	\begin{equation}
	W^\pm = \frac{1}{\sqrt{2}}(W_1 \mp i W_2).
	\label{eq:W}
	\end{equation}
	The other two fields $W_3$ and $B$ mix and result in the $Z^0$ boson from the weak and the photon $\gamma$ of the electromagnetic interaction:
	\begin{equation}
	\begin{pmatrix}\gamma \\ Z^0\end{pmatrix} =
	\begin{pmatrix}
	\cos \theta_W && \sin \theta_W \\
	-\sin \theta_W && \cos \theta_W 
	\end{pmatrix}
	\begin{pmatrix}B \\ W_3\end{pmatrix}
	\label{eq:Z}
	\end{equation}	
	Here the weak mixing angle $\theta_W$ is introduced, which also relates the masses $M_W$ and $M_Z$ of $W^\pm$ and $Z^0$ bosons via
	\begin{equation}
	M_Z = \frac{M_W}{\cos \theta_W}.
	\end{equation}
	Finally, also fermion masses can be included in this mechanism. Therefore, a Yukawa coupling to the Higgs field proportional to their masses is introduced. After symmetry breaking, the Standard Model of particle physics can be written in a Lagrangian representation. For this, the Lagrangian density of QCD (see Eq.~\ref{eq:qcd}) has to be combined with the full Lagrangian density of the electro-weak interaction after symmetry breaking. The latter contains all kinetic terms of fermions, mass terms of electro-weak gauge bosons and Higgs boson, couplings of fermions to electro-weak gauge bosons and the Higgs boson, self-interaction terms of called bosons, interactions between electro-weak bosons and the Higgs boson as well as the Yukawa couplings of fermions to the Higgs field.

		
\section{Top Quark}
	The top quark is an up type quark belonging to the third generation of the Standard Model and carrying electromagnetic charge of $+\frac{2}{3}e$ \cite{pdg2016}. With its mass of about $173\;\text{GeV}$ the top quark is the heaviest particle in the Standard Model. It was predicted in 1973 \cite{topPredict} to account for observed CP-violation that could not be explained with the two known quark generations. After finding the much lighter bottom quark only a few years after its prediction, the top quark was discovered in 1995 by the CDF \cite{topCDF} and D\O{} \cite{topD0} collaborations at Tevatron. Because of its high mass, it offers a large phase space for decays and has a short life time of approximately $5 \times 10^{-25}\;\text{s}$~\cite{pdg2016}. Because of the short life time the top quark does not form bound hadronic states. Thus, measurements of the bare quark are possible, providing a direct access to parameters of the Standard Model. Additionally, the top quark is from particular interest for searches of physics beyond the Standard Model since it is often part of the final state and/or a dominant background. 
	
	\subsection{Production in Hadron Colliders and Decay}
	In hadron colliders, the production of a $t\bar{t}$ pair happens via $q\bar{q}$ annihilation (see Fig.~\ref{fig:productiona}) or gluon fusion (see Fig.~\ref{fig:productionb} - \ref{fig:productiond}). At the centre-of-mass energy of $13\;\text{TeV}$, as reached with the LHC, gluon fusion is by far the dominant process. Top quarks can also be produced in single production, but this has, being a electroweak process, a much smaller cross section. Hence, this analysis will focus on pair produced top quarks and will treat single top production as a background process.
	\begin{figure}
		\centering
		\begin{subfigure}{.4\textwidth}
		\begin{tikzpicture}
		\begin{feynman}	
		\vertex(a);
		\vertex[right=of a] (b);
		\vertex[below left=of a] (i1){\(q\)};
		\vertex[above left=of a] (i2){\(\overline q\)};		
		\vertex[below right=of b] (f1){\(\overline t\)};				
		\vertex[above right=of b] (f2){\(t\)};		
		\diagram* {
		(a)-- [gluon,edge label'=\(g\)] (b),
		(i1)-- [fermion] (a) -- [fermion] (i2),
		(f1)-- [fermion] (b) -- [fermion] (f2),
		};
		\end{feynman}
		\end{tikzpicture}
		\caption{}
		\label{fig:productiona}
		\end{subfigure}
		\begin{subfigure}{.4\textwidth}
		\begin{tikzpicture}
		\begin{feynman}	
		\vertex(a);
		\vertex[right=of a] (b);
		\vertex[below left=of a] (i1){\(g\)};
		\vertex[above left=of a] (i2){\(g\)};		
		\vertex[below right=of b] (f1){\(\overline t\)};				
		\vertex[above right=of b] (f2){\(t\)};		
		\diagram* {
		(a)-- [gluon,edge label'=\(g\)] (b),
		(i1)-- [gluon] (a) -- [gluon] (i2),
		(f1)-- [fermion] (b) -- [fermion] (f2),
		};
		\end{feynman}
		\end{tikzpicture}
		\caption{}
		\label{fig:productionb}
		\end{subfigure}		
		\begin{subfigure}{.4\textwidth}
		\begin{tikzpicture}
		\begin{feynman}	
		\vertex(a);
		\vertex[below=of a] (b);
		\vertex[left=of a] (i1){\(g\)};
		\vertex[left=of b] (i2){\(g\)};		
		\vertex[right=of a] (f1){\(t\)};				
		\vertex[right=of b] (f2){\(\overline{t}\)};		
		\diagram* {
		(b)-- [fermion] (a),
		(i1)-- [gluon] (a) -- [fermion] (f1),
		(i2)-- [gluon] (b) -- [anti fermion] (f2),
		};
		\end{feynman}
		\end{tikzpicture}
		\caption{}
		\label{fig:productionc}
		\end{subfigure}
		\begin{subfigure}{.4\textwidth}
		\begin{tikzpicture}
		\begin{feynman}	
		\vertex(a);
		\vertex[below=of a] (b);
		\vertex[left=of a] (i1){\(g\)};
		\vertex[left=of b] (i2){\(g\)};		
		\vertex[right=of a] (f1){\(t\)};				
		\vertex[right=of b] (f2){\(\overline{t}\)};		
		\diagram* {
		(b)-- [fermion] (a),
		(i1)-- [gluon] (b) -- [anti fermion] (f2),
		(i2)-- [gluon] (a) -- [fermion] (f1),
		};
		\end{feynman}
		\end{tikzpicture}
		\caption{}
		\label{fig:productiond}
		\end{subfigure}
		\caption{Feynman diagrams showing the production of top quark pairs. Displayed are quark anti-quark annihilation (a), and gluon-gluon fusion in the s-channel (b), t-channel  (c) and u-channel (d). Created with \cite{feynman}.}
		\label{fig:production}
	\end{figure}	
	The top quark decays via the weak interaction with a probability of almost $100\%$ into a bottom quark and a $W$ boson. This property arises from the entry in the CKM matrix $|V_{tb}|^2 \approx 0.998$ \cite{pdg2016}. While the bottom quark is seen as a jet in the detector, the $W^\pm$ boson further decays into a $q\bar{q}$ pair (see Fig.~\ref{fig:decaya}) or into a lepton (anti-lepton) and an anti-neutrino (neutrino) (see Fig.~\ref{fig:decayb}). Looking at $t\bar{t}$ production, this results in three possible channels: 
	\begin{itemize}
	\item both $W$ bosons decay into quarks (fully hadronic)
	\item one $W$ boson decays hadronically, the other one leptonically (lepton+jets)
	\item both $W$ bosons decay leptonically (dilepton)
	\end{itemize}
	The fully hadronic and lepton+jets channels are dominant and occur $45.7\%$ and $43.8\%$ of the time. $10.5\%$ of all $t\bar{t}$ events result in two leptons in the final state \cite{pdg2016}. This analysis will focus on the lepton+jets channel which is pictured in Fig.~\ref{fig:semilep}. It is suitable because the lepton makes it easier to distinguish $t\bar{t}$ events from background events but also includes a fully hadronically decaying top quark which will be the target for the presented measurement.
	
	\begin{figure}
	\begin{subfigure}{.5\textwidth}
		\begin{tikzpicture}
		\begin{feynman}
		\vertex(a) {\(t\)};
		\vertex[right=of a] (b);
		\vertex[below right=3cm of b] (f1){\(b\)};
		\vertex[above right=of b] (c);
		\vertex[above right=1.5cm of c] (f2){\(q'\)};
		\vertex[below right=1.5cm of c] (f3){\(\overline{q}\)};
		\diagram* {
		(a)-- [fermion] (b)-- [fermion] (f1),
		(b)-- [boson,edge label'=\(W\)] (c),
		(c)-- [fermion] (f2),
		(c)-- [anti fermion] (f3),};
		\end{feynman}
		\end{tikzpicture}
		\caption{}
		\label{fig:decaya}
	\end{subfigure}	
	\begin{subfigure}{.5\textwidth}
		\begin{tikzpicture}
		\begin{feynman}
		\vertex(a) {\(t\)};
		\vertex[right=of a] (b);
		\vertex[above right=1.5cm of b] (c);
		\vertex[below right=3cm of b] (f1){\(b\)};
		\vertex[below right=1.5cm of c] (f3){\(\overline{l}\)};
		\vertex[above right=1.5cm of c] (f2){\(\nu_l\)};

		\diagram* {
		(a)-- [fermion] (b)-- [fermion] (f1),
		(b)-- [boson,edge label'=\(W\)] (c),
		(c)-- [fermion] (f2),
		(c)-- [anti fermion] (f3),};
		\end{feynman}
		\end{tikzpicture}
		\caption{}
		\label{fig:decayb}
	\end{subfigure}
	\caption{Feynman diagrams of a fully hadronic (a) and lepton + jet (b) decay of the top quark. Decays of anti top quarks are constructed similarly. Created with \cite{feynman}.}
	\end{figure}
	
	\begin{figure}
		\centering
		\begin{tikzpicture}
		\begin{feynman}	
		\vertex[blob] (a) {};
		%\vertex (a);
		\vertex[right=of a] (b);
		\vertex[left=of a] (c);
		\vertex[above right=3.5cm of b] (bb){\(b\)};
		\vertex[right=of b] (bW);
		\vertex[left=of c] (cW);
		\vertex[below left=3.5cm of c] (cb){\(\overline b\)};
		\vertex[above right=of bW] (bq){\(q'\)};		
		\vertex[below right=of bW] (baq){\(\overline q\)};				
		\vertex[above left=of cW] (cl){\(l\)};		
		\vertex[below left=of cW] (cnu){\(\overline \nu_l\)};
		\diagram* {
		(a)-- [fermion,edge label'=\(t\)] (b),
		(a)-- [anti fermion,edge label'=\(\overline t\)] (c),
		(b)-- [fermion] (bb),
		(b)-- [boson,edge label'=\(W^+\)] (bW),
		(c)-- [anti fermion] (cb),
		(c)-- [boson,edge label'=\(W^-\)] (cW),
		(bW)-- [fermion] (bq),
		(bW)-- [anti fermion] (baq),
		(cW)-- [fermion] (cl),
		(cW)-- [anti fermion] (cnu),
		};
		\end{feynman}
		\end{tikzpicture}
		\caption{Feynman diagram displaying an event from the lepton+jets channel. The centred circle indicates various $t\bar{t}$ production mechanisms (see Fig~\ref{fig:production}). Created with \cite{feynman}.}
		\label{fig:semilep}
	\end{figure}
	
\section{Physics of Proton-Proton Collisions}
	In collision experiments, the center-of-mass energy $\sqrt{s}$ defines the energy that is available to create particles and give them kinetic energy. At the LHC, $s$ is set by the momenta of the incoming protons via
	\begin{eqnarray}
	s &=& (p_1 + p_2)^2 \\
	  &=& p_1^2 + p_2^2 + 2 p_1 p_2 \\
	  &=& (E_1^2 - \vec{p}_1^2 )  + (E_2^2 - \vec{p}_2^2 ) + 2 (E_1 E_2 - \vec{p}_1 \vec{p}_2 ),
	\end{eqnarray}
	where $p_1$ and $p_2$ are the four-momenta of two colliding protons with energies $E$ and momentum vector $\vec{p}$. Because the two beams have the same energy but opposite directions, the assumptions $E_1 = E_2 = E_\text{proton}$ and $\vec{p}_1 = -\vec{p}_2$ hold. With that, the center-of-mass energy reads:
	\begin{equation}
	\sqrt{s} = 2 E_\text{proton}.
	\end{equation}
	Because protons are composite particles, the actual scattering involves quarks or gluons, called partons, carrying only a fraction of the nominal beam momentum $x p_\text{proton}$. The center-of-mass energy of the hard scattering process $\sqrt{\hat{s}}$ is therefore only a fraction of the stated $13\;\text{TeV}$:
	\begin{equation}
	\sqrt{\hat{s}} = \sqrt{x_1 x_2} \sqrt{s}.
	\end{equation} 
	Here, $x_1$ and $x_2$ indicate the fraction of the proton momentum the two colliding partons carry. Because this fraction is not known, the center-of-mass energy of the hard process is not known either. Therefore, for data analysis one has to use variables that do not depend on the initial momentum in the direction of flight of the partons, which are explained in section~\ref{sec:coordinate}. Another peculiarity in hadron colliders is the use of parton density functions (PDFs). PDFs return the probability to find a particular parton inside a given momentum interval in a proton. They are measured in electron-proton scattering experiments \cite{pdf} and crucial to predict production cross sections.
		
\subsubsection{Pile-Up}
	The high collision rate at the LHC is required to collect a large amount of data and be able to observe even very rare processes. But the high rate also has a disadvantage. Since multiple proton-proton interactions take place per bunch crossing, not only one interesting but several other scattering events, mostly soft QCD processes, are seen simultaneously in the detector. This effect is called pile-up. If not corrected for, energy measurements do always include particles not originating from the hard scattering one is interested in. To reduce pile-up effects, it is crucial to be able to identify additional energy as well as distinguish between hard scattering and vertices from additional interactions.
	
\subsubsection{Underlying Event}
	Since protons are composite particles, multiple parton interactions can occur at a proton-proton collision. This effect is called the underlying event and addresses events where multiple partons from two colliding protons interact. Since both interactions share the same interaction point, they can not be separated. Hence, the underlying event has to be understood and is included in the simulation of events.
	
\section{Event Generator and Parton Shower}
\label{sec:Simulation}
	In data analyses at the LHC, theory predictions are needed to validate the data. The state measured in a detector can be split in various physical processes. All of which have to be considered when one predicts an collision event in the LHC. Firstly, the cross section of the colliding partons has to be calculated. This involves particle density functions as well as initial state radiation. Then, the matrix element of the hard scattering is defining the occurrence, momenta and direction of final state particles. Here, it has also has to be taken care of multiple interactions in one proton-proton collision. Final state particles then can further decay and radiate. A radiation of photons or gluons then lead to a parton shower because of pair production like $\gamma \rightarrow e^+ e^-$. Furthermore, hadronisation of quarks has to be considered, forming jets in the detector. With this variety of processes, cross sections are not analytically calculable. In addition one has to account for detector effects, to be able to compare a prediction with data affected by the measurement. Therefore, simulations through Monte Carlo (MC) methods are derived, to be able to obtain a valid prediction. The initial states and hard scattering is calculated with an MC generator like MADGRAPH \cite{madgraph} or POWHEG \cite{powheg} while parton shower, hadronisation as well as particle decays are mostly taken care of with PYTHIA8 \cite{pythia8}. A detector simulation, achieved with GEANT 4 \cite{geant4}, is applied afterwards to make Monte Carlo samples comparable to data.
