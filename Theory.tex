\section{Theory}
\subsection{Standard model of particle physics}
	The standard models of particles is a quantum field theory describing elementary particles and their interaction. All particles contained in this theory have been discovered and experiments have confirmed predictions of the standard model at very high precision. In Fig. \ref{SM} all elementary particles and their basic properties are displayed. The particles can be ordered in groups by their properties. Firstly one distinguishes particles depending on their spin. Spin-$\frac{1}{2}$ particles are called fermions, Spin-$1$ and Spin-$0$ particles represent the bosons. Fermions are the building block of matter while bosons carry the fundamental forces included in the standard model. Quarks and leptons are sub groups of the fermions are divided by their possible interaction with forces. While the quarks are affected by the strong force, carried by the gluon, the leptons are not.
	% QED
	% QCD
	% WEAK
	% HIGGS
	Three fundamental forces are introduced in this theory. The strong force is carried by the gluon and affects quarks and the gluon itself. The charge of the strong interaction is called colour.
	\begin{figure}[tb]
		\centering
		\includegraphics [width=\textwidth]{../Plots/Standard_Model.png}
		%from: http://www.stfc.ac.uk/stfc/cache/file/8E510467-ACAF-4734-9FEAE2766B53FD1A.png
		\caption{}
		\label{SM}
	\end{figure}
		 
\subsubsection{Top quark}
	The top quark is the heaviest particle in the standard model and therefore offers a large phase space for decays and has a life time of approximately $0.5 \times 10^{-24}\;\text{s}$. Because of the short life time the top quark does not form bound hadronic states and thus measurement of the bare quark are possible. This provides a special access to parameters of the standard model. Especially the top quark mass is an essential parameter to check the standard model for consistency. The Higgs mechanism relates the masses of the top quark with the masses of the W and Higgs boson. Measuring these masses with high precision gives the possibility to check the standard model for consistency.
	
	Additionally the top quark is important for searches of new physics since is is often part of the final state and/or a relevant background.
	% LIVE TIME -> no hadronisation
	% YUKAWA ? 

	% MC Templates, MC Mass
	% ENERGY SCALE
	% RUNNING MASS
\subsection{Unfolding}
	Most analyses at LHC measure distributions of appropriate variables and then compare the obtained results in data with event simulations. In this method the MC samples also include detector effects. What one measures in this case is the real distribution on particle level folded with an unknown detector function. Studying the difference in MC between particle level and reconstruction level it is possible to calculate the probabilities that a measured value in a bin $x_i$ is originating from bin $y_i$ on particle level. The resulting matrix can then be applied to real data to obtain data on particle level. This can then be compared with theory calculations.
	% COMPARE TO THEORY
	% WELL DEFINED MASS
	% CORRECT FOR DETECTOR
	