\chapter{Theory}
\label{ch:Theo}
	The goal of particle physics is to explain the universe we observe with basic building blocks and fundamental interactions between them. Starting with quantum mechanics in the early 20\textsuperscript{th} century, physicists have developed a theory answering many questions about the development and observable state of the universe. Up to now, the current status of the theory, namely standard model of elementary particles, is the most precisely tested theory in physics. In the following, basics of this theory will be described. A special focus is set to the elementary particle which plays the most important role in this analysis - the top quark. Additionally, the goal and procedure of unfolding in particle physics is introduced and described.
\section{Standard model of particle physics}
	The standard model of particle physics is a quantum field theory describing elementary particles, their interactions and provides affiliated equations of motion. All particles contained in this theory have been discovered and various experiments have confirmed predictions of the standard model at very high precision. Elementary particles of the standard model and their properties - mass, electromagnetic charge and their spin - are displayed in Fig. \ref{SM}. They are ordered in two fundamental groups by their spin. Spin-$\frac{1}{2}$ particles are called fermions, particles with integer spin are named bosons. 
	\\
	Fermions are further divided into quarks and leptons. While quarks are affected by the strong force, leptons are not. Both subgroups consist of three generations with two particles in each group. Each generation of quarks contain one up-type and one down-type quark distinguished by their electromagnetic charge of $+\frac{2}{3}e$ and $-\frac{1}{3}e$, respectively. For leptons a generation is built from a particle with charge of $-1e$ and an neutral neutrino. The first generation of quarks and leptons then are the building blocks of atoms. All other particles have a larger mass and therefore a finite lifetime. Additionally every fermion has a anti-partner which has the same mass but opposite charge. 
	\\	
	Bosons are carriers of the three fundamental forces included in the standard model. Every interaction of two particles is described by a boson emitted from one and absorbed by the other particle. Thus, a boson changes energy and momentum of an absorbing or emitting particle. For each force a charge is introduced. Only particles carrying the charge connected to a force can interact with the corresponding boson. Hence, the amount of charge carried by a particle is proportional to the probability of emitting or absorbing a boson. 
	\\
	With the discovery of the Higgs boson in 2012, all particles of the standard model have been experimentally confirmed. Although, the theory is believed to be incomplete, because known phenomena like gravity and dark matter are not included, the standard model is very successful describing outcomes of experiments with very high precision. A more detailed look on the three fundamental interactions is given in the sections below. 
	\begin{figure}[htb]
		\centering
		\includegraphics [width=.8\textwidth, trim = {0 0 0 3.5cm}, clip=true]{../Images/Standard_Model_of_Elementary_Particles.pdf}
		\caption{Display of the particle content of the standard model divided into groups of bosons and fermions in their three generations. For every particle mass, electromagnetic charge and spin is shown. Although neutrinos are supposed to be massless in the SM, experiments show that they carry mass. Thus, the upper limit of the neutrino masses are specified. Taken from \cite{SM}.}
		\label{SM}
	\end{figure}
	
	%todo Gruppen angeben?
	\subsection{Quantum Electro Dynamics (QED)}
	In the history of the standard model, QED was the first mechanism to describe known interactions in a quantum field theory. The charge of QED is called electromagnetic charge and usually stated in units of elementary charge $e$. To mechanism to construct a quantum field theory is to take the Lagrange density for fermions, described by Dirac spinors, and require local gauge invariance. The QED requires the Lagrange density to be invariant under transformations of a $U(1)$ symmetry group. Equation \ref{eq:qedlag} shows the Lagrange density of the Dirac equation. 
	
	\begin{equation}
	\mathcal{L} = \underbrace{\bar{\psi} i \gamma^\mu \partial_\mu \psi}_{\text{kinetic term}} - \underbrace{m \bar{\psi} \psi}_{\text{mass term}}
	\label{eq:qedlag}
	\end{equation}
	%todo Gamma beschreiben?
	Here, $\psi$ denotes the Dirac spinors and $m$ the mass corresponding to the Dirac spinor. A vector of gamma matrices $\gamma^\mu$ is used to write the Lagrangian in a four dimensional spacetime representation. While the mass term $m \bar{\psi} \psi$ is invariant under a transformation $\psi \rightarrow \psi'$ the derivative $\partial_\mu \psi$ is not. The transformation reads:
	\begin{equation}
	\psi \rightarrow \psi' = e^{i q \alpha} \psi,
	\label{eq:trafo}
	\end{equation}
	where the parameter $q$ will later be identified with a coupling strength and $\alpha = \alpha(x)$ denotes a phase dependent on space-time coordinates. In order to construct a symmetric Lagrange density one introduces a new vector field $A_\mu(x)$ and defines a covariant derivative $D_\mu$:
	\begin{equation}
	D_\mu = \partial_\mu + i q A_\mu(x)
	\end{equation}
	With the new introduced covariant derivative the Lagrange density reads:
	\begin{equation}
	\mathcal{L} = \underbrace{\bar{\psi} i \gamma^\mu \partial_\mu \psi}_{\text{kinetic term}} - \underbrace{m \bar{\psi} \psi}_{\text{mass term}} - \underbrace{q \bar{\psi} \gamma^\mu \psi A_\mu(x)}_{\text{interaction term}}.
	\end{equation}
	Thus, by requiring the Lagrange density to be symmetric under gauge transformations a new vector field is predicted. This vector field is then associated with the photon. Because no mass term for the Photon field occurs, it is predicted to be massless. Additionally the coupling between fermions $\psi$ and photon $A_\mu$ with a coupling strength $q$ is included. \cite{ModernParticlePhysics}
	\subsection{Quantum Chromo Dynamics (QCD)}
	The theory behind QCD is developed analogously to QED. The main difference between QCD and QED is its charge space. The charge of QCD is called colour and exists in three states for particles: red, green and blue. Additionally every colour has its anti-colour carried by anti-particles. Thus, one has to consider rotations in the three dimensional colour space. Here it is important to take the generators $T_a$ of $SU(3)$ into account. Similar to QED a covariant derivative is defined to construct a symmetric Lagrange density. The covariant derivative for QCD reads:
	\begin{equation}
	D_\mu = \partial_\mu + i g_s T^a G^a_{\mu},
	\end{equation}
	where $g_s$ denotes the coupling of quarks to the strong force. Secondly eight new vector fields $G_a^\mu$ are introduced, which are associated with gluons, the bosons of the strong force. Because the generators of $SU(3)$ do not commute, not only interaction between quarks and gluons but also gluon-gluon interaction is allowed. In the following Eq. \ref{eq:qcd} the self interaction of gluons is combined with their kinetic energy term. The complete Lagrange density of QCD reads:
	\begin{equation}
	\mathcal{L} = \underbrace{\bar{\psi}(i \gamma^\mu \partial_\mu -m ) \psi}_{\text{mass and kinetic term guarks}} - \underbrace{g_s \bar{\psi}(\gamma^\mu T^a G^a_\mu) \psi}_{\text{quark-gluon interaction}} - \underbrace{\frac{1}{4} G^a_{\mu \nu}G^{a \mu \nu}}_{\substack{\text{kinetic term gluons} \\ \text{and gluon-gluon interaction}}}
	\label{eq:qcd}
	\end{equation}
	Gluon-gluon interactions lead to a special effect in QCD: the force between two colour charges increases with their distance. Thus, the energy stored in the field between the two charges will increase as well. At some point the energy is large enough to produce new colour charged particles. Because of that, quarks can not exist freely but only in colour neutral bound states, which are called hadrons.
	\subsection{Weak Interaction}
	The third force described in the standard model is the weak force. It is carried by the $W^\pm$ and $Z^0$ bosons. Because all weak bosons are massive, they decay very quickly. Therefore, weak interaction only takes place on short distances. The corresponding symmetry group is a $SU(2)_L$, where $L$ indicates a preferred coupling to left-handed particles and right-handed anti-particles. While $Z^0$ bosons also couple to right-handed particles, $W^\pm$ do not. This asymmetry in coupling strength leads to a violation of parity. Through the $W^\pm$ bosons, fermions can change their flavour. Therefore, left-handed isospin doublets are defined. The quark sector then reads:
	\begin{equation}
	\begin{pmatrix}	u \\ d'	\end{pmatrix}_L,
	\begin{pmatrix}	c \\ s'	\end{pmatrix}_L,
	\begin{pmatrix}	t \\ b'	\end{pmatrix}_L
	\end{equation}
	The theory only allows flavour changing processes only inside these doublets. Although, decays from one generation to another have been observed experimentally. The answer to this contradiction is, flavour states and mass states are not interchangeable. Weak interaction changes the flavour but the state one observes, the mass state, is a superposition of different flavour states. By definition flavour and mass state of up-type quarks are identical. For down-type quarks the Cabibbo-Kobayashi-Maskawa (CKM) matrix is introduced to account for the mixing of flavour states.
	\begin{equation}
	\begin{pmatrix}d'\\s'\\b'\end{pmatrix} = 
	\begin{pmatrix}
	V_{ud}&&V{us}&&V{ub} \\
	V_{cd}&&V{cs}&&V{cb} \\
	V_{td}&&V{ts}&&V{tb}
	\end{pmatrix}
	\begin{pmatrix}d\\s\\b\end{pmatrix}
	\end{equation}
	The probability that a quark of type $i$ will transform into a quark of type $j$ when it emits a $W^\pm$ is then calculable via $|V_{ij}|^2$. Because of the mixing of states not only a violation of parity but also CP-violation (charge and parity) is observed.
	\subsection{Electroweak Unification}
	\label{sec:elw}	
	To get a better understanding of the forces in the SM, it would be a large success to lead the different forces back to one fundamental mechanism. A first step towards a unification of all three interactions can be archived by combining QED with the weak force. Electromagnetic charge $Q$ and weak isospin $T_3$ are combined to achieve a charge for the electroweak unification, called hypercharge $Y$:
	\begin{equation}
	Y = 2(Q-I_3).
	\end{equation}
	Again, the Lagrange density is required to be invariant under transformations of the underlying $SU(2) \times U(1)$ group. This results into four new massless bosons $W_1$, $W_2$, $W_3$ and $B$. 
	\subsection{Higgs Mechanism and Spontaneous Symmetry Breaking}
	\label{sec:higgs}
	The bosons, $W_1$, $W_2$, $W_3$ and $B$, found above in section \ref{sec:elw} are massless. In contradiction to this, the masses of the $W$ and $Z$ bosons are measured to values of $80\;\text{GeV}$ and $91\;\text{GeV}$, respectively \cite{pdg2016}. Therefore, a theory was developed which solves this problem via spontaneous symmetry breaking. A field is introduced where the Lagrange density for local variations is not symmetric while the Lagrangian of the field itself is. This leads to mass terms for electro-weak bosons and a new boson - the Higgs boson - in the standard model Lagrange density. After spontaneous symmetry breaking, $W_1$ and $W_2$ are combined to get the $W^\pm$ bosons of the weak interaction via Eq. \ref{eq:W}.
	\begin{equation}
	W^\pm = \frac{1}{\sqrt{2}}(W_1 \mp i W_2)
	\label{eq:W}
	\end{equation}
	The other two fields $W_3$ and $B$ mix and result in the $Z^0$ boson from the weak and the photon $\gamma$ of the electromagnetic interaction:
	\begin{equation}
	\begin{pmatrix}\gamma \\ Z^0\end{pmatrix} =
	\begin{pmatrix}
	\cos \theta_W && \sin \theta_W \\
	-\sin \theta_W && \cos \theta_W 
	\end{pmatrix}
	\begin{pmatrix}B \\ W_3\end{pmatrix}
	\label{eq:Z}
	\end{equation}	
	with the weak mixing angle $\theta_W$, which also relates the masses $M_W$ and $M_Z$ of $W^\pm$ and $Z^0$ bosons via:
	\begin{equation}
	M_Z = \frac{M_W}{\cos \theta_W}
	\end{equation}
	Finally also fermion masses can be included in this mechanism. Therefore, a Yukawa coupling to the Higgs field proportional to their masses is introduced. 
	
	
\section{Top Quark}
	The top quark is an up type quark belonging to the third generation of the standard model and carrying electromagnetic charge of $+\frac{2}{3}e$ \cite{pdg2016}. With its mass of about $173\;\text{GeV}$ the top quark is the heaviest particle in the standard model. It was predicted in 1973 \cite{topPredict} to account for observed CP-violation that could not be explained with only the two known quark generations. After finding the much lighter bottom quark only a few years after the prediction, the top quark was discovered in 1995 by the CDF \cite{topCDF} and D\O{} \cite{topD0} collaborations at Tevatron. Because of its great mass, it offers a large phase space for decays and has a short life time of approximately $5 \times 10^{-25}\;\text{s}$ \cite{pdg2016}. Because of the short life time the top quark does not form bound hadronic states and thus, measurements of the bare quark are possible. This provides a special access to parameters of the standard model. Especially the top quark mass is an essential parameter to check the theory for consistency. For instance, the top quark has, due to its mass, a large coupling to the Higgs boson. Therefore, it has to be included in correction terms for the Higgs boson mass. This leads to a relation of the masses of top quark, Higgs boson and electro-weak bosons that can be verified with a well known top quark mass. Additionally, it is important for searches of physics behind the standard model to reach a better understanding of the top quark since it is often part of the final state and/or a dominant background. 
	\subsection{Production in Hadron Colliders and Decay}
	In hadron colliders, the production of a $t\bar{t}$ pair happens via $q\bar{q}$ annihilation or gluon fusion. At the centre-of-mass energy of $13\;\text{TeV}$ reached with LHC, gluon fusion is by far the dominant process. Top quarks can also be produced in single production, but this has, being a electroweak process, a much smaller cross section. Hence, this analysis will focus on pair produced top quarks and will treat single top production as a background process.

	\begin{figure}
		\centering
		\begin{subfigure}{.4\textwidth}
		\begin{tikzpicture}
		\begin{feynman}	
		\vertex(a);
		\vertex[right=of a] (b);
		\vertex[below left=of a] (i1){\(q\)};
		\vertex[above left=of a] (i2){\(\overline q\)};		
		\vertex[below right=of b] (f1){\(\overline t\)};				
		\vertex[above right=of b] (f2){\(t\)};		
		\diagram* {
		(a)-- [gluon,edge label'=\(g\)] (b),
		(i1)-- [fermion] (a) -- [fermion] (i2),
		(f1)-- [fermion] (b) -- [fermion] (f2),
		};
		\end{feynman}
		\end{tikzpicture}
		\caption{}
		\end{subfigure}
		\begin{subfigure}{.4\textwidth}
		\begin{tikzpicture}
		\begin{feynman}	
		\vertex(a);
		\vertex[right=of a] (b);
		\vertex[below left=of a] (i1){\(g\)};
		\vertex[above left=of a] (i2){\(g\)};		
		\vertex[below right=of b] (f1){\(\overline t\)};				
		\vertex[above right=of b] (f2){\(t\)};		
		\diagram* {
		(a)-- [gluon,edge label'=\(g\)] (b),
		(i1)-- [gluon] (a) -- [gluon] (i2),
		(f1)-- [fermion] (b) -- [fermion] (f2),
		};
		\end{feynman}
		\end{tikzpicture}
		\caption{}
		\end{subfigure}		
		\begin{subfigure}{.4\textwidth}
		\begin{tikzpicture}
		\begin{feynman}	
		\vertex(a);
		\vertex[below=of a] (b);
		\vertex[left=of a] (i1){\(g\)};
		\vertex[left=of b] (i2){\(g\)};		
		\vertex[right=of a] (f1){\(t\)};				
		\vertex[right=of b] (f2){\(\overline{t}\)};		
		\diagram* {
		(b)-- [fermion] (a),
		(i1)-- [gluon] (a) -- [fermion] (f1),
		(i2)-- [gluon] (b) -- [anti fermion] (f2),
		};
		\end{feynman}
		\end{tikzpicture}
		\caption{}
		\end{subfigure}
		\begin{subfigure}{.4\textwidth}
		\begin{tikzpicture}
		\begin{feynman}	
		\vertex(a);
		\vertex[below=of a] (b);
		\vertex[left=of a] (i1){\(g\)};
		\vertex[left=of b] (i2){\(g\)};		
		\vertex[right=of a] (f1){\(t\)};				
		\vertex[right=of b] (f2){\(\overline{t}\)};		
		\diagram* {
		(b)-- [fermion] (a),
		(i1)-- [gluon] (b) -- [anti fermion] (f2),
		(i2)-- [gluon] (a) -- [fermion] (f1),
		};
		\end{feynman}
		\end{tikzpicture}
		\caption{}
		\end{subfigure}
		\caption{Feynman diagrams showing the production of top quark pairs. Displayed are quark anti-quark annihilation (a), and gluon-gluon fusion in the s-channel (b), t-channel  (c) and u-channel (d). Created with \cite{feynman}.}
		\label{fig:production}
	\end{figure}	

	The top quark decays via the weak interaction with a probability of almost $100\%$ into a bottom quark and a $W$ boson. This property arises from the entry in the CKM matrix $|V_{tb}|^2 \approx 0.998$ \cite{pdg2016}. While the bottom quark is seen as a jet in the detector, the $W^\pm$ boson further decays into a quark anti-quark pair (see fig. \ref{fig:decaya}) or into a lepton (anti-lepton) and an anti-neutrino (neutrino) (see fig. \ref{fig:decayb}). Looking at the $t\bar{t}$ production, these two possible final states for each top quark corresponds to three channels for the $t\bar{t}$ process. 
	\begin{itemize}
	\item both top quarks decay into quarks (full hadronic)
	\item one top quark decays hadronically, the other one leptonically (lepton+jets)
	\item both top quarks decay leptonically (dilepton)
	\end{itemize}
	The full hadronic and lepton+jets channels are dominant and occur $45.7\%$ respectively $43.8\%$ of the time. $10.5\%$ of all $t\bar{t}$ events result in two leptons in the final state \cite{pdg2016}. This analysis will focus on the lepton+jets channel which is pictured in figure \ref{fig:semilep}. It is suitable because the lepton makes it easier to distinguish $t\bar{t}$ events from background events but also includes a fully hadronically decaying top quark which will be the target for the presented measurement.
	%todo wirklich nur muon channel?
	
	\begin{figure}
	\begin{subfigure}{.5\textwidth}
		\begin{tikzpicture}
		\begin{feynman}
		\vertex(a) {\(t\)};
		\vertex[right=of a] (b);
		\vertex[above right=1.5cm of b] (c);
		\vertex[below right=3cm of b] (f1){\(b\)};
		\vertex[below right=1.5cm of c] (f3){\(\overline{l}\)};
		\vertex[above right=1.5cm of c] (f2){\(\nu_l\)};

		\diagram* {
		(a)-- [fermion] (b)-- [fermion] (f1),
		(b)-- [boson,edge label'=\(W\)] (c),
		(c)-- [fermion] (f2),
		(c)-- [anti fermion] (f3),};
		\end{feynman}
		\end{tikzpicture}
		\caption{}
		\label{fig:decaya}
	\end{subfigure}
	\begin{subfigure}{.5\textwidth}
		\begin{tikzpicture}
		\begin{feynman}
		\vertex(a) {\(t\)};
		\vertex[right=of a] (b);
		\vertex[below right=3cm of b] (f1){\(b\)};
		\vertex[above right=of b] (c);
		\vertex[above right=1.5cm of c] (f2){\(q'\)};
		\vertex[below right=1.5cm of c] (f3){\(\overline{q}\)};
		\diagram* {
		(a)-- [fermion] (b)-- [fermion] (f1),
		(b)-- [boson,edge label'=\(W\)] (c),
		(c)-- [fermion] (f2),
		(c)-- [anti fermion] (f3),};
		\end{feynman}
		\end{tikzpicture}
		\caption{}
	\end{subfigure}
	\caption{Feynman diagrams of a leptonically (a) and hadronically (b) decaying top quark. Decays of anti top quarks are constructed similarly. Created with \cite{feynman}.}
	\label{fig:decayb}
	\end{figure}
	
	\begin{figure}
		\centering
		\begin{tikzpicture}
		\begin{feynman}	
		\vertex[blob] (a) {};
		%\vertex (a);
		\vertex[right=of a] (b);
		\vertex[left=of a] (c);
		\vertex[above right=3.5cm of b] (bb){\(b\)};
		\vertex[right=of b] (bW);
		\vertex[left=of c] (cW);
		\vertex[below left=3.5cm of c] (cb){\(\overline b\)};
		\vertex[above right=of bW] (bq){\(q'\)};		
		\vertex[below right=of bW] (baq){\(\overline q\)};				
		\vertex[above left=of cW] (cl){\(l\)};		
		\vertex[below left=of cW] (cnu){\(\overline \nu_l\)};
		\diagram* {
		(a)-- [fermion,edge label'=\(t\)] (b),
		(a)-- [anti fermion,edge label'=\(\overline t\)] (c),
		(b)-- [fermion] (bb),
		(b)-- [boson,edge label'=\(W^+\)] (bW),
		(c)-- [anti fermion] (cb),
		(c)-- [boson,edge label'=\(W^-\)] (cW),
		(bW)-- [fermion] (bq),
		(bW)-- [anti fermion] (baq),
		(cW)-- [fermion] (cl),
		(cW)-- [anti fermion] (cnu),
		};
		\end{feynman}
		\end{tikzpicture}
		\caption{Feynman diagram displaying an event from the lepton+jets channel. The centred circle indicates a mechanism to produce (see \ref{fig:production}) a $t\bar{t}$ pair. Created with \cite{feynman}.}
		\label{fig:semilep}
	\end{figure}
	\subsection{Mass Measurements}
	Measuring the top quark mass is usually performed via a template fit. Therefore $t\bar{t}$ events are simulated for different masses $m_\text{top}^{MC}$. Then a selection is applied to data and simulation with the goal to select preferably $t\bar{t}$ events. Finally the different simulation with different $m_\text{top}^{MC}$ are fitted to data. The most precise mass measurement of this kind is archived combining results of the CMS, ATLAS, CDF, and D0 collaborations. The result of this world average is a value of $173.34 \pm 0.27 (\text{stat}) \pm 0.71 (\text{sys})\;\text{GeV}$ \cite{topmass_combination}. Doing these kinds of measurements one relies on a correct simulation of $t\bar{t}$ events. Especially the mass parameter in simulation samples can not be easily related to a mass one would calculate in a Lagrange density because of the way simulations are computed. Here, not all effects of QCD can be calculated. This is why a cut-off scale is defined which directly influences the mass associated with a given Simulation (see section \ref{sec:Simulation}). Because of this difficulty, this analysis aims to provide a mass measurement without relying on ambiguities of the mass parameter in simulation.
	