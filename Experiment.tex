\chapter{Experiment}
\section{Large Hadron Collider}
	The Large Hadron Collider (LHC) is a circular particle collider with a circumference of $27\;\text{km}$. 
	
	Four Experiments are placed at each interaction point of LHC.
	% LHC FAKTEN
	% ANDERE EXPERIMENTE
\section{CMS Detector}
\label{sec:cms}
	The 'Compact Muon Solenoid' (CMS) experiment is one of the multi purpose detectors at LHC. 
	% FULL VIEW OF CMS?
	% AUSMAßE
	% ZWIEBELFORM, ZYLINDER
	\begin{figure}[tb]
		\centering
		\includegraphics [width=.8\textwidth]{../Plots/CMS_Slice.png}
		\caption{Slice through the CMS detector looking in direction of the beam pipe \cite{CMSslice}. Tracks of different particles on their way through the layers of the detector are displayed. From left to right, you can see the tracker, calorimeters, the superconducting solenoid and the muon system. These detector components, their purpose and how they work are described in the following sections \ref{sec:tracker} - \ref{sec:muonsystem}.}
		\label{CMS}
	\end{figure}
	
\subsection{Tracker}
\label{sec:tracker}
	%ABMAßE TRACKER, ANZAHL LAYER USW
	The purpose of the tracking system is to measure the momentum of created particles. It is taken advantage of the Lorentz force which changes the momentum of charged particles in a magnetic field. Paths of those particles are therefore bended with a bending radius proportional to the momentum in a constant magnetic field. Therefore a solenoid provides a constant magnetic field of about $4\;\text{T}$ inside the tracker. Per path one spatial point per layer is measured. These points are the input for a track finding algorithm.
	%ALGORITHMUS ANSPRECHEN
	%GENAUIGKEIT

\subsection{Calorimeters}
	To define the type of particle one has to measure not only momentum but also energy. Therefore outside the tracker are two types of calorimeters installed.
\subsubsection{Electromagnetic Calorimeter}
	%GENAUIGKEIT
\subsubsection{Hadronic Calorimeter}
	%GENAUIGKEIT
	
\subsection{Solenoid}
	Outside the hadronic calorimeter a superconducting solenoid is installed and provides a magnetic field of about $4\;\text{T}$ inside the tracking system. The solenoid has a length of $13\;\text{m}$ and a diameter of $6\;\text{m}$. The purpose of the magnetic field is to bend paths of charged particles due to the Lorentz force. The bending radius then is directly connected to the momentum of the particle.  
\subsection{Muon System}
\label{sec:muonsystem}
	Because muons are not absorbed in the calorimeters, it is possible to use an additional tracking detector for muons at the very outside of the detector. Muon chambers are embedded in the iron return yoke of the magnet. Thus a magnetic field of about $2\;\text{T}$ is present.
	%AUFBAU MUON SYSTEM
	In cooperation with the inner tracker it is possible to reconstruct muon momenta with high precision.
\subsection{Trigger}
	At the interaction points, proton bunches are brought to collision every $25\;\text{ns}$. Per crossing about ????
	%INTERACTIONNUMBER
	interactions take place. This adds up to about ???
	%COLLISIONS PER SECOND
	interactions per second. It is impossible to store data with this rate and the total required storage capacity would is not feasible either. Additionally it would need a tremendous number of computing cores to analyse this data. Therefore a trigger system is installed to select only the interesting events to reduce the amount of data. For the computation system it is required to reduce the amount from ???? to ????.
	%RATE VOR UND NACH TRIGGER
 	To decide which events are worth storing and which will be neglected different criteria are defined.
 	% TRIGGER MENU
 	% DIFFERENT TRIGGER LEVELS
	
	