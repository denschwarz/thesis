\chapter{Experiment}
\label{ch:Exp}
	Physicists use scattering experiments to dissolve processes at high energy scales to understand the basic building blocks and laws of nature. For this analysis, data from the currently worlds largest scattering experiment LHC is used. Here, protons are brought to collision in a circular particle accelerator. The outcome of the collisions is then measured with several detection systems and is used to test the current status of the standard model. A description of the accelerator (section \ref{sec:lhc}) and the detector (section \ref{sec:cms}) is presented in this chapter.
\section{Large Hadron Collider}
\label{sec:lhc}
	The Large Hadron Collider (LHC) is a circular particle collider with a circumference of $26.7\;\text{km}$ operating at CERN\footnote{Conseil europ\'{e}en pour la recherche nucl\'{e}aire}, Switzerland. At the LHC, protons, or in special runs heavy ions, are accelerated in opposite directions along the ring to perform scattering experiments at high energies. Protons are accelerated to an energy of $6.5\;\text{TeV}$ and then brought to collision with a resulting center-of-mass energy of $\sqrt{s}=13\;\text{TeV}$, almost reaching the design energy of $14\;\text{TeV}$. Protons are taken from a hydrogen source and then accelerated via superconducting cavities in various linear and circular accelerators before injected into the LHC ring. Here, protons are organised in bunches with approximately $100$ billion protons per bunch. There are $2808$ spaces for bunches in the LHC ring with a separation of about $25\;\text{ns}$. In all accelerators, magnet systems provide a handle to lead and form the proton beams. Most importantly, dipole magnets are used to keep the protons on their required orbit. Quadrupole magnets are installed to focus the beam, while higher order magnets are used to correct for higher order effects like energy dependent deviations from the orbit. Four experiments are placed at interactions point of LHC. Besides CMS\footnote{Compact Muon Solenoid}, which is described below in section \ref{sec:cms}, there is another multi-purpose detector with a broad physics program, called ATLAS\footnote{A Toroidal LHC Apparatus}. Additionally there are two more specialised experiments: LHCb\footnote{Large Hadron Collider beauty}, with a focus on $b$-physics and ALICE\footnote{A Large Ion Collider Experiment}, mainly aiming at research of quark-gluon plasma in heavy ion collisions. A display of the LHC complex with its four experiments and preaccelerators is shown in fig. \ref{fig:lhc}.
		\begin{figure}
			\centering
			\includegraphics [width=\textwidth]{../Images/lhc.jpg}
			\caption{Display of the LHC complex with the LHC ring itself and smaller accelerators used to accelerate protons to the required energy for injection in the LHC ring. Taken from \cite{lhc}.}
			\label{fig:lhc}
		\end{figure}
		
	%todo Lumi Abschnitt in pysics of proton-proton... ?
	A very important parameter of a particle collider is its luminosity \cite{luminosity}. It gives an estimate how many collisions take place per area and second, thus, how well the beam is focused at the interaction points. It is calculated via Eq. \ref{eq:lumi}. 
	\begin{equation}
	L = \frac{n N_1 N_2 f}{4 \pi \sigma_x \sigma_y}
	\label{eq:lumi}
	\end{equation} 
	In this formula $n$ denotes the number of bunches in the accelerator, $N_1$ and $N_2$ are the number of protons in the two colliding bunches and $f$ is the collision frequency. The denominator gives the cross sectional area where $\sigma_x$ and $\sigma_y$ describes the spread of the proton beam in $x$ and $y$ direction, respectively. The design luminosity of LHC is $10^{34}\;\text{cm}^{-2}\text{s}^{-1}$ but was already exceeded in the 2016 run. By multiplying the integrated luminosity (see Eq. \ref{eq:intlumi}) with the production cross section of a particular process $\sigma$, one gets an estimate how many events $N$ to expect in one second (see Eq. \ref{eq:number}).
	\begin{equation}
	L_\text{int} = \int L dt
	\label{eq:intlumi}
	\end{equation} 
	\begin{equation}
	N = L_\text{int} \sigma
	\label{eq:number}
	\end{equation} 
	LHC has performed very well and produced a lot of collision data up to now. The total integrated luminosity for every year can be read off from Fig.\ref{fig:LHClumi}.
	\begin{figure}[tb]
		\centering
		\includegraphics [width=.8\textwidth]{../Plots/LHC_Lumi.png}
		\caption{Luminosity of the data sets produced at LHC. The graphs show the development and size of data sets from 2010 (green), 2011 (red), 2012 (blue), 2015 (purple) and 2016 (orange). Taken from \cite{LHClumi}.}
		\label{fig:LHClumi}
	\end{figure}
\subsection{Physics of proton-proton collisions}
	In collision experiments, the center-of-mass energy $\sqrt{s}$ defines the energy that is available to create particles and give them kinetic energy. At LHC, $s$ is set by the momenta of incoming protons via
	\begin{eqnarray}
	s &=& (p_1 + p_2)^2 \\
	  &=& p_1^2 + p_2^2 + 2 p_1 p_2 \\
	  &=& (E_1^2 - |\vec{p}_1|^2 )  + (E_2^2 - |\vec{p}_2|^2 ) + 2 (E_1 E_2 - |\vec{p}_1| |\vec{p}_2| ),
	\end{eqnarray}
	where $p_1$ and $p_2$ are the four-momenta of two colliding protons with energies $E$ and momentum vector $\vec{p}$. Because the two beams have the same energy but opposite directions, the assumptions $E_1 = E_2 = E_\text{proton}$ and $|\vec{p}_1| = |\vec{p}_2|$ hold. With that, the center-of-mass energy reads:
	\begin{equation}
	\sqrt{s} = 2 E_\text{proton}.
	\end{equation}
	Because protons are composite particles, the actual scattering involves quarks or gluons, called partons, carrying only a fraction of the nominal beam momentum $x p_\text{proton}$. The center-of-mass energy of the hard scattering process $\sqrt{\hat{s}}$ is therefore only a fraction of the stated $13\;\text{TeV}$:
	\begin{equation}
	\sqrt{\hat{s}} = \frac{(x_1 + x_2)}{2} \sqrt{s}
	\end{equation} 
	Here, $x_1$ and $x_2$ indicate the fraction of momentum two colliding partons carry. Because this fraction is not known, the center-of-mass energy of the hard process is not either. Therefore, for data analysis one has to use variables, that do not depend on the initial state of the partons which are explained in section \ref{sec:coordinate}.
	%todo pdf erwaehnen??
	
\subsubsection{Pile-Up}
	The high rate of collisions of LHC is required to collect a large amount of data and be able to see even very rare processes. But the high rate also has its downside. Since multiple proton-proton interactions take place per bunch crossing, not only one interesting but several other scattering events, mostly soft QCD processes, are seen simultaneously in the detector. This effect is called pile-up. If not corrected for, energy measurements do always include particles not originating from the hard scattering one is interested in. To reduce pile-up effects, it is crucial to be able to resolve different primary vertices belonging to different interactions.
\subsubsection{Underlying Event}
	As result of LHC doing collision experiments with protons, thus composite particles, it has to be accounted for more than one scattering process in a proton-proton collision. This effect is called underlying event and addresses collisions where multiple partons from two colliding protons interact. Since both interactions share the same interaction point, they can hardly be separated. Hence, underlying event has to be understood and is included into simulation of events.
	
\section{CMS Detector}
\label{sec:cms}
	The 'Compact Muon Solenoid' (CMS) experiment is a multi purpose detector at LHC. It is designed to measure momentum and energy of particles produced in proton-proton interactions. With a total weight of $14000\;\text{t}$, the mass of the CMS detector is dominated by a steel return yoke installed to lead the magnetic field originating from a solenoid. The detector has a cylindrical shape with a length of $28.7\;\text{m}$ and a diameter of $15.0\;\text{m}$. The CMS detector is built in onion-like layers of subdetectors with different purposes described in the following sections. To identify and distinguish particles, the fact that different particles leave different signatures in the detector is utilised. How particles are reconstructed is detailed in chapter \ref{ch:Reco}. Besides the detector systems, a very important component is the Trigger, described in section \ref{sec:trigger}, providing fast decisions if an event is discarded or interesting enough to be stored. The design of CMS was chosen to cover a wide range of physics approaches. Nevertheless, a focus was set on the discovery of the Higgs boson, which was announced in 2012. Components, design and the general construction of the CMS detector are shown in Fig. \ref{fig:CMS}.  
	\begin{figure}[htb]
		\centering
		\includegraphics [width=.95\textwidth]{../Images/CMS_Full.png}
		\caption{Full view of the CMS detector. Basic properties as weight and dimensions are summarised in a table in the left upper corner. Components of the detector are titled and shown in different colours. Taken from \cite{CMSfull}.}
		\label{fig:CMS}
	\end{figure}
	
\subsection{Coordinate System}
\label{sec:coordinate}
	The coordinate system used in the CMS experiment is based on cartesian and right-handed coordinates. The origin is set in the center of the CMS detector. To define the direction of the axes other fix points are set. The $x$-axis point in the direction of the center of the LHC ring, the $y$-axis points up and the $z$-axis is defined parallel to the beam axis. Important variables used in analysis of CMS data are the angles $\phi$ and $\theta$. $\phi$ is defined as the angle in the $x$-$y$-plane measured from the $x$-axis and $\theta$ describes the angle from a given point to the beam axis. Because the LHC is a hadron collider and physical events are therefore not symmetric in $\theta$ it is useful to construct the Lorentz invariant variable $\eta$:
	\begin{equation}
	\eta = - \ln \left[\tan\left( \frac{\theta}{2}\right) \right]
	\end{equation} 
	The distance $\Delta R$ between two objects $i$ and $j$ is calculated using the differences $\Delta \phi = \phi_i - \phi_j$ and $\Delta \eta = \eta_i - \eta_j$:
	\begin{equation}
	\Delta R = \sqrt{\Delta \phi ^2 + \Delta \eta ^2}
	\end{equation}
	An important quantity used in this analysis is the transversal momentum $p_T$ which is constructed out of the $x$ and $y$-components ($p_x$ and $p_y$) of the total momentum of an object:
	\begin{equation}
	p_T = \sqrt{p_x^2 + p_y^2}
	\end{equation} 
	It is practical to not consider the $z$ component in a hadron collider because it depends on the initial state of interacting partons which is unknown. The $p_T$ sum of all objects is expected to vanish every event. If it is not, the $p_T$ may be reconstructed wrong for some objects or objects left the CMS experiment undetected.
	
\subsection{Solenoid}
	The CMS detector uses a magnetic field to bend paths of electromagnetic charged particles. With the resulting curvature radius and a known magnetic field strength, one is able to calculate the momenta of those particles. Therefore, a superconducting solenoid is installed to provide a magnetic field of $3.8\;\text{T}$ inside the tracking system (explained in section \ref{sec:tracker}). The solenoid has a length of $12.9\;\text{m}$ and a diameter of $5.9\;\text{m}$. Tracker and calorimeters are placed inside the solenoid, the muon system is installed around the magnet inside iron return yokes. The iron yokes are used to guide the magnetic flux outside the solenoid. The magnetic fields purpose on the outside is to bend tracks of muons inside the muon system. A solenoid form was chosen because it provides a constant magnetic flux inside the tracking system.
	
\subsection{Tracker}
\label{sec:tracker}
	The purpose of the tracking system \cite{CMSdetector} is to measure the momentum and charge of particles. It is taken advantage of the Lorentz force which changes the momentum of charged particles in a magnetic field. Paths of those particles are therefore bended with a bending radius proportional to the momentum in a constant magnetic field. The tracking system is installed at the innermost of the detector and covers, with a length of about $5.4\;\text{m}$ and radius of $1.1\;\text{m}$, a range of $| \eta | < 2.5$. The full tracker is constructed in multiple layers and subsystems. A view of the tracking system is presented in Fig. \ref{fig:tracker}, all systems are described below. Per path of a particle one spatial point per layer is measured. These points are the input for a track finding algorithm. The largest uncertainties of the measured momenta rise from the spatial resolution of the tracker. Additionally the reconstruction depends on the momentum of the particle. If it gets too large, the tracks are barely bended and the bending radius cannot be measured with high precision. Accordingly, the uncertainty of a measurement grows with momentum of a particle. Moreover, tracker information is used to reconstruct primary and secondary vertices. A reliable detection of secondary vertices is crucial for b-tagging, which is explained in section $\ref{sec:btag}$.
	\begin{figure}[tb]
			\centering
			\includegraphics [width=\textwidth]{../Images/Tracker.png}
			\caption{Sketch of the CMS tracking detector. All subsystems namely pixel and strip detector with inner barrel (TIB), outer barrel (TOB), inner endcaps (TID) and outer endcaps (TEC) are shown. Additionally the range in $\eta$, radius ($r$) and $z$-direction can be read off. Taken from \cite{CMSdetector}.}
			\label{fig:tracker}
	\end{figure}
	\\
	The innermost tracking system is a pixel detector consisting of three layers in the barrel region with a distance to the beam axis of $4.4\;\text{cm}$, $7.3\;\text{cm}$ and $10.2\;\text{cm}$, respectively. Additionally, two endcap discs are installed. Each of the $66$ million silicon pixels extends to $100\;\text{\textmu m} \times 150\;\text{\textmu m}$. With a thought through arrangement of pixels a resolution of $10\;\text{\textmu m}$ in the $r-\phi$-plane and $20\;\text{\textmu m}$ in $z$-direction is reached. Outside the pixel detector a silicon strip tracker is built. It is further divided into four subsystems: an inner barrel (TIB) with four layers, an outer barrel (TOB) with six layers, two inner endcaps (TID) consisting of three discs and two outer endcaps (TEC).

\subsection{Calorimeters}
	To define the type of a particle, one has to measure not only momentum but also energy. Therefore, outside the tracker a colorimetry system with two types of calorimeters is installed. Additionally, showers seen in the calorimeters are used as seed for track finding algorithms and particle flow (section \ref{sec:pf}). The underlying principle of this subdetector is to absorb all of the energy a particle carries and convert it to a measurable quantity. Energy deposited in the calorimeter is then transformed into a light signal using scintillating materials. The intensity of the light signal is proportional to the energy of the incoming particle. The material also has to be dense enough to stop an incoming particle entirely and transform all its energy into the light signal. As a measure for calorimeter thickness, the radiation length $X_O$ for electrons is defined as the path where an electron looses $\frac{1}{e}$ of its initial energy. Analogously, a nuclear interaction length $\lambda_n$ for hadrons is introduced, taking strong interaction into account. Of course the value of both, $X_0$ and $\lambda_n$, depends on the used material. Finally, the light is collected and measured with photo multipliers and can be converted into an energy value. Measurements from calorimeters typically get more precise with increasing energy. Therefore, they provide a good alternative method in cases where the measurement of the tracking system gets worse. The typical resolution of calorimeters is shown in Eq. \ref{eq:calo}.
	\begin{equation}
	\frac{\sigma_E}{E\left[\text{GeV}\right]} = \frac{a}{\sqrt{E\left[\text{GeV}\right]}} \oplus b \oplus \frac{c}{E\left[\text{GeV}\right]}
	\label{eq:calo}
	\end{equation}
	In this formula $\frac{\sigma_E}{E}$ denotes the relative energy resolution depending on the energy $E$, a statistical term $a$, a constant term $b$ - including systematic errors or shower leakage - and a noise term $c$, containing electronics and pile-up noise. At high energies the $b$-term is dominant but can be reduced by adequate calibration. 
\subsubsection{Electromagnetic Calorimeter}
	The inner part of the calorimetry system, the electromagnetic calorimeter (ECAL), is supposed to measure the energy of electrons and photons in a range up to $|\eta| = 3.0$. It consists of \chem{PbWO_4} crystals and is designed to deliver a very good spatial resolution. The crystals are $230\;\text{mm}$ long, which corresponds to $25.8$ times the radiation length of an electron, and have a face area of $22\;\text{mm} \times 22\;\text{mm}$. With this, photons, which are not seen in the tracker, can be reconstructed with a high precision. Even the direction of photons can be resolved from shower reconstruction. This design choice was driven by the decay of a Higgs boson into two photons ($H\rightarrow \gamma \gamma$) where the Higgs boson mass can be reconstructed from the four-momenta of the photon pair. A performance study presented in \cite{EGammaPerformance} names a resolution of less than $2\%$ for $45\;\text{GeV}$ electrons from $Z$ boson decays in the barrel region.

\subsubsection{Hadronic Calorimeter}
	Since hadrons have a larger absorption length they mostly pass the electromagnetic calorimeter and are absorbed and measured in the hadronic calorimeter (HCAL). Two alternating layers of different materials are used. One is to absorb the energy of an incoming particle and consists of brass plates. The other one is made out of a plastic scintillator, that acts as active material. The barrel part of the HCAL begins in a distance of $r=1.77\;\text{m}$ to the interaction point and reaches up to the solenoid at $r=2.95\;\text{m}$ covering a range of $|\eta| < 1.3$. The thickness corresponds to about ten times the hadronic interaction length. To cover a range of $1.3 < |\eta| < 3$, endcaps are installed. Additionally, forward calorimeters at a distance of $11.1\;\text{m}$ in $z$-direction to the interaction point are used to cover the range of $3 < |\eta| < 5$. To account for and measure leakage, a further outer calorimeter is installed outside the solenoid. 
	%todo GENAUIGKEIT: referenzwert wie bei E= XX ist Aufloesung YY

 
\subsection{Muon System}
\label{sec:muonsystem}
	Because of the possibility to reconstruct muons with very high precision, they play an important role in the design of CMS. Thus, a whole system is installed just to identify and measure muons. Muons at typical energies at LHC are minimal ionizing particles and are therefore not absorbed in the calorimeters. Thus, it is possible to use an additional tracking detector for muons at the very outside of the detector. As a muon tracker various versions of gaseous detectors are installed. In the barrel part, drift tubes and resistive plate chambers are embedded in the iron return yoke of the magnet where a magnetic field of about $2\;\text{T}$ is present. In addition, two endcaps with resistive plate chambers in combination with cathode strip chambers are installed. The design choice is due to the expected flux of muons in the different regions of the detector. While muons occur rarely in the barrel part, they are predicted to appear in high rates in the endcaps.
	%todo Genauigkeit

\subsection{Trigger}
\label{sec:trigger}
	At the interaction points, proton bunches are brought to collision every $25\;\text{ns}$. At the design luminosity, per crossing about $20$ interactions take place. This adds up to $8 \cdot 10^8$ interactions per second.	It is impossible to store that amount of data with this rate and the total required storage capacity is not feasible either. Additionally, it would need a tremendous number of computing cores to analyse this data. Therefore, a trigger system is installed selecting only interesting events to reduce the amount of data. For the computation system it is required to reduce the rate by a factor of the order $10^6$ to a few $100\;\text{Hz}$. This is achieved with two trigger stages: a hardware based Level-1 (L1) Trigger\cite{L1} and a software based High-Level Trigger (HLT)\cite{HLT}. The L1 Trigger is designed to deliver an output of about $30\;\text{kHz}$, the remaining reduction is done by the HLT.	To decide which events are worth storing and which will be neglected different criteria are defined. Fast components of the detector like calorimeter and muon system are read out and deliver information to the trigger system. Then the trigger makes decisions based on this information and stores for example events with high energy muons or jets. CMS then provides a trigger menu where data streams are ordered by the type of object that fired the trigger.

	