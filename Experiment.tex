\chapter{Experiment}
	For this analysis a data from a collision experiment is used. Protons are brought to collision in a circular particle accelerator. The outcome of the collisions is then measured with several detection systems. A description of the accelerator and the detector is presented in this section.
\section{Large Hadron Collider}
	The Large Hadron Collider (LHC) is a circular particle collider with a circumference of $27\;\text{km}$. At the LHC protons are accelerated to an energy of $6.5\;\text{TeV}$ and then brought to collision resulting in a center-of-mass energy of $\sqrt{s}=13\;\text{TeV}$. Because protons are composite particles, the actual collision involves quarks or gluons carrying only a fraction of this energy. 
	
	Protons are not injected directly into the LHC ring but run through several acceleration stages. The preaccelerators used for this purpose are old accelerators that can now be reused.
	
	Four Experiments are placed at each interaction point of LHC.
	
	A display of the LHC complex with its four experiments and preaccelerators is shown in fig. \ref{fig:lhc}.
	
	A very important parameter of a particle collider is its luminosity. It gives an estimate how many collisions take place per area and second. It is calculated via Eq. \ref{eq:lumi}. 
	\begin{equation}
	L = \frac{n N_1 N_2 f}{4 \pi \sigma_x \sigma_y}
	\label{eq:lumi}
	\end{equation} 
	In this formula $n$ denotes the number of bunches in the accelerator, $N_1$ and $N_2$ are the number of protons in the two colliding bunches and $f$ is the collision frequency. The denominator gives the cross sectional area where $\sigma_x$ and $\sigma_y$ describes the spread of the proton beam in $x$ and $y$ direction, respectively. To get an estimate of the number of events produced by a given process one has to multiply the integrated luminosity (see Eq. \ref{eq:intlumi}) with the cross section of this particular process (see Eq. \ref{eq:number})
	\begin{equation}
	L_\text{int} = \int L dt
	\label{eq:intlumi}
	\end{equation} 
	\begin{equation}
	N = L_\text{int} \sigma
	\label{eq:number}
	\end{equation} 
	
	\begin{figure}
		\centering
		\includegraphics [width=\textwidth]{../Images/lhc.jpg}
		\caption{Display the LHC complex with the LHC ring itself and every accelerator used to bring protons to the required energy for injection in the LHC ring \cite{lhc}}
		\label{fig:lhc}
	\end{figure}
	%todo LHC FAKTEN
	%todo Bild von LHC
	%todo Lumi einfuehren und Formel angeben
	%todo ANDERE EXPERIMENTE
\section{CMS Detector}
\label{sec:cms}
	The 'Compact Muon Solenoid' (CMS) experiment is a multi purpose detectors at LHC. It is designed to measure momentum and energy of particles produced in proton-proton interactions. The CMS detector is built in layers of subdetectors with different purposes described in following sections \ref{sec:tracker} - \ref{sec:muonsystem}.
	
	A very important part besides the detector systems is the Trigger, described in section \ref{sec:trigger}, providing fast decisions if an event is discarded or interesting enough to be stored.
	%todo FULL VIEW OF CMS?
	%todo AUSMAßE
	%todo ZWIEBELFORM, ZYLINDER
	%todo PLOT: Collected Lumi von meinem Datensatz!

	\begin{figure}[tb]
		\centering
		\includegraphics [width=.8\textwidth]{../Plots/CMS_Slice.png}
		\caption{Slice through the CMS detector looking in direction of the beam pipe \cite{CMSslice}. Tracks of different particles on their way through the layers of the detector are displayed. From left to right, you can see the tracker, calorimeters, the superconducting solenoid and the muon system. These detector components, their purpose and how they work are described in the following sections \ref{sec:tracker} - \ref{sec:muonsystem}.}
		\label{CMS}
	\end{figure}
	
\subsection{Tracker}
\label{sec:tracker}
	%todo ABMAßE TRACKER, ANZAHL LAYER USW
	The purpose of the tracking system is to measure the momentum of created particles. It is taken advantage of the Lorentz force which changes the momentum of charged particles in a magnetic field. Paths of those particles are therefore bended with a bending radius proportional to the momentum in a constant magnetic field. Therefore a solenoid provides a constant magnetic field of about $4\;\text{T}$ inside the tracker. Per path one spatial point per layer is measured. These points are the input for a track finding algorithm. The largest uncertainties of the measured momenta rise from the spatial resolution of the tracker. Therefore a pixel structure is used which can measure the position of a passing charged particle within a resolution of ????.
	%todo GENAUIGKEIT
	%todo MATERIAL
	%todo ALGORITHMUS ANSPRECHEN
	All measured spatial points are then put into a track finding algorithm returning track and momentum of every recorded object.

\subsection{Calorimeters}
	To define the type of particle one has to measure not only momentum but also energy. Therefore outside the tracker a colorimetry system with two types of calorimeters is installed. The underlying principle of this subdetector is to absorb all of the energy a particle carries. This energy is then transformed into a light signal. The intensity of the light signal is proportional to the energy of the incoming particle. For this approach one needs to use scintillating materials. It also has to be dense enough to stop a incoming particle to transform all its energy into the light signal. At the end the light is collected and measured with photo multipliers.
	%FUNKTIONSWEISE 
\subsubsection{Electromagnetic Calorimeter}
	The inner part of the calorimetry system is supposed to measure the energy of electrons and photons. The electormagnetic calorimeter of the CMS detector is designed to deliver a very good spatial resolution. With this, photons which are not seen in the tracker can be reconstructed with a high precision.
	
	Since hadrons have a larger absorption length they mostly pass the electromagnetic calorimeter and are measured in the hadronic calorimeter.
	%MATERIAL
	%GROESSE
	%GENAUIGKEIT
\subsubsection{Hadronic Calorimeter}
	The hadronic calorimeter uses two alternating layers of different material. One is to absorb the energy of an incoming particle. The other one is made out of a scintillator.
	%GENAUIGKEIT
	%MATERIAL
	%GROESSE
	
\subsection{Solenoid}
	Outside the hadronic calorimeter a superconducting solenoid is installed and provides a magnetic field of about $4\;\text{T}$ inside the tracking system. The solenoid has a length of $13\;\text{m}$ and a diameter of $6\;\text{m}$. The purpose of the magnetic field is to bend paths of charged particles due to the Lorentz force. The bending radius then is directly connected to the momentum of the particle.  
\subsection{Muon System}
\label{sec:muonsystem}
	Because muons are not absorbed in the calorimeters, it is possible to use an additional tracking detector for muons at the very outside of the detector. Muon chambers are embedded in the iron return yoke of the magnet. Thus a magnetic field of about $2\;\text{T}$ is present.
	%AUFBAU MUON SYSTEM
	In cooperation with the inner tracker it is possible to reconstruct muon momenta with high precision.
\subsection{Trigger}
\label{sec:trigger}
	At the interaction points, proton bunches are brought to collision every $25\;\text{ns}$. Per crossing about ????
	%INTERACTIONNUMBER
	interactions take place. This adds up to about ???
	%COLLISIONS PER SECOND
	interactions per second. It is impossible to store data with this rate and the total required storage capacity would is not feasible either. Additionally it would need a tremendous number of computing cores to analyse this data. Therefore a trigger system is installed to select only the interesting events to reduce the amount of data. For the computation system it is required to reduce the amount from ???? to ????.
	%RATE VOR UND NACH TRIGGER
 	To decide which events are worth storing and which will be neglected different criteria are defined.
 	% TRIGGER MENU
 	% DIFFERENT TRIGGER LEVELS
	
	