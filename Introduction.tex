\chapter{Introduction}
	Particle physics has made a huge success of explaining phenomena we observe on macroscopic and microscopic scales. A quantum field theory construct has been developed since the rise of quantum mechanics to describe interaction and properties of elementary particles. The current state of this theory is referred to as standard model and provides one of the most precisely tested model in physics. With this knowledge, structure and characteristics of matter as well as outcomes of experiments with elementary particles can be predicted and understood. On the other hand, the standard model is not able to describe the whole world of physics. It is not yet possible to include gravity, respectively general relativity, in the context of a quantum field theory. Furthermore, even a state of matter, called dark matter, which is experimentally proven to exists is not part of the theory. Therefore, particle colliders like the Large Hadron Collider (LHC) are built to test the current state of theory and search for new physics that is not yet included. For both, it is indispensable to perform precision measurement of parameters of the standard model. In particular to test predictions of theory but also to be able to understand background processes and final states of searches. At a proton-proton collider with a lot of additional activity in an event it is furthermore essential to be able to dissolve the particles that participated in a scattering one is interested in. Thus, reconstruction of objects, especially jets, is an crucial task in LHC analyses. To be able to pull further information out of jets and identify their origin, substructure variables are used. \\
	This analysis addresses all points above by presenting a method of a jet mass measurement in boosted top quark decays. The jet mass is a fundamental parameter to tag and distinguish between heavy objects since the invariant mass of all decay products is equal to the mass of the initial object. For this analysis, jet mass in top quark decays is interesting because of its sensitivity to the top quark mass. By choosing an unfolding approach, not only a top quark mass can be extracted independent from simulation but simulation of jet mass distribution can furthermore be tested by comparing to theory calculations. To be able to obtain the called results, first different jet algorithms have to be examined because of the direct influence of the jet mass. Studies of jet mass distribution from various clustering algorithms as well as first tests of the unfolding procedure are discussed in this analysis.
	\\
	The structure of this work is build as follows. Firstly, an overview of theory aspects, especially the standard model of particle physics is presented in chapter \ref{ch:Theo}. Afterwards an introduction of the mass measurement method and previous analysis results are discussed in chapter \ref{ch:Measure}. A description of the experiment, including the collider LHC and the detector CMS, the reconstruction of physical objects and the underlying data set follows in chapters \ref{ch:Exp}, \ref{ch:Reco} and \ref{ch:MC}, respectively. The analysis itself is presented in chapter \ref{ch:Ana}. Here, selection criteria, jet studies of various clustering algorithms as well as new derived jet energy corrections and first unfolding tests of the jet mass distribution are presented.
	\\
	It is to mention that natural units are used, where speed of light and the Planck constant are set to $1$, resulting in energies as well as momenta and masses measured in electron volts (eV). Additionally, every value of energy and momentum is given in units of GeV, unless otherwise stated.