% !TEX root = Master_Thesis.tex
\chapter{Introduction}
	Quantum field theories have made a huge success of explaining phenomena we observe on macroscopic and microscopic scales. The current state of this theory is referred to as Standard Model~(SM) and provides one of the most precisely tested theories in physics. With this knowledge, structure and characteristics of matter as well as outcomes of experiments with elementary particles can be predicted and understood. However, the Standard Model is not able to describe the whole world of physics. It is not yet possible to include gravity, respectively general relativity, in the context of a quantum field theory. Furthermore, even a state of matter, called dark matter, which is experimentally proven to exists is not part of the theory. Therefore, particle colliders like the Large Hadron Collider~(LHC) are built to test the current state of theory and search for new phenomena as well as new particles that are not yet included. For both, it is indispensable to perform precision measurements of parameters of the Standard Model. In particular, to test predictions of the theory, but also to be able to understand background processes and final states of searches. At a proton-proton collider like the LHC with a lot of additional activity in an event, it is furthermore essential to be able to dissolve the particles that participated in a scattering one is interested in. Thus, reconstruction of objects, especially jets, is a crucial task in LHC analyses. With a centre-of-mass energy of $13\;\text{TeV}$, heavy objects are often Lorentz boosted and their decay products merge into ine jet. Therefore, substructure variables are essential to be able to pull further information out of jets and identify their origin. Top quarks play a special role and have to be identified with high precision for Standard Model tests as well as searches for new particles. This is due to the large mass, providing a large coupling to the Higgs boson, but also to new particles via Yukawa coupling. \\	
	This analysis addresses all points above by presenting a method of a jet mass measurement in boosted top quark decays. The jet mass is a fundamental parameter to identify heavy objects since the invariant mass of all decay products is equal to the mass of the initial object. For this analysis, the jet mass in top quark decays is interesting because of its sensitivity to the top quark mass. By choosing an unfolding approach, not only a top quark mass can be extracted independent from simulation but simulation of jet mass distribution can furthermore be tested by comparing to theory calculations. To be able to obtain the called results, different jet algorithms have to be examined because of the direct influence on the jet mass. Studies of jet mass distributions from various clustering algorithms as well as first tests of the unfolding procedure are discussed in this analysis.
	\\
	The structure of this work is build as follows: Firstly, an overview of theory aspects, especially the Standard Model of particle physics, is presented. Afterwards an introduction of the mass measurement method and previous analysis results are discussed in chapter~\ref{ch:Measure}. A description of the experiment, including the Large Hadron Collider and the Compact Muon Solenoid~(CMS) experiment, the reconstruction of physical objects and the underlying dataset follows in chapters~\ref{ch:Exp}, ~\ref{ch:Reco} and ~\ref{ch:MC}, respectively. The analysis itself is presented in chapter~\ref{ch:Ana}. Here, selection criteria, jet studies of various clustering algorithms as well as new derived jet energy corrections, unfolding tests of the jet mass distribution as well as unfolded data are presented.
	\\
	Note that natural units are used throughout this thesis, where speed of light and the Planck constant are set to unity, resulting in energies, momenta and masses measured in electron volts~(eV). Additionally, every value of energy and momentum is given in units of GeV, unless otherwise stated.