\chapter{Reconstruction of Objects}
	From all kinds of information provided by the detector systems has to be interpreted as physical objects in order to analyse the recorded data. Therefore algorithms are run to link the information from the detector to usable objects like muons, electrons or jets. Objects are defined by their detection in different detector systems. Paths of all detectable objects through the detector are sketched in Fig. \ref{fig:CMS_reco}.
		\begin{figure}[tb]
			\centering
			\includegraphics [width=.8\textwidth]{../Plots/CMS_Slice.png}
			\caption{Slice through the CMS detector looking in direction of the beam pipe \cite{CMSslice}. Tracks of different particles on their way through the layers of the detector are displayed. From left to right, you can see the tracker, calorimeters, the superconducting solenoid and the muon system.}
			\label{fig:CMS_reco}
		\end{figure} 
\section{Coordinate System}
	The coordinate system used in the CMS experiment is based on cartesian and right-handed coordinates. The origin is set in the center of the CMS detector. To define the direction of the axes other fix points are set. The $x$-axis point in the direction of the center of the LHC ring, the $y$-axis points up and the $z$-axis is defined parallel to the beam axis. Important variables used in analysis of CMS data are the angles $\phi$ and $\theta$. $\phi$ is defined as the angle in the $x$-$y$-plane measured from the $x$-axis and $\theta$ describes the angle from a given point to the beam axis. Because the LHC is a hadron collider and physical events are therefore not symmetric in $\theta$ it is useful to construct the Lorentz invariant variable $\eta$:
	
	\begin{equation}
	\eta = - \ln \left[\tan\left( \frac{\theta}{2}\right) \right]
	\end{equation} 

	\noindent The distance $\Delta R$ between two objects $i$ and $j$ is calculated using the differences $\Delta \phi = \phi_i - \phi_j$ and $\Delta \eta = \eta_i - \eta_j$:
	
	\begin{equation}
	\Delta R = \sqrt{\Delta \phi ^2 + \Delta \eta ^2}
	\end{equation}
	%ERKLÄREN WARUM DELTA R GEEIGNET IST? -> IN PHI UND ETA ABHÄNGIG VON PT, NICHT GESAMTIMPULS!
	\noindent An important quantity used in this analysis is the transversal momentum $p_T$ which is constructed out of the $x$ and $y$-components ($p_x$ and $p_y$) of the total momentum of an object:
	
	\begin{equation}
	p_T = \sqrt{p_x^2 + p_y^2}
	\end{equation} 

	\noindent It is practical to not consider the $z$ component in a hadron collider because it depends on the initial state of interacting partons which is unknown. The $p_T$ sum of all objects is expected to be $0$ in every event. If it is not the $p_T$ may be reconstructed wrong for some objects or objects left the CMS experiment undetected.
\section{Particle Flow}
	CMS uses a special algorithm called particle flow \cite{particleflow} to combine information from tracker and calorimeters and thus reconstruct objects to a high precision. Tracks from the inner tracker are extrapolated into the calorimeters. If a shower fits to the track, information from these two subdetectors are combined. Since only charged particles will interact with the tracking system, showers from neutral hadrons and photons cannot be associated with a track.
\section{Electrons}
	Electrons are reconstructed in the tracker and electromagnetic calorimeter. Due to the magnetic field, electrons will take a curved path through the tracking system.
	%todo REICHT BAHNRADIUS SCHON FÜR IDENTIFIKATION?
\section{Muons}
	Muons have a very low probability to interact with the Calorimeters of CMS. Because of that, muons are identified by hits in the muon chambers. Combining informations from the muon system and the inner tracker leads to a very precise measurement of muons in the CMS detector. Therefore muons are very reliable objects in CMS data analyses. An object is announced a muon if %todo KRITERIEN MUON
\section{Jets}
	Jets are objects, used to reconstruct quarks. Because of confinement quarks cannot exist isolated but hadronize. This results in a particle shower consisting of hadrons. To reconstruct the initial quark one needs to sum up all particles from the final shower. To combine the tracks in a well defined way jet algorithms are used. Two important requirements for an jet algorithm are to be infrared and collinear safe. 
	%todo Infrared and Collinear safe erklären
	All presented jet algorithms in this thesis fulfil these requirements.
\subsection{Anti-$k_T$ Jet Algorithm}
	In CMS the common way to cluster jets from the detected particles is to use iterative jet algorithms. Thus particles are clustered step by step until an abort criterion is reached. The most common and proven to be useful iterative jet algorithm is the Anti-$k_T$ (AK) algorithm \cite{antikt}. As an input a list of objects, reconstructed in the detector is given. The AK algorithm then calculates two quantities $d_{ij}$ (Eq. \ref{eq:dij}) and $d_{iB}$ (Eq. \ref{eq:iB}) for each pair of objects $i$ and $j$:
	\begin{equation}
	d_{ij} = min (k_{T,i}^{-2}, k_{T,j}^{-2})  \frac{\Delta R_{ij}^2}{R^2}
	\label{eq:dij}
	\end{equation}
	\begin{equation}
	d_{iB} = k_{T,i}^{-2}.
	\label{eq:iB}
	\end{equation}
	Here $d_{ij}$ describes a effective distance between two objects $i$ and $j$, $d_{iB}$ is a distance measure from object to beam axis. $k_T$ is the transverse momentum of an object, $\Delta R_{ij}$ denotes the distance between objects $i$ and $j$ an $R$ is a constant parameter that defines the radius of the resulting jet. When $d_{ij}$ is smaller than $d_{iB}$ both objects $i$ and $j$ are combined and both quantities are calculated again. At some point, $d_{iB}$ will be smaller than any $d_{ij}$, then object $i$ is called a jet and is removed from the list of objects. This procedure is repeated until the list of objects is empty. With the anti-$k_T$ jet algorithm, particles with large transverse momenta are clustered first because $d_{ij}$ (Eq. \ref{eq:dij}) is easily smaller then $d{iB}$ for large $p_T$. The resulting jets are very circular. 
	%todo plot mit AK shape.
	%todo KT und CA auch erwähnen?

\subsection{HOTVR Jet Algorithm}
	Another approach to cluster jets is the 'heavy object tagger with variable R' (HOTVR) \cite{hotvr}. This algorithm does not use a constant radius parameter $R$ but a $p_T$ dependent effective $R_\text{eff}$ (see Eq. \ref{eq:HOTVR}). Thus the $R_\text{eff}$ decreases with increasing $p_T$ leading to smaller jets when the decay products are expected to be more close because of the Lorentz boost. The $p_T$ dependence is scaled with a parameter $\rho$ with a default value of $600\;\text{GeV}$. Furthermore, upper and lower boundaries for the jet radius can be set. The default values are $R_\text{min} = 0.1$ and $R_\text{max} = 1.5$. 	
	\begin{equation}
	\label{eq:HOTVR}
	  R_\text{eff} =
	   \begin{cases}
	     R_\text{min} & \text{for } \rho / p_T < R_\text{min} \\
	     R_\text{max} & \text{for } \rho / p_T > R_\text{max} \\
	     \rho / p_T & \text{else}  
	   \end{cases}
	\end{equation}
	
	\noindent This effective $R$ is then used with the equations of the Anti-$k_T$ algorithm described earlier (see Eq. \ref{eq:dij} - \ref{eq:iB}). Additionally, a mass-jump criterion is used.
	%todo mass jump Kriterium, auch mit Formel
	

\subsection{XCone Jet Algorithm}
\label{sec:xcone}
	XCone \cite{xcone} is an exclusive jet algorithm, returning conical jets. It is well suited for analysis where the final state and therefore the expected number of jets is known since it returns a fixed number of jets. Thus, a physical final state has direct influence on the jet finding. \\
	Starting with a fixed number of jet axes $N$ the algorithm calculates the direction of these axes by minimizing the N-jettiness variable. N-jettiness is a measure for how N-jet-like an event looks. The definition is shown in Eq. \ref{njettines}.
	\begin{equation}
	\tilde{\tau}_N = \sum_i \min\{\rho_\text{jet}(p_i, n_1), \dots, \rho_\text{jet}(p_i, n_N), \rho_\text{beam}(p_i)\}
	\label{njettines}
	\end{equation}
	%todo Eigenschaft mit aneinandergrenzenden Jets erklären
	Once the minimizing process converges, all particles inside a radius $R$ from a jet axis are added to one jet. \\	
	\\ Since the N-jettiness variable is often used to do theory calculations of particle physics events, the XCone algorithm is easier to include in these calculations.

\subsection{b-Jets}
	In this analysis jets originating from a bottom quark are identified to reduce background. To identify a jet as an b-jet the "Combined Secondary Vertex" (CSV) algorithm is used. Since b-hadrons have a large lifetime of $1.5\;\text{ps}$ they travel about $450\;\text{\textmu m}$ in the detector before decaying. This leads to a secondary vertex at the spatial point where the hadron decays which can be reconstructed in the tracking system. Additionally the composition of hadrons in a b-jet is different from other jets.
	%todo ALLES AUßER secondary vertex NOCHMAL NACHLESEN 	
	%todo working points beschreiben
	These properties are taken advantage of in the CSV algorithm. 
	
\section{$\cancel{E}_T$ and $S_T$}
	With the information of mentioned objects two important variables are defined. The missing transverse energy $\cancel{E}_T$ is defined to estimate the energy carried away by particles which leave the experiment undetected. Summing up all transverse momenta of each particle in the final state returns the $p_T$ of the system in the initial state which is $p_T = 0$ at LHC. Thus the transverse energy of all undetected particles is defined as the absolute value of the negative sum over all transverse momenta of detected objects (Eq. \ref{eq:MET}). Since every object, independent from it's $\eta$, has to be taken into account for this variable, $\cancel{E}_T$ is independent from the selection applied. The missing energy is due to neutrinos which can not be detected with CMS because of their low probability to interact with the detector material or a new physics state. 
	
	\begin{equation}
		\cancel{E}_T = \left| - \sum_{\textrm{leptons, jets}}^{} \vec{p}_T \right|
		\label{eq:MET}
	\end{equation}
	
	\noindent Another important variable to describe an event is $S_T$. As represented in Eq. \ref{eq:ST} is defined as the scalar sum of all transverse momenta of reconstructed objects plus the missing transverse energy mentioned above. Different from $\cancel{E}_T$ only objects surviving the selection are considered. $S_T$ gives an estimate how much energy one event contains and is therefore used to distinguish between low energy QCD and high energy processes. 
	
	\begin{equation}
		S_T = \left( \sum_{\textrm{leptons, jets}}^{} \vert \vec{p}_T \vert \right) + \cancel{E}_T
		\label{eq:ST}
	\end{equation}
	
	\noindent Additionally, a definition for $S_T$ which just takes leptonic activity into account is used in this analysis. It is similarly defined and referred to as $S_T^{\text{lep}}$ (see Eq. \ref{eq:STlep}).
	
	\begin{equation}
		S_T^\text{lep} = \left( \sum_{\textrm{leptons}}^{} \vert \vec{p}_T \vert \right) + \cancel{E}_T
		\label{eq:STlep}
	\end{equation}	
