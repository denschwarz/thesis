\section{Analysis}
\subsection{Selection}
	First a pre-selection is applied to obtain a clean $t\bar{t}$ data set with a low fraction of background processes. An event must contain a muon or electron.
\subsection{Jet studies}
	For this analysis it is crucial to choose a suitable jet algorithm and cone size. A similar analysis has been performed by CMS on LHC's 2012 data set with a center-of-mass energy of $8\;\text{TeV}$ with Cambridge-Aachen jets with a radius parameter of $R=1.2$. This large radius was chosen to compensate for low statistics in the boosted $t\bar{t}$ regime. Since the cross-section of $t\bar{t}$ production is much higher at a center-of-mass energy of $13\;\text{TeV}$ it has to be studied, how the jet radius parameter $R$ influences the measured distribution and what influence different jet algorithms have.
\subsection{Comparison of different jet algorithms}
\subsection{Unfolding}