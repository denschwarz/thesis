\chapter{Analysis}
\label{ch:Ana}
	This chapter will cover the analysis performed for this thesis. A basic idea of the goals and strategy of this analysis is given in section \ref{sec:strategy}. Previous results, this analysis refers to, are presented in the following. Then a detailed look into event selections, studies on particle level, differences between jet algorithms and finally the unfolding process and its results follows in sections \ref{sec:jet_studies} to \ref{sec:results}.
\section{Analysis Strategy}
\label{sec:strategy}
	This analysis aims for boosted $t\bar{t}$ where all decay products from the top decays merge into a single jet. To measure in this phase space a selection of events is applied. The detailed selection is presented in section \ref{sec:selection}. In the required phase space the distribution of the jet mass of a top quark decaying into quarks ($t\rightarrow W^{+} b \rightarrow b q \bar{q}'$) is measured. Following a unfolding is performed using the TUnofld \cite{tunfold} software package. The goal is to compare data unfolded to particle level with first principle calculations.
\section{Previous Results}
	A measurement of the top quark mass in highly boosted $t\bar{t}$ events has already been performed by CMS with the $8\;\text{TeV}$ dataset corresponding to an integrated luminosity of $19.7\;\text{fb}^{-1}$. Details can be found in \cite{torben_paper}, while in this section a general description is given focussing on possible improvements with the larger dataset at $13\;\text{TeV}$. In the mentioned publication, Cambridge/Aachen jets with a radius parameter of $R=1.2$ were used to reconstruct top quark decays.
	%todo Plots torben
	%todo Limitierung (statistik, pile-up)
	%todo warum ist 13Tev Analyse sinnvoll?
\section{Jet Studies on Particle Level}
\label{sec:jet_studies}
	%todo goal: distribution with nice peak!
	%todo show deltaR(lepton, fatjet) to show, that categorisation into had and lep works
	For this analysis it is crucial to choose a suitable jet algorithm and cone size. The previous mentioned analysis \cite{torben_paper} uses Cambridge-Aachen jets with the radius parameter set to $R=1.2$ for its measurement. This large radius was chosen to compensate for low statistics in the boosted $t\bar{t}$ regime. Since the cross-section of $t\bar{t}$ production is much higher at a center-of-mass energy of $13\;\text{TeV}$ it has to be studied, how the jet radius parameter $R$ influences the measured distribution and what influence different jet algorithms have. The goal is to select a jet algorithm which returns jets in which all decay products of a hadronically decaying top quark are merged. In this case, the jet mass $M_\text{jet}$ is sensitive to the top quark mass $M_\text{top}$.
	\\
	The jet studies are performed with a $t\bar{t}$ simulation using the information of MC simulations at particle level. The detailed selection is described in Section \ref{sec:GenSel}.

\subsection{Selection on particle level}
\label{sec:GenSel}
	Only events are considered where one top quark decays leptonically into $t\rightarrow W^{+} b\rightarrow \mu^{+} \nu_{\mu} b$ or $t\rightarrow W^{+} b\rightarrow e^{+} \nu_e b$ while the hadronically decaying top quark carries a transverse momentum greater $300\;\text{GeV}$. On this $t\bar{t}$ sample, a selection is applied to select boosted top quarks. Cuts are applied on the jets clustered by the according jet algorithm:
	\begin{itemize}
	\item $p_T^{\text{1st jet}} > 400\;\text{GeV}$ 
	\item $p_T^{\text{2st jet}} > 200\;\text{GeV}$ 
	\item Veto on additional jets with $p_T > 200\;\text{GeV}$ 
	\item $M^{\text{1st jet}} > M^{\text{2nd jet}}$
	\end{itemize}
	Where the first jet refers to the leading jet in $p_T$. The first jet is expected to be originating from the hadronically decaying top quark, the second one is expected contain the products of the leptonically decaying top quark. The purpose of the $p_T$ thresholds is to select boosted top decays. The Veto on additional jets is set to select $t\bar{t}$ events where one jet per top quark decay is expected. It is to mention, that the veto is not present in combination with XCone since it will always return exactly two jets in the used set up. To suppress events where not all decay products of the hadronically decaying top quark end up in the jet with the highest transverse momentum, a mass criterion $M^{\text{1st jet}} > M^{\text{2nd jet}}$ is set. This selection is applied to every jet algorithm output to be able to compare these different approaches.
	
	%todo Selection in Tabellen zusammenfassen, siehe Teresa
	
\subsection{Studies with Anti-$k_T$ and HOTVR}
\label{sec:AKHOTVR}	
	Since the top quark decay should be reconstructed with one jet, all decay products need to lay inside the defined jet cone. Choosing different cone sizes has various effects. When the cone is small, not all decay products may end up in the jet and the jet mass is reconstructed smaller than the top mass. If the cone size is large, the probability of additional radiation and pile-up grows and the resulting jet mass is reconstructed too high. As a starting point Anti-$k_T$ jets with a radius of $0.8$ are selected since this is the CMS intern standard to reconstruct top quark jets. To study the influence of the cone size, AK jets with a radius of $0.8$ and $1.2$ are presented in Fig. \ref{fig:GEN_AK08} and \ref{fig:GEN_AK12}. Additionally a matching is performed. If all three decay products of the top quark are clustered into the jet, the jet is called 'matched'. One can see that AK8 jets tend to deliver a jet mass lower than the top quark mass of about $173\;\text{GeV}$. This is due to the higher fraction of 'not matched' events. In this case, these are events where not every decay product ends up in the jet. AK12 jets on the other hand often return a mass higher than the top quark mass which is due to underlying event effects. A larger cone size has a higher probability of including particles not originating from the top quark one is interested in.
	\begin{figure}[tb]
		\begin{subfigure}{.5\textwidth}
	    \centering
		\includegraphics [width=\textwidth]{../Plots/GenStudies/AK08_matching}
		\caption{}
		\label{fig:GEN_AK08}
		\end{subfigure}
		\begin{subfigure}{.5\textwidth}
		\centering
		\includegraphics [width=\textwidth]{../Plots/GenStudies/AK12_matching}
		\caption{}
		\label{fig:GEN_AK12}
		\end{subfigure}
		\caption{Comparison of jet mass distributions of AK8 (a) and AK12 (b) jets. A smaller cone size (a) leads to a lower reconstructed mass while a large cone (b) returns higher masses. The fraction of 'matched' and 'not matched' events is shown in the histograms.}
	\end{figure}
	
\subsection{Studies with XCone}
\label{sec:XCone_strat}
	The XCone jet algorithm described in section \ref{sec:xcone} has already been tested resolving $t\bar{t}$ decays. Studies for hadronically decaying top quark pairs are presented in a paper from Thaler and Wilkason \cite{xconetop}. Here, the XCone algorithm is tuned to the $t\bar{t}$ final state, expecting six jets. Using the information that it is expected to find three jets from each top quark, a promising approach to reconstruct the top quark decays was made with a strategy using two clustering steps. Firstly, XCone is required to find exactly two jets with a large radius ensuring that all decay products of the top quark end up in the jet. Thus, every particle from the hard scattering should be clustered into one of the jets. The goal of this first step is to separate the two top quarks into independent jets. Now, a second clustering step finding three jets is run where separate lists of particles from each fat jet are used as an input. Thus, in each fat jet, three smaller jets with $R=0.4$ are found, which will be referred to as subjets. These three small jets are then combined and used as the final top jets. \\
	\\ In this analysis the lepton+jets channel is used. Therefore the jet originating from the hadronically decaying top quark is identified while calculating the distance between each jet and the lepton. Additionally studies with a method where the fat jet originating from the leptonically decaying top quark is divided into two instead of three jets is presented ('$2+5$'). This approach was made because the neutrino from the $W$ decay cannot be seen in the detector. Nevertheless it proved handy to use the '$2+6$' method for data and therefore also for MC samples after detector simulation.  
	%todo JetDisplays!
	%todo Softdrop vergleich zeigen?
 
	\begin{figure}[tb]
		\begin{subfigure}{.5\textwidth}
	    \centering
		\includegraphics [width=\textwidth]{../Plots/JetDisplayR15/xcone_incjets_event04}
		\caption{}
		\label{fig:JetDisplay1}
		\end{subfigure}
		\begin{subfigure}{.5\textwidth}
	    \centering
		\includegraphics [width=\textwidth]{../Plots/JetDisplayR15/xcone_subjets_event04}
		\caption{}
		\label{fig:JetDisplay2}
		\end{subfigure}
		\caption{Jet displays }
	\end{figure}	
 	\begin{figure}[tb]
  		\centering
 		\includegraphics [width=.5\textwidth]{../Plots/GenStudies/XCone23_matching}
 		\label{fig:GEN_XCone}
 		\caption{Jet mass distribution of XCone jets.}
 	\end{figure}
 	
\subsection{Comparing Jet Algorithms}
%todo vergleich auch mit XCone Softdrop!!!

\section{Selection on Reconstruction Level}
\label{sec:selection}
	To obtain a data set consisting of mostly $t\bar{t}$ events in the lepton+jets channel, a selection is applied to simulation and data. The selection can be divided into two steps. Firstly, a baseline selection is used to suppress background processes (see section \ref{sec:PreSel}). Secondly, the final phase space is defined (see section \ref{sec:FinalSel}) to select $t\bar{t}$ events with boosted top quarks. This is crucial for this analysis because the goal is to reconstruct the top quark with one jet. This can only be done if all of its decay products merge into one jet.
%todo PRESEL UND SUPPRESS ZUSAMMEN?
\subsection{Baseline Selection}
\label{sec:PreSel}
	In the lepton+jets channel of the $t\bar{t}$ process one expects to find exactly one muon or electron, two small jets from the hadronically decaying $W$ boson, two b-jets and missing transverse energy since the neutrino cannot be detected. This baseline selection is designed to remove non-$t\bar{t}$ events. After applying this selection the remaining sample consists of about $80\%$ $t\bar{t}$ events. The main remaining backgrounds are $W+$jets and Single-Top production. 
	\begin{itemize}
	\item Single Muon trigger 
	\item missing transverse energy of more than $50\;\text{GeV}$,
	\item 2 or more AK4 jets with a $p_T$ greater than $50\;\text{GeV}$.
	\item $1$ muon or electron candidate with a veto on additional leptons
	\item $S_T^\text{lep} > XX$
	\item B-TAG
	\item 2D Cut
	\item $\cdots$	
	\end{itemize}
	
\subsection{Jet Energy Corrections for XCone Jets} 
	The normal procedure in CMS analyses is to apply jet energy corrections (see section \ref{sec:jec}) to every jet collection used. Those jet energy corrections have been derived by dedicated CMS groups and are different depending on the jet algorithm used to cluster jets. Since the XCone algorithm is not a standard jet finding procedure in CMS, there are no valid corrections available. The first attempt to correct XCone jets is to use the AK4 jet corrections since the jet shape should be very similar to XCone jets with $R=0.4$ as they were used in this analysis.
	%todo show plots and explain why it does not work
	
	To still be able to correct XCone jets for response non linearities and pile-up effects, a correction factor on top of AK4 corrections is derived for XCone jets. For this, only events from $t\bar{t}$ simulation are used. Furthermore, only the subjets from the jet belonging to the hadronically decaying top quark are considered. Now a matching to generator jets is executed and the fraction $R=\frac{p_T^{\text{rec}}}{p_T^{\text{gen}}}$ calculated. This is done in different $p_T$ and $\eta$ regions. The mean $R$ is then filled in a two dimensional histogram representing the $p_T$-$\eta$-plane.
	%todo show 2D Plots
	
	The correction factor applied to every XCone jet is now $f = \frac{1}{R}$. To get a smooth transition between the different regions, in every $\eta$ bin a function is fitted to get a factor $f(p_T)$.
	%todo show fit Plots
	%todo show resulting plots
\subsection{Measurement Phase Space}
\label{sec:FinalSel}
	%VERWEIS AUF GENSEL
	This analysis focuses on boosted top quarks. Therefore the leading jet, which is supposed to contain all decay products of the hadronically decaying top quark, is required to carry a transverse momentum greater than $400\;\text{GeV}$. Additionally only events are selected where the leading jet has a greater mass than the second jet. This last selection step prefers events with merged jets because here the jet from the leptonic top quark will only contain the lepton and a jet from the bottom quark since neutrinos cannot be detected. Therefore the mass of the hadronic jet is expected to be larger.
	\begin{itemize}
	\item $p_T^{\text{1st jet}} > 400\;\text{GeV}$ 
	\item $M^{\text{1st jet}} > M^{\text{2nd jet}}$
	\end{itemize}

\section{Unfolding}
\section{Results}
\section{Outlook}
%todo elec channel
%todo sys uncertainties
%todo unfolding optimisation, pt bins

\label{sec:results}