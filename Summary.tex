% !TEX root = Master_Thesis.tex
\chapter{Summary and Outlook}
	An analysis performing a measurement of the jet mass distribution in boosted top quark decays has been presented in this thesis. Studies of various jet clustering algorithms were performed to find the most suitable jet definition to reconstruct boosted top quarks. On particle level, the influence of the jet radius on Anti-$k_T$ jets as well as jet mass distributions obtained by using HOTVR jets were examined and finally compared to a novel clustering method, namely XCone. The exclusive approach of XCone is used to define a specialised jet finding method for this analysis. This method proofed to be very successful in reconstructing hadronically decaying top quarks, returning a jet mass distribution with a narrow peak at the top quark mass. Together with high statistics and a large fraction of jets containing all decay products, the presented jet clustering method with XCone was chosen to provide the final jet which is then measured. The jet definition as well as the selection of the measurement phase space are then analogously applied on reconstruction level. \\
	On reconstruction level, baseline criteria are defined to select a preferably pure $t\bar{t}$ sample. Additionally, a further selection is applied defining a measurement phase space with boosted top quark decays similar to particle level. The jet mass distribution after passing every selection criterium shows is exceptional features on reconstruction level as well. Furthermore, also pile-up is not effecting the distribution much, indicating a well suited input for an unfolding procedure. Derived from simulation, jet energy corrections for XCone jets were derived. Based on AK4 corrections, that are expected to behave similarly to XCone subjets with the same radius, a correction factor was obtained to fit XCone jets. The $p_T$ and $\eta$ dependent correction was found via a matching from reconstructed jets to jets on particle level in simulation. After a validation with the well known $W$ boson mass, the new jet energy corrections are applied to every measured jet.\\	
	A first unfolding approach was tested with Monte Carlo samples. Unfolding of a simulation with itself confirmed a correctly filled migration matrix. Checks with independent samples indicated a modelling dependent bias, though. Nevertheless, an unfolding result was presented, treating the observed bias as model uncertainties. \\
	The large uncertainties do not allow a precision measurement yet, but are expected to decrease a lot with a more complex migration matrix including migrations from lower jet $p_T$ regions. Furthermore, a model dependency can be validated and better understood with simulations from different generators. Besides that, the binning of the matrix is yet not examined in full detail. Here, purity and stability are useful and recommended measures to define a well suited binning scheme. Moreover, the electron + jets channel can be added to this analysis to increase statistics and systematic uncertainties have to be covered to present a final result. \\
	After obtaining a well understood unfolded distribution with a sophisticated migration matrix, a top quark mass can be extracted. First tests with simulations of different mass points but especially comparisons with theory calculations are planned in the future of this analysis. \\
	At this point, the analysis already shows its capability to improve the result obtained with the 2012 dataset at $8\;\text{TeV}$ because of a large improvement in statistics and top quark reconstruction with XCone jets.
