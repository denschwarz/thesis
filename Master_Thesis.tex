\documentclass[12pt, a4paper, twoside, openright]{report}

% ===========================================================
% ======================== packages =========================
% ===========================================================
\usepackage[ngerman, english]{babel}
\usepackage[ansinew]{inputenc}
\usepackage[T1]{fontenc}
\usepackage{mathptmx}
\usepackage{amsmath}
\usepackage{amssymb}
\usepackage{amstext}
\usepackage{amsfonts}
\usepackage{mathrsfs}
\usepackage{enumitem}
\usepackage{graphicx}
\usepackage[left=2.5cm,right=2.5cm,top=2.5cm,bottom=2cm]{geometry}
\usepackage[onehalfspacing]{setspace}
\usepackage[section]{placeins} %Bilder rutschen nicht aus section
\usepackage[markup=nocolor,deletedmarkup=xout]{changes}
\usepackage{cancel}			%MET symbol
\usepackage{chngcntr}
\usepackage{lscape}
\usepackage[normal]{caption} %style f�r Bildunterschriften
\usepackage{multicol} 
\usepackage{epstopdf}		%.eps get converted to pdf
\usepackage{textcomp}
\usepackage{url}			%for hyperlinks
\usepackage{blindtext} 		%use blindtext
\usepackage{subcaption} 	%for (a) in subfigures
\usepackage{tikz} 			%new graphic options
\usepackage{verbatim}		
\usepackage{microtype} 		%bessere Zeilenumbrueche, z.B. ISBN in Literatur
\usepackage{typearea}
\usepackage{tikz-feynman} 	%Feynman Diagrams
\tikzfeynmanset{compat=1.1.0}
\usepackage{fancyhdr}		%new head and foot lines
\usepackage[babel, german=quotes]{csquotes}
\usepackage{hyperref}
\usepackage[style=ieee, maxbibnames=100, sorting=none, backend=bibtex]{biblatex}
% ===========================================================
% ======================== new commands =====================
% ===========================================================
\newcommand*\chem[1]{\ensuremath{\mathrm{#1}}} %chem formulas

% ===========================================================
% ======================== configuration ====================
% ===========================================================
% customised header and footer
\pagestyle{fancy}
\fancyhf{}
\fancyfoot[CE,CO]{\thepage}
\fancyhead[CO]{\rightmark}
\fancyhead[CE]{\leftmark}
% redefine cleardoublepage: do not print fancy headline if page is empty before new chapter
\let\mtcleardoublepage\cleardoublepage
\renewcommand{\cleardoublepage}{\clearpage{\pagestyle{empty}\mtcleardoublepage}}
% =======================
% title style
\usepackage{color}
\definecolor{gray75}{gray}{0.75}
\usepackage{titlesec}
\titleformat{\chapter}[block]
  {\normalfont\Huge\bfseries}
  {\thechapter \hspace{20pt} \textcolor{gray75}{|} \hspace{20pt}}
  {0pt}
  {\Huge}
\titlespacing*{\chapter}{0pt}{0cm}{1cm}
\titleformat{\section}[block]
  {\normalfont\Large\bfseries}
  {\thesection}
  {1em}
  {}
\titleformat{\subsection}[block]
  {\normalfont\large\bfseries}
  {\thesubsection}
  {1em}
  {}
\titleformat{\subsubsection}[block]
  {\normalfont\normalsize\bfseries}
  {}
  {1em}
  {}
% =======================           
% let figure, tables, etc stay in chapter
\counterwithin{figure}{chapter}
\counterwithin{table}{chapter}
\counterwithin{equation}{chapter}
% ======================= 
% bibliography stuff
\addbibresource{MasterBib.bib}
\ExecuteBibliographyOptions{% 
firstinits=true, 
maxbibnames=3, % Alle Autoren (kein et al.) 
minalphanames=3, 
maxalphanames=3, 
}% 


\urlstyle{same}
% =======================
\begin{document}
% ===========================================================
% ======================== Deckblatt ========================
% ===========================================================
\selectlanguage{ngerman}
\pagenumbering{roman}
\thispagestyle{empty}
\begin{titlepage}

\newcommand{\HRule}{\rule{\linewidth}{0.5mm}} % Defines a new command for the horizontal lines, change thickness here

\center % Center everything on the page
 
%----------------------------------------------------------------------------------------
%	HEADING SECTIONS
%----------------------------------------------------------------------------------------

\textsc{\LARGE Universit"at Hamburg}\\[1.5cm] % Name of your university/college
\textsc{\Large Masterarbeit}\\[0.5cm] % Major heading such as course name
\textsc{\large im Studiengang Physik}\\[0.5cm] % Minor heading such as course title

%----------------------------------------------------------------------------------------
%	TITLE SECTION
%----------------------------------------------------------------------------------------

\HRule \\[0.4cm]
{\bfseries \huge Measurement of the Jet Mass Distribution in Boosted Top Quark Decays \\[0.4cm] } 
%{\bfseries \LARGE Measurement of the Jet Mass Distribution in Boosted Top Quark Decays \\[0.1cm] \rule[0mm]{50mm}{0.4mm} \\[0.2cm] Messung der Jetmassenverteilung in kollimierten Top-Quark Zerf"allen \\[0.4cm] } 
\HRule \\[4cm]
 
%----------------------------------------------------------------------------------------
%	AUTHOR SECTION
%----------------------------------------------------------------------------------------
\begin{minipage}[t l]{0.5\textwidth}
\begin{flushleft} \large
\emph{Autor:}\\
\textsc{Dennis Schwarz} % Your name
\\[1cm]
\emph{Gutachter:} \\
\textsc{Prof. Dr. Johannes Haller} \\ 
\textsc{Prof. Dr. Peter Schleper}
\end{flushleft}
\end{minipage}
~
\begin{minipage}[t r]{0.4\textwidth}
\begin{flushright}
\includegraphics[width=.7\textwidth]{../Logos/UHHlogo.png}
\end{flushright}
\end{minipage}
\\[4cm]
%----------------------------------------------------------------------------------------
%	DATE SECTION
%----------------------------------------------------------------------------------------
{\large 2017} % Date, change the \today to a set date if you want to be precise

\vfill % Fill the rest of the page with whitespace

\end{titlepage}
% clearpage without page number:
{\clearpage{\pagestyle{empty}\mtcleardoublepage}} 
%Abstract
\section*{Zusammenfassung}
\thispagestyle{plain}
\blindtext
\selectlanguage{english}
\section*{Abstract}
\blindtext

\clearpage
\tableofcontents
\addtocontents{toc}{~\hfill\textbf{Page}\par}
\clearpage
{\pagestyle{empty} \cleardoublepage}



\pagenumbering{arabic}
\selectlanguage{english}
% ===========================================================
% ======================== Inhalt ===========================
% ===========================================================
%todo Ueberschriftren einheitlich gross/klein!
%todo Absaetze, indent pruefen
\section{Introduction}
\chapter{Theory}
\section{Standard model of particle physics}
	The standard model of particle physics is a quantum field theory describing elementary particles, their interaction and provides the affiliated equations of motion. All particles contained in this theory have been discovered and various experiments have confirmed predictions of the standard model at very high precision. All known elementary particles of the standard model and their properties - mass, electromagnetic and colour charge and their spin - are displayed in Fig. \ref{SM}. They are ordered in two fundamental groups by their spin. spin-$\frac{1}{2}$ particles are called fermions, particles with integer spin represent the bosons. 
	\\
	Fermions are further divided into quarks and leptons. While quarks are affected by the strong force, leptons are not. Both subgroups consist of three generations with two particles in each group. Each generation of quarks contain one up-type and one down-type quark distinguished by their electromagnetic charge. For leptons a generation is built from a particle with charge of $1e$ and an neutral neutrino. The first generation of quarks and leptons then are the building blocks of atoms. All other particles have a larger mass and therefore a finite lifetime. Additionally every fermion has a anti-partner which has the same mass but charge are multiplied by $-1$. 
	\\	
	Bosons are the carriers of the three fundamental forces included in the standard model. Every interaction of two particles is described by a boson emitted from one and absorbed by the other particle. The boson of the strong force is a massless gluon. It couples to particles carrying a so-called colour charge. Thus, quarks and the gluon itself are the only candidates affected by the strong force. 
	% CONFINEMENT
	\\	
	Photons carry the electromagnetic force. All particles besides the photon, gluon, Higgs boson and the neutrinos can interact with this force due to their electrical charge. 
	\\
	The third force described in the standard model is the weak force. It is carried by the $W$ and $Z$ bosons. This force only affects left-handed fermions and right-handed anti-fermions. 
	\\
	A hypothetical graviton is also included in Fig. \ref{SM} representing the last known fundamental force, gravity, which is not yet included in the standard model. 
	\\
	The remaining boson, the Higgs boson, was discovered the most recent in 2012. In the standard model without a Higgs mechanism, all bosons should be massless. In contradiction to this, the mass of the $W$ and $Z$ bosons are measured to values of ??(genaue Werte inkl Quelle)??. Therefore a theory was developed which solves this problem via spontaneous symmetry breaking. A field is predicted giving particles mass proportional to ther coupling strength to this field. This is also leading to an additional particle, the Higgs boson.

	% QED
	% QCD
	% WEAK
	% HIGGS
	% PROBLEME???
	\begin{figure}[tb]
		\centering
		\includegraphics [width=\textwidth]{../Plots/Standard_Model.png}
		\caption{Display of the particle content of the standard model divided into groups of fermions and their three generations and bosons. Additionally a hypothetical boson ('graviton') for the not included gravity is shown.\cite{SM}}
		\label{SM}
	\end{figure}
		 
\subsection{Top Quark}
	The top quark is an up type quark belonging to the third generation and carrying electromagnetic charge of $Q=+\frac{2}{3}e$ \cite{pdg2016}. With its mass of about $173\;\text{GeV}$ the top quark is the heaviest particle in the standard model. Therefore it offers a large phase space for decays and  has a short life time of approximately $0.5 \times 10^{-24}\;\text{s}$ \cite{pdg2016}. Because of the short life time the top quark does not form bound hadronic states and thus measurement of the bare quark are possible. This provides a special access to parameters of the standard model. Especially the top quark mass is an essential parameter to check the standard model for consistency. The Higgs mechanism relates the masses of the top quark with the masses of the W and Higgs boson. This could be done, because the top quark plays an extraordinary role due to its high mass -- and therefore large coupling to the Higgs boson -- in this mechanism. Measuring these masses with high precision gives the possibility to check the standard model for consistency. 
	\\
	Additionally the top quark is important for searches of new physics since is is often part of the final state and/or a dominant background. 
	\\
	In hadron colliders, the production of a $t\bar{t}$ pair happens via $g\bar{q}$ annihilation or gluon fusion. At the centre-of-mass energy of $13\;\text{TeV}$ from the LHC, gluon fusion is by far the dominant process. Top quarks can also be produced in single production, but has, being a electroweak process, a much smaller cross section. Hence, this analysis will focus on pair produced top quarks and will treat single top production as a background process.

	\begin{figure}
		\centering
		\begin{subfigure}{.4\textwidth}
		\begin{tikzpicture}
		\begin{feynman}	
		\vertex(a);
		\vertex[right=of a] (b);
		\vertex[below left=of a] (i1){\(g\)};
		\vertex[above left=of a] (i2){\(g\)};		
		\vertex[below right=of b] (f1){\(\overline t\)};				
		\vertex[above right=of b] (f2){\(t\)};		
		\diagram* {
		(a)-- [gluon,edge label'=\(g\)] (b),
		(i1)-- [gluon] (a) -- [gluon] (i2),
		(f1)-- [fermion] (b) -- [fermion] (f2),
		};
		\end{feynman}
		\end{tikzpicture}
		\caption{}
		\end{subfigure}
		\begin{subfigure}{.4\textwidth}
		\begin{tikzpicture}
		\begin{feynman}	
		\vertex(a);
		\vertex[right=of a] (b);
		\vertex[below left=of a] (i1){\(q\)};
		\vertex[above left=of a] (i2){\(\overline q\)};		
		\vertex[below right=of b] (f1){\(\overline t\)};				
		\vertex[above right=of b] (f2){\(t\)};		
		\diagram* {
		(a)-- [gluon,edge label'=\(g\)] (b),
		(i1)-- [fermion] (a) -- [fermion] (i2),
		(f1)-- [fermion] (b) -- [fermion] (f2),
		};
		\end{feynman}
		\end{tikzpicture}
		\caption{}
		\end{subfigure}
		\begin{subfigure}{.4\textwidth}
		\begin{tikzpicture}
		\begin{feynman}	
		\vertex(a);
		\vertex[below=of a] (b);
		\vertex[above left=of a] (i1){\(g\)};
		\vertex[below left=of b] (i2){\(g\)};		
		\vertex[above right=of a] (f1){\(t\)};				
		\vertex[below right=of b] (f2){\(\overline{t}\)};		
		\diagram* {
		(b)-- [fermion] (a),
		(i1)-- [gluon] (a) -- [fermion] (f1),
		(i2)-- [gluon] (b) -- [anti fermion] (f2),
		};
		\end{feynman}
		\end{tikzpicture}
		\caption{}
		\end{subfigure}
		\begin{subfigure}{.4\textwidth}
		\begin{tikzpicture}
		\begin{feynman}	
		\vertex(a);
		\vertex[below=of a] (b);
		\vertex[above left=of a] (i1){\(g\)};
		\vertex[below left=of b] (i2){\(g\)};		
		\vertex[above right=of a] (f1){\(t\)};				
		\vertex[below right=of b] (f2){\(\overline{t}\)};		
		\diagram* {
		(b)-- [gluon] (a),
		(i1)-- [gluon] (a) -- [fermion] (f1),
		(i2)-- [gluon] (b) -- [anti fermion] (f2),
		};
		\end{feynman}
		\end{tikzpicture}
		\caption{}
		\end{subfigure}
		\caption{Feynman diagrams \cite{feynman} showing the production of top quark pairs. Displayed are the gluon-gluon fusion (a), the quark anti-quark annihilation (b) and the t-channel (c+d).}
		\label{fig:production}
	\end{figure}	
	
	The top quark decays via the weak interaction with a probability of almost $100\%$ into a bottom quark and a $W$ boson. While the bottom quark is seen as a jet in the detector, the $W$ boson decays further into a quark anti-quark pair (see fig. \ref{fig:decaya}) or into a lepton and a neutrino (see fig. \ref{fig:decayb}). Looking at the $t\bar{t}$ production, these two possible final states for each top quark corresponds to three channels for the $t\bar{t}$ process. 
	\begin{itemize}
	\item both top quarks decay into quarks (full hadronic)
	\item one top quark decays hadronically, the other one leptonically (lepton+jets)
	\item both top quarks decay leptonically (dilepton)
	\end{itemize}
	The full hadronic and lepton+jets channels are dominant and occur $45.7\%$ respectively $43.8\%$ of the time. $10.5\%$ of all $t\bar{t}$ events result in two leptons in the final state \cite{pdg2016}. This analysis will focus on the lepton+jets channel which is pictured in figure \ref{fig:semilep}. It is suitable because the lepton makes it easier to distinguish $t\bar{t}$ events from background events but also includes a fully hadronically decaying top quark which will be the target for the presented measurement.
	
	% LIVE TIME -> no hadronisation
	% YUKAWA ? 
	% DECAY AND PRODUCTION
	% MC Templates, MC Mass
	% ENERGY SCALE
	% RUNNING MASS	
	\begin{figure}
	\begin{subfigure}{.5\textwidth}
		\begin{tikzpicture}
		\begin{feynman}
		\vertex(a) {\(t\)};
		\vertex[right=of a] (b);
		\vertex[above right=1.5cm of b] (c);
		\vertex[below right=3cm of b] (f1){\(b\)};
		\vertex[below right=1.5cm of c] (f3){\(\overline l\)};
		\vertex[above right=1.5cm of c] (f2){\(\nu\)};

		\diagram* {
		(a)-- [fermion] (b)-- [fermion] (f1),
		(b)-- [boson,edge label'=\(W\)] (c),
		(c)-- [fermion] (f2),
		(c)-- [anti fermion] (f3),};
		\end{feynman}
		\end{tikzpicture}
		\caption{}
		\label{fig:decaya}
	\end{subfigure}
	\begin{subfigure}{.5\textwidth}
		\begin{tikzpicture}
		\begin{feynman}
		\vertex(a) {\(t\)};
		\vertex[right=of a] (b);
		\vertex[below right=3cm of b] (f1){\(b\)};
		\vertex[above right=of b] (c);
		\vertex[above right=1.5cm of c] (f2){\(q\)};
		\vertex[below right=1.5cm of c] (f3){\(\overline q\)};
		\diagram* {
		(a)-- [fermion] (b)-- [fermion] (f1),
		(b)-- [boson,edge label'=\(W\)] (c),
		(c)-- [fermion] (f2),
		(c)-- [anti fermion] (f3),};
		\end{feynman}
		\end{tikzpicture}
		\caption{}
	\end{subfigure}
	\caption{Feynman diagrams \cite{feynman} of a leptonically (a) and hadronically (b) decaying top quark.}
	\label{fig:decayb}
	\end{figure}
	
	\begin{figure}
		\centering
		\begin{tikzpicture}
		\begin{feynman}	
		\vertex[blob] (a) {};
		%\vertex (a);
		\vertex[right=of a] (b);
		\vertex[left=of a] (c);
		\vertex[above right=3.5cm of b] (bb){\(b\)};
		\vertex[right=of b] (bW);
		\vertex[left=of c] (cW);
		\vertex[below left=3.5cm of c] (cb){\(\overline b\)};
		\vertex[above right=of bW] (bq){\(q\)};		
		\vertex[below right=of bW] (baq){\(\overline q\)};				
		\vertex[above left=of cW] (cl){\(l\)};		
		\vertex[below left=of cW] (cnu){\(\overline \nu\)};
		\diagram* {
		(a)-- [fermion,edge label'=\(t\)] (b),
		(a)-- [anti fermion,edge label'=\(\overline t\)] (c),
		(b)-- [fermion] (bb),
		(b)-- [boson,edge label'=\(W^+\)] (bW),
		(c)-- [anti fermion] (cb),
		(c)-- [boson,edge label'=\(W^-\)] (cW),
		(bW)-- [fermion] (bq),
		(bW)-- [anti fermion] (baq),
		(cW)-- [fermion] (cl),
		(cW)-- [anti fermion] (cnu),
		};
		\end{feynman}
		\end{tikzpicture}
		\caption{Feynman diagram \cite{feynman} displaying an event from the lepton+jets channel. The centred circle indicates a mechanism to produce a $t\bar{t}$ pair.}
		\label{fig:semilep}
	\end{figure}
	
	Measuring the top quark mass is usually performed via a template fit. Therefore $t\bar{t}$ events are simulated for different masses $m_\text{top}^{MC}$. Then a selection is applied to data and simulation with the goal to select preferably $t\bar{t}$ events. Finally the different simulation with different $m_\text{top}^{MC}$ are fitted to data.	The most precise mass measurement of this kind is archived combining results of the CMS, ATLAS, CDF, and D0 collaborations. The result of this world average is a value of $173.34 \pm 0.27 (\text{stat}) \pm 0.71 (\text{sys})\;\text{GeV}$ \cite{topmass_combination}. Doing these kinds of measurements one relies on a correct simulation of $t\bar{t}$ events. Especially the mass parameter in Monte-Carlo generators can not be related to a mass one would calculate in an Lagrangian.
\section{Unfolding}
	Most analyses at LHC measure distributions of appropriate variables and then compare the obtained results in data with event simulations. In this method the MC samples also include detector effects. What one measures in this case is the real distribution on particle level folded with an unknown detector function. Studying the difference in MC between particle level and reconstruction level, it is possible to calculate the probabilities that a measured value in a bin $x_i$ is originating from bin $y_i$ on particle level. The resulting matrix can then be applied to real data to obtain data on particle level. The obtained distribution is corrected for detector effects by the unfolding and can then be compared with theory calculations. Additionally, measurements on particle level do not depend on simulation ambiguities like the mass parameter put into the Monte-Carlo generator.

	
% !TEX root = Master_Thesis.tex
\chapter{Measurement of the Top Quark Mass}
\label{ch:Measure}
	The top quark mass is an essential parameter to check the Standard Model of particle physics for consistency. For instance, the top quark has, due to its high mass, a large coupling to the Higgs boson. Therefore, it has to be included in correction terms for the Higgs boson mass. This makes it possible to relate the masses of Higgs boson, $W$ boson and top quark and thus validate measured values, as performed in \cite{ewfit}. In this chapter conventional mass measurement methods are presented and compared to the measurement method this analysis aims at. Additionally, a previous result from CMS as well as its possible improvements is discussed. 
	
\section{Conventional Mass Measurements}
	The most precise measurements of the top quark mass are performed by reconstructing and combining the top quark decay products to obtain the invariant mass of the top quark. Usually a template fit to the distribution of a variable sensitive to the mass is used, meaning a comparison to Monte Carlo simulations obtained by using different values of the top quark mass $m_\text{top}^{MC}$ of the event generator. A selection is applied to data and simulation with the goal to select $t\bar{t}$ events. The MC samples with different values of $m_\text{top}^{MC}$, including detector simulation, are then compared to data. The most precise mass measurement of this kind is achieved by combining results of the CMS collaboration. The result of this combination is a value of $172.44 \pm 0.13 (\text{stat}) \pm 0.47 (\text{sys})\;\text{GeV}$ \cite{topmass_combination}. \\
	A summary of top quark mass measurements is given in Fig.~\ref{fig:TopMasses}. As seen in Fig.~\ref{fig:TopMasses}, the most precise single measurements were performed in the lepton+jets channel. While the dilepton channel offers a good separation from background because of the two leptons, but both neutrinos have to be reconstructed from missing transverse energy to reconstruct the full top quark decay. Since the missing transverse energy only provides information of the sum of the neutrinos, a large uncertainty coming from the neutrino reconstruction limits the top quark mass measurement. The all-jets channel offers a good reconstruction because the all decay products can be reconstructed in the detector, but without a lepton in the final state, the separation from multijet events is difficult. The lepton-jets channel combines both advantages. The $t\bar{t}$ event can be distinguished from background with the lepton, but only one neutrino has to be reconstructed from missing transverse energy and a restraint on the lepton and neutrino reconstruct the $W$ boson mass.\\
	Performing these kinds of measurements, one relies on a correct simulation of $t\bar{t}$ events and is essentially measuring the top quark mass of a given simulation. This mass can not easily be related to a mass parameter in a Lagrange density because of non-perturbative contributions~\cite{tevjetmass} and is therefore referred to as the Monte Carlo mass $m_\text{top}^{MC}$. In addition, not all effects of QCD can be calculated \cite{nonperturbative} in simulation. This is why a cut-off scale \cite{cutoff} is defined which directly influences the mass associated with a given simulation (see section~\ref{sec:Simulation}). Because of this difficulty, this analysis aims to provide a measurement which can help to resolve ambiguities of the mass parameter in Monte Carlo simulation.
	\begin{figure}[tb]
	    \centering
		\includegraphics [width=.7\textwidth]{../Plots/TopMasses}
		\caption{Summary of previous top quark mass measurements. The most precise single measurement was performed by CMS. Taken from \cite{TopMasses}.}
		\label{fig:TopMasses}
	\end{figure}

\section{Measurement in Boosted Decays}
	The present analysis aims at a measurement of the top quark jet mass, which is independent of $m_\text{top}^{MC}$ used in simulations. The idea is to obtain a distribution with high sensitivity to the value of $m_\text{top}$, which can be calculated from first principle and thus be directly compared to analytical calculations. For lepton collisions it is shown in \cite{eejetmass} that a jet mass distribution can be calculated and its peak is sensitive to the top quark mass. These calculations are performed dividing the event in two hemispheres containing one top quark decay each. The jet mass is then the invariant mass of the sum of all particles in one hemisphere.\\
	Since events at the LHC contain many more objects not belonging to the $t\bar{t}$ system, jets with a finite cone size are defined. All decay products have to fit into the selected cone size, to reconstruct a top quark using a single jet. Therefore, Lorentz boosted top quarks are selected to obtain small distances between the decay products. The jet mass $m_\text{jet}$ in this analysis is then defined as the invariant mass of the jet four-vector,
	\begin{equation}
	m_\text{jet} = \sqrt{\left( \sum_{i} p_i \right)^2},
	\end{equation}
	where $i$ runs over all constituents of a jet and $p_i$ indicates the four-vector of the $i$-th constituent. After selecting suitable events, an unfolding is performed to obtain a jet mass distribution comparable to particle level as well as analytical calculations. 

\section{Previous Results}
	A measurement of the jet mass in highly boosted $t\bar{t}$ events \cite{torben_paper} has already been performed by CMS with the $8\;\text{TeV}$ dataset corresponding to an integrated luminosity of $19.7\;\text{fb}^{-1}$. The resulting differential cross section measurement and a summary of the uncertainties are shown in Fig.~\ref{fig:Torben1} and Fig.~\ref{fig:Torben2}, respectively. In this analysis, Cambridge/Aachen jets with a radius parameter of $R=1.2$ were used to reconstruct top quark decays, proving the capability of the method but not reaching high precision. Two main points were identified, which a subsequent analysis on $13\;\text{TeV}$ could largely improve: statistical uncertainties and effects due to pile-up resulting from the large jet radius. First, statistics are very high in the selected phase space on $13\;\text{TeV}$. Since the $t\bar{t}$ cross section increases by a factor of about three and about a factor of ten in the highly boosted regime with a centre-of-mass energy of $13\;\text{TeV}$, the statistical uncertainty is expected to decrease. This also influences the choice of jet algorithm and cone size. While in the $8\;\text{TeV}$ analysis large cones are crucial to obtain enough statistics, smaller cone sizes are possible on $13\;\text{TeV}$. The second limitation is pile-up. Because of the large cone size, pile-up largely influences the measurement, visible as a shift of the peak position in reconstructed events to values above the top quark mass. Again, a smaller cone size on $13\;\text{TeV}$ is expected to decrease this effect. In addition, various jet clustering algorithms can be studied to find the most suitable method to reconstruct top quark decays inside one jet.

	\begin{figure}[tb]
		\begin{subfigure}{.5\textwidth}
	    \centering
		\includegraphics [width=\textwidth]{../Plots/Torben/Torben_result_paper}
		\caption{}
		\label{fig:Torben1}
		\end{subfigure}
		\begin{subfigure}{.5\textwidth}
		\centering
		\includegraphics [width=\textwidth]{../Plots/Torben/Torben_error_paper}
		\caption{}
		\label{fig:Torben2}
		\end{subfigure}
		\caption{Unfolded cross section measurement~(a) and display of all uncertainties~(b) from a similar analysis performed at $8\;\text{TeV}$. Taken from \cite{torben_paper}.}
		\label{fig:Torben}
	\end{figure}	
	

\chapter{Experiment}
\section{Large Hadron Collider}
	The Large Hadron Collider (LHC) is a circular particle collider with a circumference of $27\;\text{km}$. At the LHC protons are accelerated to an energy of $6.5\;\text{TeV}$ and then brought to collision resulting in a center-of-mass energy of $\sqrt{s}=13\;\text{TeV}$. Because protons are composite particles, the actual collision involves quarks or gluons carrying only a fraction of this energy. 
	
	Protons are not injected directly into the LHC ring but run through several acceleration stages. The preaccelerators used for this purpose are old accelerators that can now be reused.
	
	Four Experiments are placed at each interaction point of LHC.
	
	A display of the LHC complex with its four experiments and preaccelerators is shown in fig. \ref{fig:lhc}.
	
	A very important parameter of a particle collider is its luminosity. It gives an estimate how many collisions take place per area and second. It is calculated vie Eq. \ref{eq:lumi}. To get an estimate of the number of events produced by a given process one has to multiply the integrated luminosity (see Eq. \ref{eq:intlumi}) with the cross section of this particular process (see Eq. \ref{eq:number})
	
	\begin{equation}
	L = \frac{}{}
	\label{eq:lumi}
	\end{equation} 
	\begin{equation}
	L_\text{int} = \int L dt
	\label{eq:intlumi}
	\end{equation} 
	\begin{equation}
	N = L_\text{int} \sigma
	\label{eq:number}
	\end{equation} 
	
	\begin{figure}
		\centering
		\includegraphics [width=\textwidth]{../Images/lhc.jpg}
		\caption{Display the LHC complex with the LHC ring itself and every accelerator used to bring protons to the required energy for injection in the LHC ring \cite{fig:lhc}}
		\label{SM}
	\end{figure}
	%todo LHC FAKTEN
	%todo Bild von LHC
	%todo Lumi einfuehren und Formel angeben
	%todo ANDERE EXPERIMENTE
\section{CMS Detector}
\label{sec:cms}
	The 'Compact Muon Solenoid' (CMS) experiment is a multi purpose detectors at LHC. It is designed to measure momentum and energy of particles produced in proton-proton interactions. The CMS detector is built in layers of subdetectors with different purposes described in following sections \ref{sec:tracker} - \ref{sec:muonsystem}.
	
	A very important part besides the detector systems is the Trigger, described in section \ref{sec:trigger}, providing fast decisions if an event is stored or discarded.
	%todo FULL VIEW OF CMS?
	%todo AUSMAßE
	%todo ZWIEBELFORM, ZYLINDER
	%todo PLOT: Collected Lumi von meinem Datensatz!

	\begin{figure}[tb]
		\centering
		\includegraphics [width=.8\textwidth]{../Plots/CMS_Slice.png}
		\caption{Slice through the CMS detector looking in direction of the beam pipe \cite{CMSslice}. Tracks of different particles on their way through the layers of the detector are displayed. From left to right, you can see the tracker, calorimeters, the superconducting solenoid and the muon system. These detector components, their purpose and how they work are described in the following sections \ref{sec:tracker} - \ref{sec:muonsystem}.}
		\label{CMS}
	\end{figure}
	
\subsection{Tracker}
\label{sec:tracker}
	%todo ABMAßE TRACKER, ANZAHL LAYER USW
	The purpose of the tracking system is to measure the momentum of created particles. It is taken advantage of the Lorentz force which changes the momentum of charged particles in a magnetic field. Paths of those particles are therefore bended with a bending radius proportional to the momentum in a constant magnetic field. Therefore a solenoid provides a constant magnetic field of about $4\;\text{T}$ inside the tracker. Per path one spatial point per layer is measured. These points are the input for a track finding algorithm. The largest uncertainties of the measured momenta rise from the spatial resolution of the tracker. Therefore a pixel structure is used which can measure the position of a passing charged particle within a resolution of ????.
	%todo GENAUIGKEIT
	%todo MATERIAL
	%todo ALGORITHMUS ANSPRECHEN
	All measured spatial points are then put into a track finding algorithm returning track and momentum of every recorded object.

\subsection{Calorimeters}
	To define the type of particle one has to measure not only momentum but also energy. Therefore outside the tracker a colorimetry system with two types of calorimeters is installed. The underlying principle of this subdetector is to absorb all of the energy a particle carries. This energy is then transformed into a light signal. The intensity of the light signal is proportional to the energy of the incoming particle. For this approach one needs to use scintillating materials. It also has to be dense enough to stop a incoming particle to transform all its energy into the light signal. At the end the light is collected and measured with photo multipliers.
	%FUNKTIONSWEISE 
\subsubsection{Electromagnetic Calorimeter}
	The inner part of the calorimetry system is supposed to measure the energy of electrons and photons. The electormagnetic calorimeter of the CMS detector is designed to deliver a very good spatial resolution. With this, photons which are not seen in the tracker can be reconstructed with a high precision.
	
	Since hadrons have a larger absorption length they mostly pass the electromagnetic calorimeter and are measured in the hadronic calorimeter.
	%MATERIAL
	%GROESSE
	%GENAUIGKEIT
\subsubsection{Hadronic Calorimeter}
	The hadronic calorimeter uses two alternating layers of different material. One is to absorb the energy of an incoming particle. The other one is made out of a scintillator.
	%GENAUIGKEIT
	%MATERIAL
	%GROESSE
	
\subsection{Solenoid}
	Outside the hadronic calorimeter a superconducting solenoid is installed and provides a magnetic field of about $4\;\text{T}$ inside the tracking system. The solenoid has a length of $13\;\text{m}$ and a diameter of $6\;\text{m}$. The purpose of the magnetic field is to bend paths of charged particles due to the Lorentz force. The bending radius then is directly connected to the momentum of the particle.  
\subsection{Muon System}
\label{sec:muonsystem}
	Because muons are not absorbed in the calorimeters, it is possible to use an additional tracking detector for muons at the very outside of the detector. Muon chambers are embedded in the iron return yoke of the magnet. Thus a magnetic field of about $2\;\text{T}$ is present.
	%AUFBAU MUON SYSTEM
	In cooperation with the inner tracker it is possible to reconstruct muon momenta with high precision.
\subsection{Trigger}
\label{sec:trigger}
	At the interaction points, proton bunches are brought to collision every $25\;\text{ns}$. Per crossing about ????
	%INTERACTIONNUMBER
	interactions take place. This adds up to about ???
	%COLLISIONS PER SECOND
	interactions per second. It is impossible to store data with this rate and the total required storage capacity would is not feasible either. Additionally it would need a tremendous number of computing cores to analyse this data. Therefore a trigger system is installed to select only the interesting events to reduce the amount of data. For the computation system it is required to reduce the amount from ???? to ????.
	%RATE VOR UND NACH TRIGGER
 	To decide which events are worth storing and which will be neglected different criteria are defined.
 	% TRIGGER MENU
 	% DIFFERENT TRIGGER LEVELS
	
	
\chapter{Reconstruction of Objects}
\label{ch:Reco}
	From all kinds of information provided by the detector systems has to be interpreted as physical objects in order to analyse the recorded data. Therefore algorithms are run to link the information from the detector to usable objects like muons, electrons or jets. Objects are defined by their detection in different detector systems. Paths of all detectable objects through the detector are sketched in Fig. \ref{fig:CMS_reco}.


\section{Particle Flow}
\label{sec:pf}
	CMS uses a special algorithm called particle flow \cite{particleflow} to combine information from tracker and calorimeters and thus reconstruct objects to a high precision. Tracks from the inner tracker are extrapolated into the calorimeters. If a shower fits to the track, information from these two sub detectors are combined. Since only charged particles will interact with the tracking system, showers from neutral hadrons and photons cannot be associated with a track.
	\begin{figure}[tb]
		\centering
		\includegraphics [width=.75\textwidth]{../Images/CMS_Slice_white.png}
		\caption{Slice through the CMS detector looking in direction of the beam pipe. Tracks of different particles on their way through the layers of the detector are displayed. From left to right, you can see the tracker, calorimeters, the superconducting solenoid and the muon system. Taken from \cite{CMSslicewhite}.}
		\label{fig:CMS_reco}
	\end{figure} 
	
\section{Muon Identification}
	Muons have a very low probability to interact with the Calorimeters of CMS. Because of that, muons are identified by hits in the tracker and muon chambers. Three working points are defined by the muon reconstruction efficiency. In this analysis the tight working point \cite{MuonID} is used, providing the lowest efficiency but purest muon selection. The criteria a muon candidate has to pass to fulfil the tight working point are:
	\begin{itemize}
	\item the candidate is reconstructed as a Global Muon 
	\item the candidate is reconstructed with the particle flow algorithm 
	\item the global-muon track fit returns $\chi^2/ \text{ndof} < 10$, where ndof are the number of degrees of freedom
	\item at least one muon-chamber hit included in the global-muon track fit 
	\item the candidate has to be hits in at least two muon stations
	\item Its tracker track has transverse impact parameter $d_{xy} < 2\;\text{mm}$ with respect to the primary vertex 
	\item the longitudinal distance of the tracker track with respect to the primary vertex is $d_{z} < 5\;\text{mm}$
	\item at least one hit in the pixel tracker 
	\item at least hits in five different tracker layers
	\end{itemize}
	If a candidate fulfils these criteria, it is stored and called muon in this analysis.
\section{Electron Identification}
	Electrons are reconstructed in the tracker and electromagnetic calorimeter. Due to the magnetic field, electrons will take a curved path through the tracking system. They are then stopped in the ECAL where their energy is measured.

\section{Jets}
	Jets are objects, used to reconstruct quarks. Because of confinement quarks cannot exist isolated but hadronize. This results in a particle shower consisting of hadrons. To reconstruct the initial quark one needs to sum up all particles from the final shower. To combine the tracks in a well defined way jet algorithms are used. Two important requirements for an jet algorithm are to be infrared and collinear safe. The first criterion means, a jet should not change when including or excluding soft radiation. Collinear safety addresses collinear splitting of particles in a jet which should also not change the jet. All presented jet algorithms in this thesis fulfil these requirements.
\subsection{Anti-$k_T$ and Cambridge/Aachen Jet Algorithms}
	In CMS the common way to cluster jets from the detected particles is to use iterative jet algorithms. Thus particles are clustered step by step until an abort criterion is reached. The most common group of iterative jet algorithms are $k_T$-like algorithms. As an input, a list of objects reconstructed in the detector is given. The Algorithm then calculates two quantities $d_{ij}$ (Eq. \ref{eq:dij}) and $d_{iB}$ (Eq. \ref{eq:iB}) for each pair of objects $i$ and $j$:
	\begin{equation}
	d_{ij} = min (k_{T,i}^{n}, k_{T,j}^{n})  \frac{\Delta R_{ij}^2}{R^2}
	\label{eq:dij}
	\end{equation}
	\begin{equation}
	d_{iB} = k_{T,i}^{-n}.
	\label{eq:iB}
	\end{equation}
	Here, $d_{ij}$ describes a effective distance between two objects $i$ and $j$, $d_{iB}$ is a distance measure from object to beam axis. The variable $k_T$ is the transverse momentum of an object, $\Delta R_{ij}$ denotes the distance between objects $i$ and $j$ and $R$ is a constant parameter that defines the radius of the resulting jet. Changing the exponent $n$ influences the order of clustering. If $n=-2$, the algorithm is called Anti-$k_T$ \cite{antikt} and particles with high transverse momenta are clustered first. Using the Cambridge/Aachen \cite{CA1, CA2} algorithm, and therefore choosing $n=0$ will not weight the objects by momenta and provides a pure geometrical measure. When $d_{ij}$ is smaller than $d_{iB}$ both objects $i$ and $j$ are combined and both quantities are calculated again. At some point, $d_{iB}$ will be smaller than any $d_{ij}$, then object $i$ is called a jet and is removed from the list of objects. This procedure is repeated until the list of objects is empty. With the anti-$k_T$ jet algorithm, particles with large transverse momenta are clustered first because $d_{ij}$ (Eq. \ref{eq:dij}) is easily smaller then $d{iB}$ for large $p_T$. The resulting jets are very circular. 
	%todo plot mit AK shape.
	%todo KT auch erwähnen?

\subsection{HOTVR Jet Algorithm}
	Another approach to cluster jets is the 'heavy object tagger with variable R' (HOTVR) \cite{hotvr}. This algorithm does not use a constant radius parameter $R$ but a $p_T$ dependent effective $R=R_\text{eff}(p_T)$ (see Eq. \ref{eq:HOTVR}). Thus the $R_\text{eff}$ decreases with increasing $p_T$ leading to smaller jets when the decay products are expected to be more close because of the Lorentz boost. The $p_T$ dependence is scaled with a parameter $\rho$ with a default value of $600\;\text{GeV}$. Furthermore, upper and lower boundaries for the jet radius can be set. The default values are $R_\text{min} = 0.1$ and $R_\text{max} = 1.5$. 	
	\begin{equation}
	\label{eq:HOTVR}
	  R_\text{eff} =
	   \begin{cases}
	     R_\text{min} & \text{for } \rho / p_T < R_\text{min} \\
	     R_\text{max} & \text{for } \rho / p_T > R_\text{max} \\
	     \rho / p_T & \text{else}  
	   \end{cases}
	\end{equation}
	
	\noindent This effective $R$ is then used with the equations of the Anti-$k_T$ algorithm described earlier (see Eq. \ref{eq:dij} - \ref{eq:iB}). Additionally, a mass-jump criterion is used.
	%todo mass jump Kriterium, auch mit Formel
	

\subsection{XCone Jet Algorithm}
\label{sec:xcone}
	XCone \cite{xcone} is an exclusive jet algorithm, returning conical jets. It is well suited for analysis where the final state and therefore the expected number of jets is known since it returns a fixed number of jets. Thus, a physical final state has direct influence on the jet finding. \\
	Starting with a fixed number of jet axes $N$ the algorithm calculates the direction of these axes by minimizing the N-jettiness variable. N-jettiness is a measure for how N-jet-like an event looks. The definition is shown in Eq. \ref{njettines}.
	\begin{equation}
	\tilde{\tau}_N = \sum_i \min\{\rho_\text{jet}(p_i, n_1), \dots, \rho_\text{jet}(p_i, n_N), \rho_\text{beam}(p_i)\}
	\label{njettines}
	\end{equation}
	Once the minimizing process converges, all particles inside a radius $R$ from a jet axis are added to one jet. Here, it is important to mention that XCone handles overlapping jets differently than common iterative jet algorithms. HOTVR and the Anti-$k_T$ algorithm remove already clustered objects from the input list. Thus, one of two nearby jets is crescent shaped while the other one is circular. XCone however, will assign every object to the closest jet resulting in a straight border between two jets. This feature is shown in Fig. \ref{fig:XCone_overlap}. 
	\begin{figure}[tb]
		\centering
		\includegraphics [width=.6\textwidth]{../Plots/XCone_Overlap.png}
		\caption{The separation of nearby jets clustered with the XCone jet algorithm in a $t\bar{t}$ event, requiring six jets with a radius of $0.5$, is depicted. The area of jets is shown in the $\phi$-$y$-plane. Here $y$ indicates the rapidity. Taken from \cite{xcone}.}
		\label{fig:XCone_overlap}
	\end{figure} 
	A more general advantage of XCone is that the N-jettiness variable is often used to do theory calculations of particle physics events, the XCone algorithm is easier to include in these calculations.
	
\subsection{Jet Energy Corrections}
\label{sec:jec}
	Jets found by clustering sequences are furthermore corrected to account for non linearities in the detector response and differences between data and simulation because of modelling \cite{JEC}. The approach is to factorize different sources of variance and scale momentum vectors of jets with a factor addressing each source. Thus, four factors are introduced which then result in one final $p_T$ and $\eta$ dependent correction factor as shown in Eq. \ref{eq:jec} and \ref{eq:cjec}.
	\begin{equation}
	p^{\text{corrected}} = C_{\text{JEC}} \cdot p^{\text{raw}}
	\label{eq:jec}
	\end{equation}
	where
	\begin{equation}
	C_{\text{JEC}} = C_{\text{offset}} \cdot C_{\text{MC}} \cdot C_{\text{rel}} \cdot C_{\text{abs}}
	\label{eq:cjec}
	\end{equation}
	Firstly, a factor is applied to address the additional energy in a jet because of pile-up ($C_{\text{offset}}$), which is not taken care of in simulation. The factor is constructed $p_T$ dependent to subtract a constant energy from a jet in data. Secondly, reconstructed jets are corrected to match the generated jet momentum. This is done by calculating a factor $R=\frac{p_T^{\text{rec}}}{p_T^{\text{gen}}}$ in different $p_T$ and $\eta$ regions and apply this to reconstructed jets as $C_{\text{MC}} = \frac{1}{<R>}$, where $<R>$ indicates a mean in the given region. Next, a relative correction $C_{\text{rel}}$ is applied which accounts for non linearities in the detector response. Therefore, a $\eta$ dependent correction factor is used. Lastly, an absolute factor $\cdot C_{\text{abs}}$ derived from data in $Z$ + jets and $\gamma$ + jets events to fit the absolute energy scale.

\subsection{b-Jets}
\label{sec:btag}
	In this analysis jets originating from a bottom quark are identified to reduce background. To identify a jet as an b-jet the "Combined Secondary Vertex" (CSV) algorithm is used. Since b-hadrons have a large lifetime of $1.5\;\text{ps}$ they travel about $450\;\text{\textmu m}$ in the detector before decaying. This leads to a secondary vertex at the spatial point where the hadron decays which can be reconstructed in the tracking system. Additionally the composition of hadrons in a b-jet is different from other jets.
	%todo ALLES AUßER secondary vertex NOCHMAL NACHLESEN 	
	%todo working points beschreiben
	%todo Quelle
	These properties are taken advantage of in the CSV algorithm. 
	
\section{$\cancel{E}_T$ and $S_T$}
	With the information of mentioned objects two important variables are defined. The missing transverse energy $\cancel{E}_T$ is defined to estimate the energy carried away by particles which leave the experiment undetected. Summing up all transverse momenta of each particle in the final state returns the $p_T$ of the system in the initial state which is $p_T = 0$ at LHC. Thus the transverse energy of all undetected particles is defined as the absolute value of the negative sum over all transverse momenta of detected objects (Eq. \ref{eq:MET}). Since every object, independent from it's $\eta$, has to be taken into account for this variable, $\cancel{E}_T$ is independent from the selection applied. The missing energy is due to neutrinos which can not be detected with CMS because of their low probability to interact with the detector material or a new physics state. 
	
	\begin{equation}
		\cancel{E}_T = \left| - \sum_{\textrm{leptons, jets}}^{} \vec{p}_T \right|
		\label{eq:MET}
	\end{equation}
	
	\noindent Another important variable to describe an event is $S_T$. As represented in Eq. \ref{eq:ST} is defined as the scalar sum of all transverse momenta of reconstructed objects plus the missing transverse energy mentioned above. Different from $\cancel{E}_T$ only objects surviving the selection are considered. $S_T$ gives an estimate how much energy one event contains and is therefore used to distinguish between low energy QCD and high energy processes. 
	
	\begin{equation}
		S_T = \left( \sum_{\textrm{leptons, jets}}^{} \vert \vec{p}_T \vert \right) + \cancel{E}_T
		\label{eq:ST}
	\end{equation}
	
	\noindent Additionally, a definition for $S_T$ which just takes leptonic activity into account is used in this analysis. It is similarly defined and referred to as $S_T^{\text{lep}}$ (see Eq. \ref{eq:STlep}).
	
	\begin{equation}
		S_T^\text{lep} = \left( \sum_{\textrm{leptons}}^{} \vert \vec{p}_T \vert \right) + \cancel{E}_T
		\label{eq:STlep}
	\end{equation}	

\chapter{Data and Simulation}
\label{ch:MC}
	In almost every particle physics analysis data is compared to simulations. Physicists have to rely on simulations because it is not possible to purely calculate the outcome one sees in the detector. In this chapter the used data set is described as well as the production of a Monte Carlo Simulation of a given process.
	\section{Data}
	This thesis analyses data recorded by the CMS detector in 2016 at a centre-of-mass energy of $13\;\text{TeV}$. The size of the data set corresponds to an integrated luminosity of $37.76\;\text{fb}^{-1}$. A detailed view of the data taking is presented in Fig. \ref{fig:CMS_lumi}. Data used in CMS analyses is required to pass a quality criterion. For this, only data from selected runs are used where the accelerator and detector worked properly.
	\begin{figure}[tb]
		\centering
		\includegraphics [width=.5\textwidth]{../Plots/CMS_Lumi.pdf}
		\caption{Integrated luminosity recorded by CMS (orange) in comparison with the data delivered by LHC (blue). The growth of the dataset corresponding to $37.76\;\text{fb}^{-1}$ is shown for the data taking period of 2016. Taken from \cite{CMSlumi}.}
		\label{fig:CMS_lumi}
	\end{figure}



	\section{Monte Carlo Samples}
	Additionally Monte Carlo (MC) simulations are used to be able to see influences of selections in all relevant processes. Simulations provide an estimate of the outcome of a collision as it would be seen in the detector. This level is called reconstruction or detector level throughout this analysis. Furthermore, simulations also provide information on particle level, meaning the physical process after hadronisation but without detector effects and pile-up. This is referred to as particle or generator level. The most important simulation for this analysis is of course the $t\bar{t}$ sample which will be used to provide an input for the unfolding procedure. Since the lepton+jets channel is selected, processes that have a similar final state (exactly one lepton and (b-) jets) have to be included. The background processes considered in this work are: Single Top production, $W$+jets production, $Z$+jets production, Diboson production and general QCD events. These processes are relevant in the final phase space because leptons are possible in the final state. Most dominant processes are expected to be $W$+jets and Single Top production, because they provide a process with exactly one detectable lepton in the final states and additional jets. $Z$+jets production is reduced by vetoing additional leptons. Diboson production has a very low cross section, but $WW$ and $WZ$ can lead to a similar final state as $t\bar{t}$. QCD is expected to produce mostly leptons with small transverse momenta and is therefore reduced by selecting a lepton in the high energy region. Table \ref{MC_Tab} summarises all MC samples used with additional information like MC generator, cross section and number of events one sample contains. Every process is split into different regions to obtain enough statistics in high energy regions.	
	\begin{table}
	\centering
	 \begin{tabular}{l l c r@{.}l r }
	 	%\hline
	 	Process & Sample  & MC Generator &  \multicolumn{2}{r}{Cross Section [pb]} & Number of Events \\
	 	\hline
	 	\hline
	 	$t\bar{t}$ & $0 < M_{t\bar{t}} < 700$ & POWHEG & 831 & 76 & 77932119 \\
	 	           & $700 < M_{t\bar{t}} < 1000$ & " & 76 & 605 & 38219132 \\
	 	           & $1000 < M_{t\bar{t}} < \infty$ &  " & 20 & 578 & 24480678 \\
	 	\hline
		Single Top & t-channel  & POWHEG & 136 & 0 & 5993570 \\
		           & t-channel (anti top) & " & 80 & 95 & 3927980\\
		           & $tW$ and $\bar{t}W$ & " & 71 & 20 & 13875810 \\
		           & s-channel & MADGRAPH & 3 & 36 & 3370581 \\
		\hline
		$W$+jets & $100 < p_T < 250$ & MADGRAPH  & 676 & 3 & 176792599561 \\
	 	         & $250 < p_T < 400$ & " & 23 & 9 & 617720200 \\
	 	         & $400 < p_T < 600$ & " & 3 & 03 & 11690912 \\
	 	         & $600 < p_T < \infty$ & " & 0 & 45 & 1775927 \\
	 	\hline
	 	$Z$+jets & $70 < S_T < 100$ & MADGRAPH & 215 & 62 & 9608508 \\
	 	         & $100 < S_T < 200$ & " & 181 & 3 & 10606926 \\
	 	         & $200 < S_T < 400$ & " & 50 & 42 & 9646008 \\
	 	         & $400 < S_T < 600$ & " & 6 & 98 & 10008141 \\
	 	         & $600 < S_T < 800$ & " & 1 & 68 & 8292160 \\
	 	         & $800 < S_T < 1200$ & " & 0 & 78 & 2668311 \\
	 	         & $1200 < S_T < 2500$ & " & 0 & 19 & 595906 \\
	 	         & $2500 < S_T < \infty$ &"  & 0 & 004 & 399147 \\
	 	\hline
	 	QCD             & $15 < p_T < 20$ & PYTHIA8 & 3819570 & & 4141208 \\
		(muon enriched) & $20 < p_T < 30$ & " & 2960198 & &  31475095 \\
		                & $30 < p_T < 50$ & " & 1652471 & & 29944719\\
		                & $50 < p_T < 80$ & " & 437504 & & 19806515 \\
		                & $80 < p_T < 120$ & " & 106033 & & 13778177 \\
		                & $120 < p_T < 170$ & " & 25190 & & 8042660 \\
		                & $170 < p_T < 300$ & " & 8654 & & 7946703 \\
		                & $300 < p_T < 470$ & " & 797 & 4 & 7936465 \\
		                & $470 < p_T < 600$ & " & 79 & 03 & 3850466 \\
		                & $600 < p_T < 800$ & " & 25 & 10 & 4008200 \\
		                & $800 < p_T < 1000$ & " & 4 & 71 & 3959757 \\
		                & $1000 < p_T < \infty$ & " & 1 & 62 &  3976075 \\
		\hline
	 	Diboson & $WW$ & PYTHIA8 & 118 & 7 & 994017 \\
	 			& $WZ$ & " & 47 & 13 & 990003 \\
	 	        & $ZZ$ & " & 16 & 52 & 993154 \\
	 	 \hline
	 \end{tabular}
	\caption{Summary of MC samples used in this analysis. Assumed cross section, MC generator and number of events are displayed for each sample. All processes are further divided into numerous independent samples to obtain enough statistics, even in high energy regions.}
	\label{MC_Tab}	
	\end{table}

\chapter{Analysis}
\label{ch:Ana}
	This chapter will cover the analysis performed for this thesis. A basic idea of the goals and strategy of this analysis is given in section \ref{sec:strategy}. Previous results, this analysis refers to, are presented in the following. Then a detailed look into event selections, studies on particle level, differences between jet algorithms and finally the unfolding process and its results follows in sections \ref{sec:jet_studies} to \ref{sec:results}.
\section{Analysis Strategy}
\label{sec:strategy}
	This analysis aims for boosted $t\bar{t}$ where all decay products from the top decays merge into a single jet. To measure in this phase space a selection of events is applied. The detailed selection is presented in section \ref{sec:selection}. In the required phase space the distribution of the jet mass of a top quark decaying into quarks ($t\rightarrow W^{+} b \rightarrow b q \bar{q}'$) is measured. Following a unfolding is performed using the TUnofld \cite{tunfold} software package. The goal is to compare data unfolded to particle level with first principle calculations. To fins a jet algorithm that fits this purpose, studies on particle level are presented in section \ref{sec:jet_studies}.
	

\section{Jet Studies on Particle Level}
\label{sec:jet_studies}
	For this analysis it is crucial to choose a suitable jet algorithm and cone size. The previous mentioned analysis \cite{torben_paper} uses Cambridge-Aachen jets with the radius parameter set to $R=1.2$ for its measurement. This large radius was chosen to compensate for low statistics in the boosted $t\bar{t}$ regime. Since the cross-section of $t\bar{t}$ production is much higher at a center-of-mass energy of $13\;\text{TeV}$ a smaller cone is expected to be applicable for this analysis. Additionally jets clustered with Anti-$k_T$, HOTVR and XCone algorithms are studied and optimized in sections \ref{sec:AKHOTVR} as well as \ref{sec:XCone_strat} and finally compared (section \ref{sec:jet_comp}) to find the algorithm most suitable for this analysis. The goal is to select a jet algorithm which returns jets in which all decay products of a hadronically decaying top quark are merged. In this case, the jet mass $M_\text{jet}$ is sensitive to the top quark mass $M_\text{top}$. To be able to extract the top quark mass, the jet mass distribution should return a sharp peak at the top quark mass of around $173\;\text{GeV}$. The jet studies are performed with a $t\bar{t}$ simulation using the information of MC simulations at particle level. The detailed selection is described in Section \ref{sec:GenSel}.

\subsection{Selection on Particle level}
\label{sec:GenSel}
	All studies on particle level are performed with a $t\bar{t}$ sample only. Several selection criteria are used to select boosted top quark decays in the lepton + jets channel. Since the selection is applied on particle level, also particle level information is used. The selection reads:
	\begin{itemize}
	\item direct selection of lepton + jets channel
	\item hadronically decaying top quark with $p_T > 300\;\text{GeV}$
	\item exactly one electron or muon
	\item veto on additional leptons
	\item $p_T^{\text{1st jet}} > 400\;\text{GeV}$ 
	\item $p_T^{\text{2nd jet}} > 200\;\text{GeV}$ 
	\item Veto on additional jets with $p_T > 200\;\text{GeV}$ 
	\item $\Delta R (\text{lepton, 2nd jet}) < \text{jet radius}$
	\item $M^{\text{1st jet}} > M^{\text{2nd jet + lepton}}$
	\end{itemize}
	%todo pt muon/elec, eta ranges
	Where the first jet refers to the leading jet in $p_T$ of the respective jet clustering algorithm. It is expected to be originating from the hadronically decaying top quark, the second one is expected contain the products of the leptonically decaying top quark. The purpose of the $p_T$ thresholds is to select boosted top decays. The Veto on additional jets is set to select $t\bar{t}$ events where one jet per top quark decay is expected. It is to mention, that the veto is not present in combination with XCone since it will always return exactly two jets in the used set up. A cut on the distance between lepton and second leading jet in $p_T$ prefers boosted topologies where the lepton is inside the jet of the leptonically decaying top quark. This criterion is also obsolete for XCone because of the selected clustering sequence (see section \ref{sec:XCone_strat}). To suppress events where not all decay products of the hadronically decaying top quark end up in the jet with the highest transverse momentum, a mass criterion $M^{\text{1st jet}} > M^{\text{2nd jet + lepton}}$ is set. The criterion includes the assumption that the mass of the jet on the leptonic side is lower because the neutrino cannot be reconstructed. This selection is applied to every jet algorithm output to be able to compare these different approaches.

\FloatBarrier %draw figures of previous section before the new one starts	
\subsection{Studies with Anti-$k_T$ and HOTVR}
\label{sec:AKHOTVR}	
	Since the top quark decay should be reconstructed with one jet, all decay products need to lay inside the defined jet cone. Choosing different cone sizes has various effects. When the cone is small, not all decay products may end up in the jet and the jet mass is reconstructed smaller than the top mass. If the cone size is large, the probability of additional radiation and pile-up grows and the resulting jet mass is reconstructed too high. 
	\subsubsection{Anti-$k_T$ Jets}
	As a starting point Anti-$k_T$ jets with a radius of $0.8$ are selected since this is the CMS intern standard to reconstruct top quark jets. To study the influence of the cone size, AK jets with a radius of $0.8$ and $1.2$ are presented in Fig. \ref{fig:GEN_AK08} and \ref{fig:GEN_AK12}, respectively. Additionally a matching is performed. If all three decay products of the top quark are clustered into the jet, the jet is called 'matched'. One can see that AK8 jets tend to deliver a jet mass lower than the top quark mass of about $173\;\text{GeV}$. This is due to the higher fraction of 'not matched' events. In this case, these are events where not every decay product ends up in the jet. AK12 jets on the other hand often return a mass higher than the top quark mass which is due to underlying event effects. A larger cone size has a higher probability of including particles not originating from the top quark one is interested in. Furthermore, even the peak position for AK12 jets is reconstructed above the top quark mass. It is also to mention that with a larger cone size more events survive the selection criteria because a large jet sums up more particles and thus more transverse momentum. Because of the better resolution in the peak region and much less tail, AK4 fits the purpose of this analysis well and is further used. To reduce additional energy clustered into the jet further, grooming algorithms like soft drop are used. A comparison of AK8 jets with and without soft drop is depicted in Fig. \ref{fig:GEN_AK08sd}. With soft drop applied, the $W$ peak can be clearly identified. Additionally, masses above the top quark mass are reduced. The distribution with AK8 jets and soft drop (Fig. \ref{fig:GEN_AK08sd1}) will be compared to other clustering methods in section \ref{sec:jet_comp}.

	\begin{figure}[tb]
		\begin{subfigure}{.5\textwidth}
	    \centering
		\includegraphics [width=\textwidth]{../Plots/GenStudies/AK08_matching}
		\caption{}
		\label{fig:GEN_AK08}
		\end{subfigure}
		\begin{subfigure}{.5\textwidth}
		\centering
		\includegraphics [width=\textwidth]{../Plots/GenStudies/AK12_matching}
		\caption{}
		\label{fig:GEN_AK12}
		\end{subfigure}
		\caption{Comparison of jet mass distributions of AK8 (a) and AK12 (b) jets. A smaller cone size (a) leads to a lower reconstructed mass while a large cone (b) returns higher masses. The fraction of 'matched' and 'not matched' events is shown in the histograms.}
	\end{figure}
	
	\begin{figure}[tb]
		\begin{subfigure}{.5\textwidth}
	    \centering
		\includegraphics [width=\textwidth]{../Plots/GenStudies/AK08softdrop_matching}
		\caption{}
		\label{fig:GEN_AK08sd1}
		\end{subfigure}
		\begin{subfigure}{.5\textwidth}
		\centering
		\includegraphics [width=\textwidth]{../Plots/GenStudies/AK08_matching}
		\caption{}
		\label{fig:GEN_AK08sd2}
		\end{subfigure}
		\caption{Comparison of jet mass distributions of AK8 with (a) and without (b) the soft drop algorithm applied. }
		\label{fig:GEN_AK08sd}
	\end{figure}
		
	\subsubsection{HOTVR Jets}
	Another approach to find a appropriate cone size is to used not a constant but $p_T$ dependent radius parameter. Since decay products are Lorentz boosted with high momentum, the higher the $p_T$, the smaller cone size is necessary to contain all decay products. Thus, HOTVR (see section \ref{sec:HOTVR}) directly addresses this property. The default settings set the effective radius to $R_\text{eff} = \frac{600\;\text{GeV}}{p_T}$, which corresponds to a maximum radius of $1.5$ with the used selection of jets with $p_T > 400\;\text{GeV}$. The result is visible in Fig. \ref{fig:GEN_HOTVR}. Due to the large cone size for the lowest possible jet momentum, the distribution is very similar to a AK jet with a large radius parameter. The parameter to tune the HOTVR clustering is $\rho$. By lowering $\rho$, the effective radius shrinks. Figure \ref{fig:GEN_HOTVRrho} shows a comparison of the jet mass for different $\rho$. It is decreased in $100\;\text{GeV}$ steps to a value of $300\;\text{GeV}$ which corresponds to a radius of $0.75$ for jets with a transverse momentum of $400\;\text{GeV}$. A behaviour similar to decreasing the radius of Anti-$k_T$ jets is visible. Increasing $\rho$ returns more events where all decay products end up in the jet cone, but a jet also includes more additional particles leading to a higher mass.
	%todo Entscheidung welcher HOTVR benutzt wird

	\begin{figure}[tb]
		\centering
		\includegraphics [width=.5\textwidth]{../Plots/GenStudies/HOTVR_matching}
		\caption{Jet mass distribution of jets clustered with the HOTVR algorithm. Here, default values of the clustering procedure are used. The large tail can be explained with large jet radii for jets with a transverse momentum around $400\;\text{GeV}$.}
		\label{fig:GEN_HOTVR}
	\end{figure}	
	%todo caption
	
	\begin{figure}[tb]
		\begin{subfigure}{.5\textwidth}
	    \centering
		\includegraphics [width=\textwidth]{../Plots/GenStudies/HOTVRrho600_matching}
		\caption{}
		\label{fig:GEN_HOTVR600}
		\end{subfigure}
		\begin{subfigure}{.5\textwidth}
		\centering
		\includegraphics [width=\textwidth]{../Plots/GenStudies/HOTVRrho500_matching}
		\caption{}
		\label{fig:GEN_HOTVR500}
		\end{subfigure}
		\begin{subfigure}{.5\textwidth}
	    \centering
		\includegraphics [width=\textwidth]{../Plots/GenStudies/HOTVRrho400_matching}
		\caption{}
		\label{fig:GEN_HOTVR400}
		\end{subfigure}
		\begin{subfigure}{.5\textwidth}
		\centering
		\includegraphics [width=\textwidth]{../Plots/GenStudies/HOTVRrho300_matching}
		\caption{}
		\label{fig:GEN_HOTVR300}
		\end{subfigure}		
		\caption{Study of the influence of the HOTVR $\rho$ parameter on the jet mass.}
		\label{fig:GEN_HOTVRrho}
	\end{figure}
	%todo caption

\FloatBarrier %draw figures of previous section before the new one starts		
\subsection{Studies with XCone}
\label{sec:XCone_strat}
	The XCone jet algorithm described in section \ref{sec:xcone} has already been tested resolving $t\bar{t}$ decays. Studies for hadronically decaying top quark pairs are presented in a paper from Thaler and Wilkason \cite{xconetop}. Here, the XCone algorithm is tuned to the $t\bar{t}$ final state, expecting six jets. Using the information that it is expected to find three jets from each top quark, a promising approach to reconstruct the top quark decays was made with a strategy using two clustering steps (see Fig. \ref{fig:JetDisplay}). Firstly, XCone is required to find exactly two jets with a large radius ensuring that all decay products of the top quark end up in the jet. Thus, every particle from the hard scattering should be clustered into one of the jets. The goal of this first step is to separate the two top quarks into independent jets.  Now, the jets are identified if they contain the decay products of the hadronically decaying or leptonically decaying top quark via a distance measure $\Delta R (\text{lepton, jet})$. To check if the categorization works, the distance $\Delta R$ between the hadronically decaying top quark on particle level and the selected jet is calculated. The distribution is shown in Fig. \ref{fig:XCone_dR}. According to this plot, almost every jet is categorized correctly. The jet shape in the $\eta$-$\phi$-plane, identification (orange indicates the jet identified as originating from the hadronically decaying top quark) and all particles in the event are shown in Fig. \ref{fig:JetDisplay1}. After that, the jets are further divided into smaller subjets. Since one expects only two visible components on the leptonic side, only two subjets are required, while the other jet contains three. This strategy will be referred to as '$2+5$' and is depicted in Fig. \ref{fig:JetDisplay2}. The subjets are then combined to form a final jet that is used from now on.	
	
	\begin{figure}[tb]
		\centering
		\includegraphics [width=.5\textwidth]{../Plots/GenStudies/XCone_dR_GEN_R20}
		\caption{Check of categorisation of XCone jets. The distance between generated hadronically decaying top quark and jet identified as containing its decay products is shown. The distribution is expected to peak at low values if the categorisation works, which is the case.}
		\label{fig:XCone_dR}
	\end{figure} 
	\begin{figure}[tb]
		\begin{subfigure}{.5\textwidth}
	    \centering
		\includegraphics [width=\textwidth]{../Plots/JetDisplayR15/xcone_incjets_event04}
		\caption{}
		\label{fig:JetDisplay1}
		\end{subfigure}
		\begin{subfigure}{.5\textwidth}
	    \centering
		\includegraphics [width=\textwidth]{../Plots/JetDisplayR15/xcone_subjets_event04}
		\caption{}
		\label{fig:JetDisplay2}
		\end{subfigure}
		\caption{Display of the jet area from XCone jets clustered with the '$2+5$' approach after the first (a) and the second step (b) . The grey dots show particles identified by the PF algorithm, red circles indicate decay products from the hadronically decaying top quark where the bottom quark is additionally marked. Decay products from the leptonically decaying top quark are marked as follows: the black circle with star illustrates the bottom quark, the black circle with plus sign shows the lepton and the white circle marks the neutrino.}
		\label{fig:JetDisplay}
	\end{figure}
	
	Finally, the cone sizes have to be chosen. The subjets are set to a radius of $R=0.4$ to be comparable to the CMS standard for subjets, which are AK4 jets. To determine the most suitable radius for the first clustering step, studies of the jet mass are made. A comparison of the jet area in Fig. \ref{fig:JetDisplayR} shows a higher expected misidentification rate for larger cones because of additional radiation or underlying event which leads to high masses especially when yet all decay products end up in the subjets. On the other hand, a small cone may not include all decay products. This effect can also be seen in the jet mass distributions in Fig. \ref{fig:XConeR1}. Based on these studies, the radius parameter is chosen to be $R=1.2$ because of less events in the shoulder around the $W$ mass and a smaller tail.

	%todo sagen, dass das gut funktioniert
	To perform the clustering with data, an easier method is tested, where both fat jets contain three subjets '$2+6$'. Afterwards the final jets are analogously categorized into a jet originating from the hadronically and the leptonically decaying top quark. A comparison between the two methods (see Fig \ref{fig:GEN_XCone_comp}) shows that both return almost the same distribution. Thus, the '$2+6$' method is chosen to represent the XCone result.
 	%todo show deltaR(lepton, fatjet) to show, that categorisation into had and lep works
	
	
	\begin{figure}[tb]
		\begin{subfigure}{.5\textwidth}
	    \centering
		\includegraphics [width=\textwidth]{../Plots/JetDisplayR10/xcone_subjets_event09}
		\caption{}
		\end{subfigure}
		\begin{subfigure}{.5\textwidth}
	    \centering
		\includegraphics [width=\textwidth]{../Plots/JetDisplayR20/xcone_subjets_event09}
		\caption{}
		\end{subfigure}
		\caption{Jet area comparison between a small (a) and a large $R_1$ (b). The small cone only barely contains all decay products while the larger cone leads to misidentification of particles not belonging to the top quark decay.}
		\label{fig:JetDisplayR}
	\end{figure}	
		
	\begin{figure}[tb]
		\begin{subfigure}{.5\textwidth}
  		\centering
		\includegraphics [width=\textwidth]{../Plots/GenStudies/XCone_GEN_R10}
		\caption{}
		\end{subfigure}
		\begin{subfigure}{.5\textwidth}
  		\centering
		\includegraphics [width=\textwidth]{../Plots/GenStudies/XCone_GEN_R12}
		\caption{}
		\end{subfigure}
		\begin{subfigure}{.5\textwidth}
  		\centering
		\includegraphics [width=\textwidth]{../Plots/GenStudies/XCone_GEN_R15}
		\caption{}
		\end{subfigure}
		\begin{subfigure}{.5\textwidth}
  		\centering
		\includegraphics [width=\textwidth]{../Plots/GenStudies/XCone_GEN_R20}
		\caption{}
		\end{subfigure}
						
		\caption{Jet mass distributions for different $R_1$. The smaller $R_1$ is, the more jets are reconstructed at the $W$ mass. If $R_1$ increases, the probability of radiation ending up in the final jet grows and jets are more likely to have a mass above the top quark mass.}
		\label{fig:XConeR1}
	\end{figure}	
	
 	\begin{figure}[tb]
 		\begin{subfigure}{.5\textwidth}
  		\centering
 		\includegraphics [width=\textwidth]{../Plots/GenStudies/XCone23_matching}
 		\label{fig:GEN_XCone23}
 		\caption{}
 		\end{subfigure}
 		\begin{subfigure}{.5\textwidth}
  		\centering
 		\includegraphics [width=\textwidth]{../Plots/GenStudies/XCone33_matching}
 		\label{fig:GEN_XCone33}
 		\caption{}
 		\end{subfigure}
 		\caption{Comparison of the jet mass distribution of XCone jets clustered with the '$2+5$' (a) and the '$2+6$' (b) method.}
 		\label{fig:GEN_XCone_comp}
 	\end{figure}
 	
\FloatBarrier %draw figures of previous section before the new one starts 	
\subsection{Comparing Jet Algorithms}
\label{sec:jet_comp}
	In this section, the resulting jet mass distributions of different jet clustering algorithms discussed above are compared.
	
 	\begin{figure}[tb]
 		\begin{subfigure}{.5\textwidth}
  		\centering
 		\includegraphics [width=\textwidth]{../Plots/GenStudies/AK08softdrop_matching}
 		\label{fig:Jet_Comp_ak}
 		\caption{}
 		\end{subfigure}
 		\begin{subfigure}{.5\textwidth}
  		\centering
 		\includegraphics [width=\textwidth]{../Plots/GenStudies/HOTVRrho400_matching}
 		\label{fig:Jet_HOTVR}
 		\caption{}
 		\end{subfigure}
 		\begin{subfigure}{.5\textwidth}
  		\centering
 		\includegraphics [width=\textwidth]{../Plots/GenStudies/XCone33_matching}
 		\label{fig:Jet_XCone}
 		\caption{}
 		\end{subfigure} 		
 		\caption{}
 		\label{fig:Jet_Comp}
 	\end{figure}	
	
	%todo irriduceble background (deltaR (top, b))
	%todo vergleich auch mit XCone Softdrop!!! 		
	
\section{Studies on Reconstruction Level}
\label{sec:selection}
	To obtain a data set consisting of mostly $t\bar{t}$ events in the lepton+jets channel, a selection is applied to simulation and data. The selection can be divided into two steps. Firstly, a baseline selection is used to suppress background processes (see section \ref{sec:PreSel}). Secondly, the final phase space is defined (see section \ref{sec:FinalSel}) to select $t\bar{t}$ events with boosted top quarks. This is crucial for this analysis because the goal is to reconstruct the top quark with one jet. This can only be done if all of its decay products merge into one jet.

\subsection{Baseline Selection}
\label{sec:PreSel}
	In the lepton+jets channel of the $t\bar{t}$ process one expects to find exactly one muon or electron, two small jets from the hadronically decaying $W$ boson, two b-jets and missing transverse energy since the neutrino cannot be detected. This baseline selection is designed to remove non-$t\bar{t}$ events. Since this analysis focuses on the muon channel, the selection are:
	\begin{itemize}
	\item single muon trigger (combination of "HLT\_Mu50\_v*" and "HLT\_TkMu50\_v*") with $p_T > 50\;\text{GeV}$ threshold
	\item exactly one tight muon with $p_T > 55\;\text{GeV}$ and $|\eta| < 2.4$
	\item veto on additional leptons
	\item 2D Cut: $\Delta R(\text{lepton, next AK4 jet}) > 0.4$ or $p_T^{\text{rel}}(\text{lepton, next AK4 jet}) > 40\;\text{GeV}$ \footnote{$p_T^{\text{rel}}(a,b) = \frac{|\vec{p_a} \times \vec{p_b}|}{|\vec{p_b}|}$}
	\item at least two AK4 jets with $p_T > 50\;\text{GeV}$ and $|\eta| < 2.4$
	\item $\cancel{E}_T > 50\;\text{GeV}$
	%\item $S_T^\text{lep} > 100\;\text{GeV}$
	\item at least one tight b-tag

	\end{itemize}
	Because the data set corresponding to the called single muon trigger is used in this analysis, the trigger criteria have to be fulfilled in simulation as well. An additional cut on the muon $p_T$ at $55\;\text{GeV}$ is recommended for this trigger to reach the plateau of trigger efficiency. Furthermore, a scale factor is applied to simulation to account for efficiency differences in data and simulation. Since only one lepton is expected in the lepton+jets channel of $t\bar{t}$, a veto on additional leptons is used to suppresses diboson events. To reject QCD events, a two-dimensional cut is applied. A window in the $\Delta R$-$p_T^{\text{rel}}$-plane is cut out where the majority of QCD that survive the lepton criteria accumulates. A display of the 2D cut is presented in Fig. \ref{fig:2D}. The requirements to find two AK4 jets with at least $50\;\text{GeV}$, missing transverse energy of at least $50\;\text{GeV}$ and a b-tag prefer $t\bar{t}$ events because of the similar signature. For b-tagging a scale factor is applied to match efficiency in data and simulation. After applying the selection the remaining events contain about $80\%$ $t\bar{t}$, the main remaining backgrounds are $W+$jets and Single-Top production. 
	
	%todo ref muon pt cut recommendation
	%todo ref Muon Trigger SF
	%todo Lumi Plot

 	\begin{figure}[tb]
 		\begin{subfigure}{.5\textwidth}
  		\centering
 		\includegraphics [width=\textwidth]{../Plots/TwoD_QCD}
 		\caption{}
 		\end{subfigure}
 		\begin{subfigure}{.5\textwidth}
  		\centering
 		\includegraphics [width=\textwidth]{../Plots/TwoD_TTbar}
 		\caption{}
 		\end{subfigure}
 		\caption{Distribution in the $\Delta R$-$p_T^{\text{rel}}$-plane for QCD (a) and $t\bar{t}$ events (b). The window affected by the 2D cut is surrounded by red lines. While QCD events mostly accumulate in the low left corner, many $t\bar{t}$ events will survive this cut.}
 		\label{fig:2D}
 	\end{figure}

 	\begin{figure}[tb]
 		\begin{subfigure}{.5\textwidth}
  		\centering
 		\includegraphics [width=\textwidth, trim=0 0 3cm 0, clip]{../Plots/PreSel/08_bTag_Muon/number_lin.pdf}
 		\caption{}
 		\end{subfigure}
 		\begin{subfigure}{.5\textwidth}
  		\centering
 		\includegraphics [width=\textwidth, trim=0 0 3cm 0, clip]{../Plots/PreSel/08_bTag_Muon/pt_1_log.pdf}
 		\caption{}
 		\end{subfigure} 		
 		\begin{subfigure}{.5\textwidth}
  		\centering
 		\includegraphics [width=\textwidth, trim=0 0 3cm 0, clip]{../Plots/PreSel/08_bTag_jets/number_lin.pdf}
 		\caption{}
 		\end{subfigure}
 		\begin{subfigure}{.5\textwidth}
  		\centering
 		\includegraphics [width=\textwidth, trim=0 0 3cm 0, clip]{../Plots/PreSel/08_bTag_jets/pt_jet_log.pdf}
 		\caption{}
 		\end{subfigure}
		\begin{subfigure}{.5\textwidth}
  		\centering
 		\includegraphics [width=\textwidth, trim=0 0 3cm 0, clip]{../Plots/PreSel/08_bTag_Event/BTAG_T_lin.pdf}
 		\caption{}
 		\end{subfigure} 	
  		\begin{subfigure}{.5\textwidth}
   		\centering
  		\includegraphics [width=\textwidth, trim=0 0 3cm 0, clip]{../Plots/PreSel/08_bTag_Event/MET_log.pdf}
  		\caption{}
  		\end{subfigure}  	 			
 		\caption{Control distributions after applying the baseline selection. Displayed are number of muons (a), $p_T$ spectrum of muons(b), number of AK4 jets (c), $p_T$ distribution of all AK4 jets (d), number of b-tagged AK4 jets (e) and the spectrum of the missing transverse energy (f). In all histograms simulation exceeds the number of events in data. Furthermore a trend in the $p_T$ and $\cancel{E}_T$ spectra is visible.}
 		\label{fig:PreSel}
 	\end{figure}
 	
 	A difference in total number of events between simulation and data  as well as a $p_T$ dependent trend is visible, probably coming from a mismodelled top quark $p_T$ distribution. The difference in $p_T$ spectra of the top quark has been observed in several publications, for example in a $t\bar{t}$ differential cross section measurement from CMS \cite{ttreweight}. To justify this assumption a reweighting of the top quark $p_T$ spectrum in $t\bar{t}$ simulation is applied. The resulting histograms are presented in Fig. \ref{fig:PreSel_reweight}. In high energy regions simulation still exceeds data. After this procedure, MC and data are well in agreement. Hence, data is well understood at this point and the measurement phase space can be defined on top of the presented baseline selection.
	%todo reweight in Appendix??
 	\begin{figure}[tb]
 		\begin{subfigure}{.5\textwidth}
  		\centering
		\includegraphics [width=\textwidth, trim=0 0 3cm 0, clip]{../Plots/PreSel/ttbar_reweight_Muon/number_lin.pdf}
 		\caption{}
 		\end{subfigure}
 		\begin{subfigure}{.5\textwidth}
  		\centering
 		\includegraphics [width=\textwidth, trim=0 0 3cm 0, clip]{../Plots/PreSel/ttbar_reweight_Muon/pt_1_log.pdf}
 		\caption{}
 		\end{subfigure} 		
 		\begin{subfigure}{.5\textwidth}
  		\centering
 		\includegraphics [width=\textwidth, trim=0 0 3cm 0, clip]{../Plots/PreSel/ttbar_reweight_Jets/number_lin.pdf}
 		\caption{}
 		\end{subfigure}
 		\begin{subfigure}{.5\textwidth}
  		\centering
 		\includegraphics [width=\textwidth, trim=0 0 3cm 0, clip]{../Plots/PreSel/ttbar_reweight_Jets/pt_jet_log.pdf}
 		\caption{}
 		\end{subfigure}
		\begin{subfigure}{.5\textwidth}
  		\centering
 		\includegraphics [width=\textwidth, trim=0 0 3cm 0, clip]{../Plots/PreSel/ttbar_reweight_Event/BTAG_T_lin.pdf}
 		\caption{}
 		\end{subfigure} 	
  		\begin{subfigure}{.5\textwidth}
   		\centering
  		\includegraphics [width=\textwidth, trim=0 0 3cm 0, clip]{../Plots/PreSel/ttbar_reweight_Event/MET_log.pdf}
  		\caption{}
  		\end{subfigure}  	 			
 		\caption{Control distributions after applying a reweighting of the top quark $p_T$ in $t\bar{t}$ simulation. Displayed are number of muons (a), $p_T$ spectrum of muons (b), number of AK4 jets (c), $p_T$ distribution of all AK4 jets (d), number of b-tagged AK4 jets (e) and the spectrum of the missing transverse energy (f). }
 		\label{fig:PreSel_reweight}
 	\end{figure}	

\FloatBarrier %draw figures of previous section before the new one starts
\subsection{Measurement Phase Space}
\label{sec:FinalSel}
	The measurement phase space on reconstruction level is defined analogously to the particle level selection. Boosted topologies are selected by requiring the leading jet to surpass a cut on $p_T > 400\;\text{GeV}$. In addition the mass of the leading jet is expected to be higher then the second jet mass if all top quark decay products are reconstructed correctly. Thus, on top of the baseline selection presented above, following criteria are checked:
	\begin{itemize}
	\item only sum subjets with $p_T > 30\;\text{GeV}$ 
	\item $p_T^{\text{1st jet}} > 400\;\text{GeV}$ 
	\item $M^{\text{1st jet}} > M^{\text{2nd jet + lepton}}$
	\end{itemize}
	In this phase space, the selection criteria refer to XCone jets clustered with the '$2+6$' method.	In addition to the definition on particle level, only subjets with a $p_T$ larger then $30\;\text{GeV}$ are considered to form the final jet. This requirement is set so suppress subjets which do not contain any decay product of the top quark. Figure \ref{fig:MJet_raw1} shows the jet mass distribution after the called jet requirements. Similar as on particle level, XCone returns a peak at the expected bin of the top quark mass. Thus, the reconstruction with XCone jets works very well. As seen after the baseline selection, simulation exceeds the number of events in data. In Fig. \ref{fig:MJet_raw2}, $t\bar{t}$ simulation is scaled with a constant factor to match the total events in data and simulation. A good agreement between data and MC is visible. Of course, further procedure like unfolding will use the unscaled version as input. In addition another variant of jet clustering with XCone is presented in Fig. \ref{fig:MJet_raw3}. Here, the large XCone jet with radius $R=1.2$ is put into the soft drop algorithm. Albeit the grooming of soft drop, the '2+6' method returns a much sharper mass peak. While the turn on of the soft drop mass is comparable with '2+6', there are many event with a mass reconstructed too high.
 	\begin{figure}[tb]
 		\begin{subfigure}{.5\textwidth}
  		\centering
 		\includegraphics [width=\textwidth, trim=0 0 3cm 0, clip]{../Plots/PostSel/XCone_raw/M_jet1__lin.pdf}
 		\caption{}
 		\label{fig:MJet_raw1}
 		\end{subfigure}
 		\begin{subfigure}{.5\textwidth}
 		\centering
		\begin{tikzpicture}
		 \node[anchor=south west,inner sep=0] (image) at (0,0)
		 {\includegraphics[width=\textwidth, trim=0 0 3cm 0, clip]{../Plots/PostSel/XCone_raw_SF/M_jet1__lin.pdf}};
		 \node[align=left,font=\tiny] at (2.2, 4.2) {$t\bar{t}$ scaled};
		\end{tikzpicture} 
 		\caption{}
 		\label{fig:MJet_raw2}
 		\end{subfigure}
 		\begin{subfigure}{.5\textwidth}
 		\centering
		\begin{tikzpicture}
		 \node[anchor=south west,inner sep=0] (image) at (0,0)
		 {\includegraphics[width=\textwidth, trim=0 0 3cm 0, clip]{../Plots/PostSel/XCone_raw_SF/SoftdropMass_had_lin.pdf}};
		 \node[align=left,font=\tiny] at (2.2, 4.2) {$t\bar{t}$ scaled};
		\end{tikzpicture} 
 		\caption{}
 		\label{fig:MJet_raw3}
 		\end{subfigure} 		
 		\caption{Jet mass distribution of XCone jets after applying the measurement phase space requirements. Additional to the raw output (a) a distribution where $t\bar{t}$ simulation is scaled to data is presented (b). Histogram (c) shows the soft drop mass of the large XCone jet ($R=1.2$) in comparison. It shows that a grooming via subjet finding is much more effective than soft drop.}
 		\label{fig:MJet_raw}
 	\end{figure}
	
\FloatBarrier %draw figures of previous section before the new one starts	
\subsection{Jet Energy Corrections for XCone Jets} 
	The normal procedure in CMS analyses is to apply jet energy corrections (see section \ref{sec:jec}) to every jet collection used. Those jet energy corrections (JEC) have been derived by dedicated CMS groups and are different depending on the jet algorithm used to cluster jets. Since the XCone algorithm is not a standard jet finding procedure in CMS, there are no valid corrections available. Because the final jet measured in this a analysis is a combination of subjets, only the subjets are corrected. A XCone jet with applied jet energy corrections refers then to a jet put together from corrected subjets. The first attempt to correct XCone jets is to use the AK4 jet corrections since the jet shape should be very similar to XCone jets with $R=0.4$ as they were used in this analysis. Figure \ref{fig:MJet_jec} shows a comparison between jets with and without JEC applied. 
 	\begin{figure}[tb]
 		\begin{subfigure}{.5\textwidth}
 		\centering
		\begin{tikzpicture}
		 \node[anchor=south west,inner sep=0] (image) at (0,0)
		 {\includegraphics[width=\textwidth, trim=0 0 3cm 0, clip]{../Plots/PostSel/XCone_raw_SF/M_jet1__lin.pdf}};
		 \node[align=left,font=\tiny] at (2.2, 4.2) {$t\bar{t}$ scaled};
		\end{tikzpicture} 
 		\caption{}
 		\label{fig:MJet_jec1}
 		\end{subfigure}
 		\begin{subfigure}{.5\textwidth}
 		\centering
		\begin{tikzpicture}
		 \node[anchor=south west,inner sep=0] (image) at (0,0)
		 {\includegraphics[width=\textwidth, trim=0 0 3cm 0, clip]{../Plots/PostSel/XCone_jec_SF/M_jet1__lin.pdf}};
		 \node[align=left,font=\tiny] at (2.2, 4.2) {$t\bar{t}$ scaled};
		\end{tikzpicture} 
 		\caption{}
 		\label{fig:MJet_jec2}
 		\end{subfigure}
 		\caption{Jet mass distribution of XCone jets before (a) and after applying AK4 jet energy corrections (b). A shift in jet masses to higher values is visible.}
 		\label{fig:MJet_jec}
 	\end{figure}	
 	An obvious shift in jet mass is visible, resulting in a peak position above the top quark mass. Since the data-simulation agreement is well in both cases, different measures have to be defined to validate the jet energy correction. Now, an attempt is made to compare the subjets with a well known parameter in the theory. Therefore all three jet mass combinations $M_{ij}$ of two of the three subjets on the hadronic side are calculated. It is expected that the combination with the lowest jet mass should be sensitive to the $W$ mass which is very precisely measured to $80.4\;\text{GeV}$ \cite{Wmass}. As shown in Fig. \ref{fig:Wmass} the $W$ boson mass is reconstructed too high at values above $85\;\text{GeV}$. Therefore, applying jet energy correction from AK4 jets to XCone subjets is not valid.
  	\begin{figure}[tb]
  		\begin{subfigure}{.5\textwidth}
  		\centering
 		\begin{tikzpicture}
 		 \node[anchor=south west,inner sep=0] (image) at (0,0)
 		 {\includegraphics[width=\textwidth, trim=0 0 3cm 0, clip]{../Plots/PostSel/XCone_raw_subjets_SF/min_mass_Wjet_zoom_lin.pdf}};
 		 \node[align=left,font=\tiny] at (2.2, 4.2) {$t\bar{t}$ scaled};
 		\end{tikzpicture} 
  		\caption{}
  		\label{fig:Wmass1}
  		\end{subfigure}
  		\begin{subfigure}{.5\textwidth}
  		\centering
 		\begin{tikzpicture}
 		 \node[anchor=south west,inner sep=0] (image) at (0,0)
 		 {\includegraphics[width=\textwidth, trim=0 0 3cm 0, clip]{../Plots/PostSel/XCone_jec_subjets_SF/min_mass_Wjet_zoom_lin.pdf}};
 		 \node[align=left,font=\tiny] at (2.2, 4.2) {$t\bar{t}$ scaled};
 		\end{tikzpicture} 
  		\caption{}
  		\label{fig:Wmass2}
  		\end{subfigure}
  		\caption{Lowest jet mass combination $\min(M_{ij})$ without (a) and with (b) JEC applied. The position of the peak is expected to match the $W$ boson mass of $80.4\;\text{GeV}$. The shift after applying JEC shifts $\min(M_{ij})$ to higher values.} 
  		\label{fig:Wmass}
  	\end{figure}	
 	%todo Resolution (nur nach correction zeigen?)
	To still be able to correct XCone jets for response non linearities and pile-up effects, a correction factor on top of AK4 corrections is derived for XCone jets. For this, only events from $t\bar{t}$ simulation are used. Furthermore, only the subjets from the jet belonging to the hadronically decaying top quark are considered. Now a matching to generator jets is executed and the fraction $R=\frac{p_T^{\text{rec}}}{p_T^{\text{gen}}}$ calculated. This is done in different $p_T$ and $\eta$ regions. The mean $R$ is then filled in a two dimensional histogram representing the $p_T$-$\eta$-plane (see Fig. \ref{fig:Correction}). The bin boundaries are chosen to obtain enough statistics in each bin to suppress uncertainties (RMS and uncertainties in appendix Fig. \ref{fig:A_err} and \ref{fig:A_rms}). 
		\begin{figure}[tb]
			\centering
			\includegraphics [width=.7\textwidth]{../Plots/Correction/Mean_numbers}
			\caption{Mean values of $R=\frac{p_T^{\text{rec}}}{p_T^{\text{gen}}}$ in the $p_T$-$\eta$ plane.}
			\label{fig:Correction}
		\end{figure}	
	The correction factor applied to every XCone jet is now $f = \frac{1}{R}$. To get a smooth transition between the different regions, in every $\eta$ bin a polynomial function is fitted to get a factor $f(p_T)$. An example of the fit is shown in Fig. \ref{fig:Correction_fit} (all fit functions can be found in the appendix in Fig. \ref{fig:A_fits}). Now, every subjet from the first jet is corrected with a $p_T$ dependent function corresponding to its $\eta$ value.
	\begin{figure}[tb]
		\centering
		\includegraphics [width=.5\textwidth]{../Plots/Correction/Fits_example}
		\caption{Example of fit function. The correction factors are derived from Fig. \ref{fig:Correction} and then fitted with a polynomial function of degree $2$.}
		\label{fig:Correction_fit}
	\end{figure}
	Now every subjet is corrected with a different $p_T$ dependent function according to its direction in $\eta$. It is to mention that because of the loose ends of the fit, the correction factor is set constant for $p_T > 425\;\text{GeV}$. To verify this correction the minimum jet mass from two subjets is again compared with the $W$ boson mass in Fig. \ref{fig:Wmass_cor}. It is shown that the peak position is in agreement with the measured $M_W = 80.4\;\text{GeV}$.
  	\begin{figure}[tb]
  		\centering
 		\begin{tikzpicture}
 		 \node[anchor=south west,inner sep=0] (image) at (0,0)
 		 {\includegraphics[width=.8\textwidth, trim=0 0 3cm 0, clip]{../Plots/PostSel/XCone_cor_subjets_SF/min_mass_Wjet_zoom_lin.pdf}};
 		 \node[align=left,font=\small] at (3.5, 7.0) {$t\bar{t}$ scaled};
 		\end{tikzpicture} 
   		\caption{Lowest jet mass combination $\min(M_{ij})$ after AK4 jet energy correction and the correction for XCone. The peak position is in agreement with the value of the $W$ boson mass of $80.4\;\text{GeV}$.} 
  		\label{fig:Wmass_cor}
  	\end{figure}	  
  	Furthermore, a matching via $\Delta R$ between jets on particle and reconstruction level is performed to calculate the relative $p_T$ deviation. A graph showing the mean and width (here calculated as RMS \footnote{root mean square}) of $\frac{p_T^\text{rec} - p_T^\text{gen}}{p_T^\text{gen}}$ in bins of $p_T^\text{rec}$ are presented in Fig. \ref{fig:Reso}. While the width of the distribution (Fig. \ref{fig:Reso2}) is not much influenced by the different jet energy correction stages, the mean spectrum shows a large improvement after additional correction. The raw XCone jets and the ones corrected with AK4 corrections show a dependence on $p_T$ where the corrected jets are flatly distributed. Additionally, the corrected jets show a constant mean around $0$ which indicates a well performing correction.
  	
  	\begin{figure}[tb]
  		\begin{subfigure}{.5\textwidth}
   		\centering
  		\includegraphics [width=\textwidth]{../Plots/Resolution_Subjets/pt_mean_rec_after}
  		\caption{}
  		\label{fig:Reso1}
  		\end{subfigure}
  		\begin{subfigure}{.5\textwidth}
   		\centering
  		\includegraphics [width=\textwidth]{../Plots/Resolution_Subjets/pt_rms_rec_after}
  		\caption{}
  		\label{fig:Reso2}
  		\end{subfigure}
  		\caption{Mean (a) and width (b) of the relative deviation in jet $p_T$ between particle and reconstruction level.}
  		\label{fig:Reso}
  	\end{figure}
  	
  	
\FloatBarrier %draw figures of previous section before the new one starts	
\subsection{Final Jet Mass Distribution}
	After the events ran through a baseline selection, a measurement phase space requirement and a new derived jet energy correction, the final jet mass distribution is presented in Fig. \ref{fig:MJet_final}. After scaling $t\bar{t}$ simulation with a constant factor, data and Monte Carlo prediction agree very well. Furthermore, the distribution peaks at the top quark mass and shows a good resolution. In addition the large amount of statistics is a promising point for the unfolding procedure.	 
  	\begin{figure}[tb]
  		\centering
 		\begin{tikzpicture}
 		 \node[anchor=south west,inner sep=0] (image) at (0,0)
 		 {\includegraphics[width=.8\textwidth, trim=0 0 3cm 0, clip]{../Plots/PostSel/XCone_cor_SF/M_jet1__lin.pdf}};
 		 \node[align=left,font=\small] at (3.5, 7.0) {$t\bar{t}$ scaled};
 		\end{tikzpicture} 
  		\caption{Jet mass distribution of the corrected jets. This distribution will be used as input for further analysis steps.} 
  		\label{fig:MJet_final}
  	\end{figure}	

  	
\section{Unfolding}
\label{sec:unfolding}
	Most analyses at LHC measure distributions of appropriate variables and then compare the obtained results in data with event simulations. In this method the MC samples also include detector effects. What one measures in this case is the real distribution on particle level folded with an unknown detector function. Studying the difference in MC between particle level and reconstruction level, it is possible to calculate the probabilities that a measured value in a bin $y_i$ is originating from bin $x_i$ on particle level. A visualisation of this problem is drawn in Fig. \ref{fig:Unfolding}.	
	\begin{figure}[tb]
		\centering
		\includegraphics [width=.6\textwidth]{../Images/Unfolding.png}
		\caption{Schematic view of an unfolding procedure. The goal is to unfold a measured distribution $\mathbf{y}$ to obtain a true distribution $\mathbf{x}$ without detector effects. Taken from \cite{tunfold}.}
		\label{fig:Unfolding}
	\end{figure}
	The matrix $\mathbf{A}$ describing migrations from bins in $\mathbf{x}$ to bins in $\mathbf{y}$ can be calculated using simulations where both distributions $\mathbf{x}$ and $\mathbf{y}$ are known. Then, the migration matrix can be used obtain an estimate for data on particle level. Following Eq. \ref{eq:unfold} denotes the basic problem one has to solve:
	\begin{equation}
	\tilde{y}_i = \sum_{j=1}^{m} A_{ij}\tilde{x}_j, 1 \leq i \leq n
	\label{eq:unfold}
	\end{equation}
	where $m$ and $n$ are the number of bins of the true and measured distributions, respectively. The tilde marks the statistical mean of $\mathbf{x}$ and $\mathbf{y}$. Here, one is interested in a distribution $x_j$ and not $\tilde{x}_j$ what makes this problem not trivial. If one only replaces $\tilde{y}_i \rightarrow y_i$ and $\tilde{x}_j \rightarrow x_j$ and solve for $x_j$ by inverting the matrix $\mathbf{A}$, statistical fluctuations of $\mathbf{y}$ would be amplified. Thus, fluctuations have to be damped with a regularisation.
	
\subsection{Regularised Unfolding with TUnfold}
	The TUnfold software package \cite{tunfold} provides a framework for regularised unfolding procedures in high energy physics. Equation \ref{eq:unfold_lagrange} shows the Lagrangian implemented in TUnfold that is minimised.
	\begin{eqnarray}
	\label{eq:unfold_lagrange}
	\mathcal{L}(x,\lambda) &=& \mathcal{L}_1 + \mathcal{L}_2 + \mathcal{L}_3 
	\\ \nonumber \text{with}
	\\ 
	\label{eq:unfold_lagrange1}
	\mathcal{L}_1 &=& (\mathbf{y} - \mathbf{Ax})^\intercal \mathbf{V_{yy}}^{-1} (\mathbf{y} - \mathbf{Ax}) 
	\\
	\label{eq:unfold_lagrange2}
	\mathcal{L}_2 &=& \tau^2 (\mathbf{x} - f_b \mathbf{x}_0)^\intercal (\mathbf{L}^\intercal \mathbf{L}) (\mathbf{x} - f_b \mathbf{x}_0) 
	\\
	\label{eq:unfold_lagrange3}
	\mathcal{L}_3 &=& \lambda (Y-\mathbf{e}^\intercal \mathbf{x}) \ \text{with} \ Y=\sum_{i} y_i \ \text{and} \ e_j = \sum_{i}A_{ij}
	\end{eqnarray}
	The first term $\mathcal{L}_1$ contains a standard least square minimisation where $\mathbf{V_{yy}}$ is the covariance matrix describing uncertainties. Secondly a regularisation with strength $\tau^2$ is used. A bias vector can be introduced using a factor $f_b$ and a vector $\mathbf{x}_0$ to suppress deviations of $\mathbf{x}$ from $f_b\mathbf{x}_0$. Additionally, three choices for the matrix $\mathbf{L}$ can be made to either regularise the absolute value, first or second derivative of $\mathbf{x}$. The final term $\mathcal{L}_3$ expresses an optional area constraint, checking differences in event counts between input and output.
	
	%todo eigenes unfolding
 
 
\section{Results}
\label{sec:results}
%todo nochmal phasenraum hinschreiben


\chapter{Summary and Outlook}
\appendix
\chapter{Jet Energy Corrections for XCone}
	After applying AK4 jet energy correction to the XCone subjets, an additional correction is derived to account for slight differences in XCone and Anti-$k_T$ jets. While figure \ref{fig:A_mean} shows the ratios between jet $p_T$ on reconstruction and particle level in the $\eta$-$p_T$-plane, Fig. \ref{fig:A_rms} shows the belonging RMS width. Since the width shows small values, each bin contains enough statistics to derive a correction factor. In every $\eta$ bin a $p_T$ dependent function is fitted to obtain a smooth correction. All separate fits are displayed in Fig. \ref{fig:A_fits}. 

	\begin{figure}[h]
		\centering
		\includegraphics [width=.9\textwidth]{../Plots/Correction/Mean_numbers}
		\caption{Mean of the ratio between jet $p_T$ on reconstruction and particle level.}
		\label{fig:A_mean}
	\end{figure}
	
	\begin{figure}[h]
		\centering
		\includegraphics [width=.9\textwidth]{../Plots/Correction/RMS_numbers}
		\caption{RMS values of $R=\frac{p_T^{\text{rec}}}{p_T^{\text{gen}}}$ in the $p_T$-$\eta$ plane.}
		\label{fig:A_rms}
	\end{figure}	
	
	\begin{figure}[h]
		\centering
		\includegraphics [width=.98\textwidth]{../Plots/Correction/Fits}
		\caption{Fit functions for XCone jet energy corrections in every $\eta$ region.}
		\label{fig:A_fits}
	\end{figure}


\pagestyle{plain}
% ===========================================================
% ======================== Literaturverzeichnis =============
% ===========================================================
% Eintr�ge erscheinen nur, wenn sie auch zitiert werden!!!
\printbibliography
%todo in bib style formatieren. z.B. darf "CMS Collaboration" nicht C. Collaboration werden
%todo D0 richtig formatiert? 
%----------------------------------------------------------------
%------------------Unterschrift----------------------------------
%----------------------------------------------------------------
\selectlanguage{ngerman}
\cleardoublepage
\noindent Hiermit best�tige ich, dass die vorliegende Arbeit von mir selbstst"andig verfasst wurde und ich keine anderen als die angegebenen Hilfsmittel -- insbesondere keine im Quellenverzeichnis nicht benannten Internet-Quellen -- benutzt habe und die Arbeit von mir vorher nicht einem anderen Pr�fungsverfahren eingereicht wurde. Die eingereichte schriftliche Fassung entspricht der auf dem elektronischen Speichermedium. Ich bin damit einverstanden, dass die Masterarbeit ver�ffentlicht wird.\\
\newline
\newline
\newline
\begin{tabular*}{\textwidth}{lll}
	\hline
	Ort, Datum & \hspace{10.4cm} & Unterschrift  \\
\end{tabular*}

\cleardoublepage
%----------------------------------------------------------------
%---------------------Danksagung---------------------------------
%----------------------------------------------------------------
\section*{Danksagung}
Zun"achst m"ochte ich Prof. Dr. Johannes Haller daf"ur danken, dass ich die M"oglichkeit hatte, meine Masterarbeit in seiner Forschungsgruppe "uber dieses interessante Thema zu schreiben. Desweiteren bedanke ich mich f"ur die vielen R"ucksprachen und Denkanst"o"se, die meine Arbeit stets bereichert haben.\\
Prof. Dr. Peter Schleper m"ochte ich daf"ur danken das Zweitgutachten meiner Arbeit zu "ubernehmen.\\
Ein besonderer Dank gilt Dr. Roman Kogler, der viel Zeit in die Betreuung meiner Analyse, zahlreiche Diskussionen "uber Jets und die Korrektur meiner Arbeit gesteckt hat. \\
Au"serdem danke ich allen Mitgliedern der Arbeitsgruppe f"ur die Menge an gekl"arte Fragen, eine nette Atmosph"are und erhellende Gespr"ache in und abseits der Teilchenphysik. Zuletzt m"ochte ich meiner Familie und vor allem Inga danken, die mich immer unterst"utzt haben. 

\end{document}  
%----------------------------------------------------------------
%----------------------------------------------------------------
%----------------------------------------------------------------
