\documentclass[12pt, a4paper, twoside, openright]{report}

% ===========================================================
% ======================== packages =========================
% ===========================================================
\usepackage[ngerman, english]{babel}
\usepackage[ansinew]{inputenc}
\usepackage[T1]{fontenc}
\usepackage{mathptmx}
\usepackage{amsmath}
\usepackage{amssymb}
\usepackage{amstext}
\usepackage{amsfonts}
\usepackage{mathrsfs}
\usepackage{enumitem}
\usepackage{graphicx}
\usepackage[left=2.5cm,right=2.5cm,top=2.5cm,bottom=2cm]{geometry}
\usepackage[onehalfspacing]{setspace}
\usepackage[section]{placeins} %Bilder rutschen nicht aus section
\usepackage[markup=nocolor,deletedmarkup=xout]{changes}
\usepackage{cancel}			%MET symbol
\usepackage{chngcntr}
\usepackage{lscape}
\usepackage[normal]{caption} %style f�r Bildunterschriften
\usepackage{multicol} 
\usepackage{epstopdf}		%.eps get converted to pdf
\usepackage{textcomp}
\usepackage{url}			%for hyperlinks
\usepackage{blindtext} 		%use blindtext
\usepackage{subcaption} 	%for (a) in subfigures
\usepackage{tikz} 			%new graphic options
\usepackage{verbatim}		
\usepackage{microtype} 		%bessere Zeilenumbrueche, z.B. ISBN in Literatur
\usepackage{typearea}
\usepackage{tikz-feynman} 	%Feynman Diagrams
\tikzfeynmanset{compat=1.1.0}
\usepackage{fancyhdr}		%new head and foot lines
\usepackage[babel, german=quotes]{csquotes}
\usepackage{hyperref}
%\usepackage[style=lucas_unsrt, maxbibnames=100, sorting=none, backend=bibtex]{biblatex}
%\usepackage{biblatex}
\usepackage{bibgerm}
\bibliographystyle{lucas_unsrt}
% ===========================================================
% ======================== new commands =====================
% ===========================================================
\newcommand*\chem[1]{\ensuremath{\mathrm{#1}}} %chem formulas

% ===========================================================
% ======================== configuration ====================
% ===========================================================
% customised header and footer
\pagestyle{fancy}
\fancyhf{}
\fancyfoot[CE,CO]{\thepage}
\fancyhead[CO]{\rightmark}
\fancyhead[CE]{\leftmark}
% redefine cleardoublepage: do not print fancy headline if page is empty before new chapter
\let\mtcleardoublepage\cleardoublepage
\renewcommand{\cleardoublepage}{\clearpage{\pagestyle{empty}\mtcleardoublepage}}
% =======================
% title style
\usepackage{color}
\definecolor{gray75}{gray}{0.75}
\usepackage{titlesec}
\titleformat{\chapter}[block]
  {\normalfont\Huge\bfseries}
  {\thechapter \hspace{20pt} \textcolor{gray75}{|} \hspace{20pt}}
  {0pt}
  {\Huge}
\titlespacing*{\chapter}{0pt}{0cm}{1cm}
\titleformat{\section}[block]
  {\normalfont\Large\bfseries}
  {\thesection}
  {1em}
  {}
\titleformat{\subsection}[block]
  {\normalfont\large\bfseries}
  {\thesubsection}
  {1em}
  {}
\titleformat{\subsubsection}[block]
  {\normalfont\normalsize\bfseries}
  {}
  {1em}
  {}
% =======================           
% let figure, tables, etc stay in chapter
\counterwithin{figure}{chapter}
\counterwithin{table}{chapter}
\counterwithin{equation}{chapter}
% ======================= 
% bibliography stuff
%\addbibresource{MasterBib.bib}
%\ExecuteBibliographyOptions{% 
%firstinits=true, 
%maxbibnames=3, % Alle Autoren (kein et al.) 
%minalphanames=3, 
%maxalphanames=3, 
%}% 


\urlstyle{same}
% =======================
\begin{document}
% ===========================================================
% ======================== Deckblatt ========================
% ===========================================================
\selectlanguage{ngerman}
\pagenumbering{roman}
\thispagestyle{empty}
\begin{titlepage}

\newcommand{\HRule}{\rule{\linewidth}{0.5mm}} % Defines a new command for the horizontal lines, change thickness here

\center % Center everything on the page
 
%----------------------------------------------------------------------------------------
%	HEADING SECTIONS
%----------------------------------------------------------------------------------------

\textsc{\LARGE Universit"at Hamburg}\\[1.5cm] % Name of your university/college
\textsc{\Large Masterarbeit}\\[0.5cm] % Major heading such as course name
\textsc{\large im Studiengang Physik}\\[0.5cm] % Minor heading such as course title

%----------------------------------------------------------------------------------------
%	TITLE SECTION
%----------------------------------------------------------------------------------------

\HRule \\[0.4cm]
{\bfseries \huge Measurement of the Jet Mass Distribution in Boosted Top Quark Decays \\[0.4cm] } 
%{\bfseries \LARGE Measurement of the Jet Mass Distribution in Boosted Top Quark Decays \\[0.1cm] \rule[0mm]{50mm}{0.4mm} \\[0.2cm] Messung der Jetmassenverteilung in kollimierten Top-Quark Zerf"allen \\[0.4cm] } 
\HRule \\[4cm]
 
%----------------------------------------------------------------------------------------
%	AUTHOR SECTION
%----------------------------------------------------------------------------------------
\begin{minipage}[t l]{0.5\textwidth}
\begin{flushleft} \large
\emph{Autor:}\\
\textsc{Dennis Schwarz} % Your name
\\[1cm]
\emph{Gutachter:} \\
\textsc{Prof. Dr. Johannes Haller} \\ 
\textsc{Prof. Dr. Peter Schleper}
\end{flushleft}
\end{minipage}
~
\begin{minipage}[t r]{0.4\textwidth}
\begin{flushright}
\includegraphics[width=.7\textwidth]{../Logos/UHHlogo.png}
\end{flushright}
\end{minipage}
\\[4cm]
%----------------------------------------------------------------------------------------
%	DATE SECTION
%----------------------------------------------------------------------------------------
{\large 2017} % Date, change the \today to a set date if you want to be precise

\vfill % Fill the rest of the page with whitespace

\end{titlepage}
% clearpage without page number:
{\clearpage{\pagestyle{empty}\mtcleardoublepage}} 
%Abstract
\section*{Zusammenfassung}
\thispagestyle{plain}

\selectlanguage{english}
\section*{Abstract}
The presented analysis aims at a measurement of the jet mass distribution in collimated full hadronic top quark decays, where the top quark is reconstructed within one single jet. The jet mass is sensitive to the top quark mass and therefore an essential jet substructure variable for top quark identification. Understanding jet substructure is essential in future analyses with LHC data, because as a consequence of the high centre-of-mass energy, heavy objects are often Lorentz boosted and have to be identified with a single jet. Studies with various jet clustering algorithms are presented to find the most suitable method to reconstruct the jet mass in top quark decays. Consequently, the exclusive jet clustering algorithm XCone shows the best performance and is used to obtain a jet mass distribution that is unfolded to particle level. The unfolding is applied to be able to compare with analytical calculations. Furthermore, the top quark mass can be extracted without relying on Monte Carlo simulations. Proving the method of this analysis, an unfolded jet mass distribution is presented as result. The measurement is performed in the $t\bar{t}$ enriched region, where the one of the top quarks is required to decay in the lepton+jet channel. This offers a good suppression of non-$t\bar{t}$ backgrounds while retaining a high selection efficiency. For this analysis, data recorded by the CMS detector in 2016 at a centre-of-mass energy of $13\;\text{TeV}$, corresponding to an integrated luminosity of $37.76\;\text{fb}^{-1}$, is used.


\clearpage
\tableofcontents
\addtocontents{toc}{~\hfill\textbf{Page}\par}
\clearpage
{\pagestyle{empty} \cleardoublepage}



\pagenumbering{arabic}
\selectlanguage{english}
% ===========================================================
% ======================== Inhalt ===========================
% ===========================================================
%todo Absaetze, indent pruefen
\chapter{Introduction}
%todo USING NATURAL UNITS!!!
%todo schreiben: wenn nicht anders gekennzeichnet, alle pT und E angaben in GeV
%todo CONTENT OF EACH CHAPTER
\chapter{Theory}
\section{Standard model of particle physics}
	The standard models of particles is a quantum field theory describing elementary particles and their interaction. All particles contained in this theory have been discovered and experiments have confirmed predictions of the standard model at very high precision. In Fig. \ref{SM} all elementary particles and their basic properties are displayed. The particles can be ordered in groups by their properties. Firstly one distinguishes particles depending on their spin. Spin-$\frac{1}{2}$ particles are called fermions, Spin-$1$ and Spin-$0$ particles represent the bosons. Fermions are the building block of matter while bosons carry the fundamental forces included in the standard model. Quarks and leptons are sub groups of the fermions are divided by their possible interaction with forces. While the quarks are affected by the strong force, carried by the gluon, the leptons are not.
	% QED
	% QCD
	% WEAK
	% HIGGS
	Three fundamental forces are introduced in this theory. The strong force is carried by the gluon and affects quarks and the gluon itself. The charge of the strong interaction is called colour.
	\begin{figure}[tb]
		\centering
		\includegraphics [width=\textwidth]{../Plots/Standard_Model.png}
		\caption{Particle content of the standard model \cite{SM}}
		\label{SM}
	\end{figure}
		 
\subsection{Top Quark}
	The top quark is an up type quark from the third generation and carries electromagnetic charge of $Q=+\frac{2}{3}e$. With a mass of $173\;\text{GeV}$ tt is the heaviest particle in the standard model, therefore offers a large phase space for decays and thus has a life time of approximately $0.5 \times 10^{-24}\;\text{s}$. Because of the short life time the top quark does not form bound hadronic states and thus measurement of the bare quark are possible. This provides a special access to parameters of the standard model. Especially the top quark mass is an essential parameter to check the standard model for consistency. The Higgs mechanism relates the masses of the top quark with the masses of the W and Higgs boson. Measuring these masses with high precision gives the possibility to check the standard model for consistency.
	\\
	Additionally the top quark is important for searches of new physics since is is often part of the final state and/or a relevant background. 
	\\
	In hadron colliders, the production of a $t\bar{t}$ pair happens via $g\bar{q}$ annihilation or gluon fusion. At the centre-of-mass energy of $13\;\text{TeV}$ from the LHC, gluon fusion is by far the dominant process. Top quarks can also be produced in single production, but has, being a electroweak process, a much smaller cross section. Hence, this analysis will focus on pair produced top quarks and will treat single top production as a background process.
	\begin{figure}
		\centering
		\begin{subfigure}{.4\textwidth}
		\begin{tikzpicture}
		\begin{feynman}	
		\vertex(a);
		\vertex[right=of a] (b);
		\vertex[below left=of a] (i1){\(g\)};
		\vertex[above left=of a] (i2){\(g\)};		
		\vertex[below right=of b] (f1){\(\overline t\)};				
		\vertex[above right=of b] (f2){\(t\)};		
		\diagram* {
		(a)-- [gluon,edge label'=\(g\)] (b),
		(i1)-- [gluon] (a) -- [gluon] (i2),
		(f1)-- [fermion] (b) -- [fermion] (f2),
		};
		\end{feynman}
		\end{tikzpicture}
		\caption{}
		\end{subfigure}
		\begin{subfigure}{.4\textwidth}
		\begin{tikzpicture}
		\begin{feynman}	
		\vertex(a);
		\vertex[right=of a] (b);
		\vertex[below left=of a] (i1){\(q\)};
		\vertex[above left=of a] (i2){\(\overline q\)};		
		\vertex[below right=of b] (f1){\(\overline t\)};				
		\vertex[above right=of b] (f2){\(t\)};		
		\diagram* {
		(a)-- [gluon,edge label'=\(g\)] (b),
		(i1)-- [fermion] (a) -- [fermion] (i2),
		(f1)-- [fermion] (b) -- [fermion] (f2),
		};
		\end{feynman}
		\end{tikzpicture}
		\caption{}
		\end{subfigure}
		\begin{subfigure}{.4\textwidth}
		\begin{tikzpicture}
		\begin{feynman}	
		\vertex(a);
		\vertex[below=of a] (b);
		\vertex[above left=of a] (i1){\(g\)};
		\vertex[below left=of b] (i2){\(g\)};		
		\vertex[above right=of a] (f1){\(t\)};				
		\vertex[below right=of b] (f2){\(\overline{t}\)};		
		\diagram* {
		(b)-- [fermion] (a),
		(i1)-- [gluon] (a) -- [fermion] (f1),
		(i2)-- [gluon] (b) -- [anti fermion] (f2),
		};
		\end{feynman}
		\end{tikzpicture}
		\caption{}
		\end{subfigure}
		\begin{subfigure}{.4\textwidth}
		\begin{tikzpicture}
		\begin{feynman}	
		\vertex(a);
		\vertex[below=of a] (b);
		\vertex[above left=of a] (i1){\(g\)};
		\vertex[below left=of b] (i2){\(g\)};		
		\vertex[above right=of a] (f1){\(t\)};				
		\vertex[below right=of b] (f2){\(\overline{t}\)};		
		\diagram* {
		(b)-- [gluon] (a),
		(i1)-- [gluon] (a) -- [fermion] (f1),
		(i2)-- [gluon] (b) -- [anti fermion] (f2),
		};
		\end{feynman}
		\end{tikzpicture}
		\caption{}
		\end{subfigure}
		\caption{Feynman diagrams \cite{feynman} showing the production of top quark pairs. Displayed are the gluon-gluon fusion (a), the quark anti-quark annihilation (b) and the t-channel (c+d).}
		\label{fig:production}
	\end{figure}	
	
	The top quark decays via the weak interaction with a probability of almost $100\%$ into a bottom quark and a $W$ boson. While the bottom quark is seen as a jet in the detector, the $W$ boson decays further into a quark anti-quark pair (see fig. \ref{fig:decaya}) or into a lepton and a neutrino. These two cases are called hadronically respectively hadronically top quark decay (see fig. \ref{fig:decayb}). Looking at the $t\bar{t}$ production, these two possible decays for each top quark corresponds to three channels for the $t\bar{t}$ process. 
	\begin{itemize}
	\item both top quarks decay into quarks (full hadronic)
	\item one top quark decays hadronically, the other one leptonically (lep+jets)
	\item both top quarks decay leptonically (dilepton)
	\end{itemize}
	The full hadronic and lepton+jets channels are dominant and occur $45.7\%$ respectively $43.8\%$ of the time. $10.5\%$ of all $t\bar{t}$ events result in two leptonically decaying top quarks \cite{pdg2016}. This analysis will focus on the lepton+jets channel which is pictured in figure \ref{fig:semilep}.
	% LIVE TIME -> no hadronisation
	% YUKAWA ? 
	% DECAY AND PRODUCTION
	% MC Templates, MC Mass
	% ENERGY SCALE
	% RUNNING MASS	
	\begin{figure}
	\begin{subfigure}{.5\textwidth}
		\begin{tikzpicture}
		\begin{feynman}
		\vertex(a) {\(t\)};
		\vertex[right=of a] (b);
		\vertex[above right=1.5cm of b] (c);
		\vertex[below right=3cm of b] (f1){\(b\)};
		\vertex[below right=1.5cm of c] (f3){\(\overline l\)};
		\vertex[above right=1.5cm of c] (f2){\(\nu\)};

		\diagram* {
		(a)-- [fermion] (b)-- [fermion] (f1),
		(b)-- [boson,edge label'=\(W\)] (c),
		(c)-- [fermion] (f2),
		(c)-- [anti fermion] (f3),};
		\end{feynman}
		\end{tikzpicture}
		\caption{}
		\label{fig:decaya}
	\end{subfigure}
	\begin{subfigure}{.5\textwidth}
		\begin{tikzpicture}
		\begin{feynman}
		\vertex(a) {\(t\)};
		\vertex[right=of a] (b);
		\vertex[below right=3cm of b] (f1){\(b\)};
		\vertex[above right=of b] (c);
		\vertex[above right=1.5cm of c] (f2){\(q\)};
		\vertex[below right=1.5cm of c] (f3){\(\overline q\)};
		\diagram* {
		(a)-- [fermion] (b)-- [fermion] (f1),
		(b)-- [boson,edge label'=\(W\)] (c),
		(c)-- [fermion] (f2),
		(c)-- [anti fermion] (f3),};
		\end{feynman}
		\end{tikzpicture}
		\caption{}
	\end{subfigure}
	\caption{Feynman diagrams \cite{feynman} of a leptonically (a) and hadronically (b) decaying top quark.}
	\label{fig:decay}
	\end{figure}
	
	\begin{figure}
		\centering
		\begin{tikzpicture}
		\begin{feynman}	
		\vertex[blob] (a) {};
		%\vertex (a);
		\vertex[right=of a] (b);
		\vertex[left=of a] (c);
		\vertex[above right=3.5cm of b] (bb){\(b\)};
		\vertex[right=of b] (bW);
		\vertex[left=of c] (cW);
		\vertex[below left=3.5cm of c] (cb){\(\overline b\)};
		\vertex[above right=of bW] (bq){\(q\)};		
		\vertex[below right=of bW] (baq){\(\overline q\)};				
		\vertex[above left=of cW] (cl){\(l\)};		
		\vertex[below left=of cW] (cnu){\(\overline \nu\)};
		\diagram* {
		(a)-- [fermion,edge label'=\(t\)] (b),
		(a)-- [anti fermion,edge label'=\(\overline t\)] (c),
		(b)-- [fermion] (bb),
		(b)-- [boson,edge label'=\(W^+\)] (bW),
		(c)-- [anti fermion] (cb),
		(c)-- [boson,edge label'=\(W^-\)] (cW),
		(bW)-- [fermion] (bq),
		(bW)-- [anti fermion] (baq),
		(cW)-- [fermion] (cl),
		(cW)-- [anti fermion] (cnu),
		};
		\end{feynman}
		\end{tikzpicture}
		\caption{Feynman diagram \cite{feynman} displaying an event from the lepton+jets channel. The centred circle indicates a mechanism to produce a $t\bar{t}$ pair.}
		\label{fig:semilep}
	\end{figure}
	

\section{Unfolding}
	Most analyses at LHC measure distributions of appropriate variables and then compare the obtained results in data with event simulations. In this method the MC samples also include detector effects. What one measures in this case is the real distribution on particle level folded with an unknown detector function. Studying the difference in MC between particle level and reconstruction level it is possible to calculate the probabilities that a measured value in a bin $x_i$ is originating from bin $y_i$ on particle level. The resulting matrix can then be applied to real data to obtain data on particle level. This can then be compared with theory calculations.
	% COMPARE TO THEORY
	% WELL DEFINED MASS
	% CORRECT FOR DETECTOR
	
\chapter{Measurement of the Top Quark Mass}
\label{ch:Measure}
	The top quark mass is an essential parameter to check the Standard Model of particle physics for consistency. For instance, the top quark has, due to its high mass, a large coupling to the Higgs boson. Therefore, it has to be included in correction terms for the Higgs boson mass. This leads to a relation of the masses of top quark, Higgs boson and electro-weak bosons that can be verified with a well known top quark mass, as performed in \cite{ewfit}. In this chapter conventional mass measurement methods are presented and compared to the measurement this analysis aims at. Additionally, previous results from CMS as well as its possible improvements are discussed. 
	
\section{Conventional Mass Measurements}
	Measuring the top quark mass is performed by reconstructing and combining the decay products to calculate back to the initial top quark. Usually a template  fit is used, meaning a comparison to Monte Carlo samples based on various top quark masses. Therefore, a selection is applied to data and simulation with the goal to select preferably $t\bar{t}$ events. Finally, the MC samples with different $m_\text{top}^{MC}$ including a detector simulation are fitted to data. The most precise mass measurement of this kind is achieved combining results of the CMS, ATLAS, CDF, and D0 collaborations. The result of this world average is a value of $173.34 \pm 0.27 (\text{stat}) \pm 0.71 (\text{sys})\;\text{GeV}$ \cite{topmass_combination}. Performing these kinds of measurements, one relies on a correct simulation of $t\bar{t}$ events and is essentially measuring the top quark mass of a given simulation. This mass can then not easily be related to a mass parameter in a Lagrange density and is therefore referred to as the Monte Carlo mass $m_\text{top}^{MC}$. In addition, not all effects of QCD can be calculated \cite{nonperturbative} in simulation. This is why a cut-off scale \cite{cutoff} is defined which directly influences the mass associated with a given simulation (see section \ref{sec:Simulation}). Because of this difficulty, this analysis aims to provide a mass measurement without relying on ambiguities of the mass parameter in Monte Carlo simulation.

\section{Measurement in Boosted Decays}
	The present analysis aims at a mass measurement independent of the mass parameter in simulations. Therefore, it provides a well defined top quark mass measurement which is capable of validating current results as well as simulations. The approach is to obtain a distribution that can be calculated and thus be directly compared to theory. For lepton collisions it is shown in \cite{eejetmass} that a jet mass distribution can be calculated and its peak is sensitive to the top quark mass. Mentioned calculations are done dividing the event in two hemispheres containing one top quark decay each. The jet mass is then the invariant mass of the sum of all particles in one hemisphere. Since events at LHC contain much more objects not belonging to the $t\bar{t}$ system, jets with a finite cone size are defined. All decay products have to fit into the selected cone size, to reconstruct a top quark using a single jet. Therefore, Lorentz boosted top quarks are selected to obtain small distances between the decay products. The jet mass $m_\text{jet}$ in this analysis is then defined as the invariant mass of the jet four-vector. All four-vectors of particles clustered into a jet a summarised to obtain the vector. In conclusion the jet mass reads:
	\begin{equation}
	m_\text{jet} = \sqrt{\left( \sum_{i} p_i \right) \cdot \left( \sum_{i} p_i \right)},
	\end{equation}
	where $i$ runs over all constituents of a jet and $p_i$ indicates the four-vector of the $i$-th constituent. After selecting suitable events, an unfolding is performed to obtain a jet mass distribution comparable to particle level as well as theory calculations. 

\section{Previous Results}
	A measurement of the jet mass in highly boosted $t\bar{t}$ events \cite{torben_paper} has already been performed by CMS with the $8\;\text{TeV}$ dataset corresponding to an integrated luminosity of $19.7\;\text{fb}^{-1}$. The resulting differential cross section measurement and a summary of the uncertainties are shown in Fig. \ref{fig:Torben1} and Fig. \ref{fig:Torben2}, respectively. In the mentioned publication, Cambridge/Aachen jets with a radius parameter of $R=1.2$ were used to reconstruct top quark decays, proving the capability of the method but not reaching high precision. Two main points a subsequent analysis on $13\;\text{TeV}$ could largely improve are statistical uncertainties and effects due to pile-up. First, statistics are now very high in the selected phase space on $13\;\text{TeV}$. Since the $t\bar{t}$ cross section increases by a factor of about three and much more in the boosted region with a centre-of-mass energy of $13\;\text{TeV}$, the statistical uncertainty is expected to decrease. This also influences the choice of jet algorithm and cone size. While in the $8\;\text{TeV}$ analysis large cones are crucial to obtain enough statistics, smaller cone sizes should be possible on $13\;\text{TeV}$. The second limitation is pile-up. Because of the large cone size, pile-up largely influences the measurement, visible as shift of the peak position to values above the top quark mass. Again, a smaller cone on $13\;\text{TeV}$ is expected to decrease this effect. In addition various jet clustering algorithm can be studied to find the most suitable method to reconstruct top quark decays inside one jet.

	\begin{figure}[tb]
		\begin{subfigure}{.5\textwidth}
	    \centering
		\includegraphics [width=\textwidth]{../Plots/Torben/Torben_result_paper}
		\caption{}
		\label{fig:Torben1}
		\end{subfigure}
		\begin{subfigure}{.5\textwidth}
		\centering
		\includegraphics [width=\textwidth]{../Plots/Torben/Torben_error_paper}
		\caption{}
		\label{fig:Torben2}
		\end{subfigure}
		\caption{Unfolded cross section measurement (a) and display of all uncertainties (b) from a similar analysis performed at $8\;\text{TeV}$. Taken from \cite{torben_paper}.}
		\label{fig:Torben}
	\end{figure}	
	

\chapter{Experiment}
\label{ch:Exp}
	Physicists use scattering experiments to dissolve processes at high energy scales to understand the basic building blocks and laws of nature. For this analysis, data from the currently worlds largest scattering experiment LHC is used. Here, protons are brought to collision in a circular particle accelerator. The outcome of the collisions is then measured with several detection systems and is used to test the current status of the standard model. A description of the accelerator (section \ref{sec:lhc}) and the detector (section \ref{sec:cms}) is presented in this chapter.
\section{Large Hadron Collider}
\label{sec:lhc}
	The Large Hadron Collider (LHC) is a circular particle collider with a circumference of $26.7\;\text{km}$ operating at CERN\footnote{Conseil europ\'{e}en pour la recherche nucl\'{e}aire}, Switzerland. At the LHC, protons, or in special runs heavy ions, are accelerated in opposite directions along the ring to perform scattering experiments at high energies. Protons are accelerated to an energy of $6.5\;\text{TeV}$ and then brought to collision with a resulting center-of-mass energy of $\sqrt{s}=13\;\text{TeV}$, almost reaching the design energy of $14\;\text{TeV}$. Protons are taken from a hydrogen source and then accelerated via superconducting cavities in various linear and circular accelerators before injected into the LHC ring. Here, protons are organised in bunches with approximately $100$ billion protons per bunch. There are $2808$ spaces for bunches in the LHC ring with a separation of about $25\;\text{ns}$. In all accelerators, magnet systems provide a handle to lead and form the proton beams. Most importantly, dipole magnets are used to keep the protons on their required orbit. Quadrupole magnets are installed to focus the beam, while higher order magnets are used to correct for higher order effects like energy dependent deviations from the orbit. Four experiments are placed at interactions point of LHC. Besides CMS\footnote{Compact Muon Solenoid}, which is described below in section \ref{sec:cms}, there is another multi-purpose detector with a broad physics program, called ATLAS\footnote{A Toroidal LHC Apparatus}. Additionally there are two more specialised experiments: LHCb\footnote{Large Hadron Collider beauty}, with a focus on $b$-physics and ALICE\footnote{A Large Ion Collider Experiment}, mainly aiming at research of quark-gluon plasma in heavy ion collisions. A display of the LHC complex with its four experiments and preaccelerators is shown in fig. \ref{fig:lhc}.
		\begin{figure}
			\centering
			\includegraphics [width=\textwidth]{../Images/lhc.jpg}
			\caption{Display of the LHC complex with the LHC ring itself and smaller accelerators used to accelerate protons to the required energy for injection in the LHC ring. Taken from \cite{lhc}.}
			\label{fig:lhc}
		\end{figure}
		
	%todo Lumi Abschnitt in pysics of proton-proton... ?
	A very important parameter of a particle collider is its luminosity \cite{luminosity}. It gives an estimate how many collisions take place per area and second, thus, how well the beam is focused at the interaction points. It is calculated via Eq. \ref{eq:lumi}. 
	\begin{equation}
	L = \frac{n N_1 N_2 f}{4 \pi \sigma_x \sigma_y}
	\label{eq:lumi}
	\end{equation} 
	In this formula $n$ denotes the number of bunches in the accelerator, $N_1$ and $N_2$ are the number of protons in the two colliding bunches and $f$ is the collision frequency. The denominator gives the cross sectional area where $\sigma_x$ and $\sigma_y$ describes the spread of the proton beam in $x$ and $y$ direction, respectively. The design luminosity of LHC is $10^{34}\;\text{cm}^{-2}\text{s}^{-1}$ but was already exceeded in the 2016 run. By multiplying the integrated luminosity (see Eq. \ref{eq:intlumi}) with the production cross section of a particular process $\sigma$, one gets an estimate how many events $N$ to expect in one second (see Eq. \ref{eq:number}).
	\begin{equation}
	L_\text{int} = \int L dt
	\label{eq:intlumi}
	\end{equation} 
	\begin{equation}
	N = L_\text{int} \sigma
	\label{eq:number}
	\end{equation} 
	LHC has performed very well and produced a lot of collision data up to now. The total integrated luminosity for every year can be read off from Fig.\ref{fig:LHClumi}.
	\begin{figure}[tb]
		\centering
		\includegraphics [width=.8\textwidth]{../Plots/LHC_Lumi.png}
		\caption{Luminosity of the data sets produced at LHC. The graphs show the development and size of data sets from 2010 (green), 2011 (red), 2012 (blue), 2015 (purple) and 2016 (orange). Taken from \cite{LHClumi}.}
		\label{fig:LHClumi}
	\end{figure}
\subsection{Physics of proton-proton collisions}
	In collision experiments, the center-of-mass energy $\sqrt{s}$ defines the energy that is available to create particles and give them kinetic energy. At LHC, $s$ is set by the momenta of incoming protons via
	\begin{eqnarray}
	s &=& (p_1 + p_2)^2 \\
	  &=& p_1^2 + p_2^2 + 2 p_1 p_2 \\
	  &=& (E_1^2 - |\vec{p}_1|^2 )  + (E_2^2 - |\vec{p}_2|^2 ) + 2 (E_1 E_2 - |\vec{p}_1| |\vec{p}_2| ),
	\end{eqnarray}
	where $p_1$ and $p_2$ are the four-momenta of two colliding protons with energies $E$ and momentum vector $\vec{p}$. Because the two beams have the same energy but opposite directions, the assumptions $E_1 = E_2 = E_\text{proton}$ and $|\vec{p}_1| = |\vec{p}_2|$ hold. With that, the center-of-mass energy reads:
	\begin{equation}
	\sqrt{s} = 2 E_\text{proton}.
	\end{equation}
	Because protons are composite particles, the actual scattering involves quarks or gluons, called partons, carrying only a fraction of the nominal beam momentum $x p_\text{proton}$. The center-of-mass energy of the hard scattering process $\sqrt{\hat{s}}$ is therefore only a fraction of the stated $13\;\text{TeV}$:
	\begin{equation}
	\sqrt{\hat{s}} = \frac{(x_1 + x_2)}{2} \sqrt{s}
	\end{equation} 
	Here, $x_1$ and $x_2$ indicate the fraction of momentum two colliding partons carry. Because this fraction is not known, the center-of-mass energy of the hard process is not either. Therefore, for data analysis one has to use variables, that do not depend on the initial state of the partons which are explained in section \ref{sec:coordinate}.
	%todo pdf erwaehnen??
	
\subsubsection{Pile-Up}
	The high rate of collisions of LHC is required to collect a large amount of data and be able to see even very rare processes. But the high rate also has its downside. Since multiple proton-proton interactions take place per bunch crossing, not only one interesting but several other scattering events, mostly soft QCD processes, are seen simultaneously in the detector. This effect is called pile-up. If not corrected for, energy measurements do always include particles not originating from the hard scattering one is interested in. To reduce pile-up effects, it is crucial to be able to resolve different primary vertices belonging to different interactions.
\subsubsection{Underlying Event}
	As result of LHC doing collision experiments with protons, thus composite particles, it has to be accounted for more than one scattering process in a proton-proton collision. This effect is called underlying event and addresses collisions where multiple partons from two colliding protons interact. Since both interactions share the same interaction point, they can hardly be separated. Hence, underlying event has to be understood and is included into simulation of events.
	
\section{CMS Detector}
\label{sec:cms}
	The 'Compact Muon Solenoid' (CMS) experiment is a multi purpose detector at LHC. It is designed to measure momentum and energy of particles produced in proton-proton interactions. With a total weight of $14000\;\text{t}$, the mass of the CMS detector is dominated by a steel return yoke installed to lead the magnetic field originating from a solenoid. The detector has a cylindrical shape with a length of $28.7\;\text{m}$ and a diameter of $15.0\;\text{m}$. The CMS detector is built in onion-like layers of sub-detectors with different purposes described in the following sections. To identify and distinguish particles, the fact that different particles leave different signatures in the detector is utilised. How particles are reconstructed is detailed in chapter \ref{ch:Reco}. Besides the detector systems, a very important component is the Trigger, described in section \ref{sec:trigger}, providing fast decisions if an event is discarded or interesting enough to be stored. The design of CMS was chosen to cover a wide range of physics approaches. Nevertheless, a focus was set on the discovery of the Higgs boson, which was announced in 2012. Components, design and the general construction of the CMS detector are shown in Fig. \ref{fig:CMS}.  
	\begin{figure}[htb]
		\centering
		\includegraphics [width=.95\textwidth]{../Images/CMS_Full.png}
		\caption{Full view of the CMS detector. Basic properties as weight and dimensions are summarised in a table in the left upper corner. Components of the detector are titled and shown in different colours. Taken from \cite{CMSfull}.}
		\label{fig:CMS}
	\end{figure}
	
\subsection{Coordinate System}
\label{sec:coordinate}
	The coordinate system used in the CMS experiment is based on cartesian and right-handed coordinates. The origin is set in the center of the CMS detector. To define the direction of the axes other fix points are set. The $x$-axis point in the direction of the center of the LHC ring, the $y$-axis points up and the $z$-axis is defined parallel to the beam axis. Important variables used in analysis of CMS data are the angles $\phi$ and $\theta$. $\phi$ is defined as the angle in the $x$-$y$-plane measured from the $x$-axis and $\theta$ describes the angle from a given point to the beam axis. Because the LHC is a hadron collider and physical events are therefore not symmetric in $\theta$ it is useful to construct the Lorentz invariant variable $\eta$:
	\begin{equation}
	\eta = - \ln \left[\tan\left( \frac{\theta}{2}\right) \right]
	\end{equation} 
	The distance $\Delta R$ between two objects $i$ and $j$ is calculated using the differences $\Delta \phi = \phi_i - \phi_j$ and $\Delta \eta = \eta_i - \eta_j$:
	\begin{equation}
	\Delta R = \sqrt{\Delta \phi ^2 + \Delta \eta ^2}
	\end{equation}
	An important quantity used in this analysis is the transversal momentum $p_T$ which is constructed out of the $x$ and $y$-components ($p_x$ and $p_y$) of the total momentum of an object:
	\begin{equation}
	p_T = \sqrt{p_x^2 + p_y^2}
	\end{equation} 
	It is practical to not consider the $z$ component in a hadron collider because it depends on the initial state of interacting partons which is unknown. The $p_T$ sum of all objects is expected to vanish every event. If it is not, the $p_T$ may be reconstructed wrong for some objects or objects left the CMS experiment undetected.
	
\subsection{Solenoid}
	The CMS detector uses a magnetic field to bend paths of electromagnetic charged particles. With the resulting curvature radius and a known magnetic field strength, one is able to calculate the momenta of those particles. Therefore, a superconducting solenoid is installed to provide a magnetic field of $3.8\;\text{T}$ inside the tracking system (explained in section \ref{sec:tracker}). The solenoid has a length of $12.9\;\text{m}$ and a diameter of $5.9\;\text{m}$. Tracker and calorimeters are placed inside the solenoid, the muon system is installed around the magnet inside iron return yokes. The iron yokes are used to guide the magnetic flux outside the solenoid. The magnetic fields purpose on the outside is to bend tracks of muons inside the muon system. A solenoid form was chosen because it provides a constant magnetic flux inside the tracking system.
	
\subsection{Tracker}
\label{sec:tracker}
	The purpose of the tracking system \cite{CMSdetector} is to measure the momentum and charge of particles. It is taken advantage of the Lorentz force which changes the momentum of charged particles in a magnetic field. Paths of those particles are therefore bended with a bending radius proportional to the momentum in a constant magnetic field. The tracking system is installed at the innermost of the detector and covers, with a length of about $5.4\;\text{m}$ and radius of $1.1\;\text{m}$, a range of $| \eta | < 2.5$. The full tracker is constructed in multiple layers and sub systems. A view of the tracking system is presented in Fig. \ref{fig:tracker}, all systems are described below. Per path of a particle one spatial point per layer is measured. These points are the input for a track finding algorithm. The largest uncertainties of the measured momenta rise from the spatial resolution of the tracker. Additionally the reconstruction depends on the momentum of the particle. If it gets too large, the tracks are barely bended and the bending radius cannot be measured with high precision. Accordingly, the uncertainty of a measurement grows with momentum of a particle. Moreover, tracker information is used to reconstruct primary and secondary vertices. A reliable detection of secondary vertices is crucial for b-tagging, which is explained in section $\ref{sec:btag}$.
	\begin{figure}[tb]
			\centering
			\includegraphics [width=\textwidth]{../Images/Tracker.png}
			\caption{Sketch of the CMS tracking detector. All sub systems namely pixel and strip detector with inner barrel (TIB), outer barrel (TOB), inner endcaps (TID) and outer endcaps (TEC) are shown. Additionally the range in $\eta$, radius ($r$) and $z$-direction can be read off. Taken from \cite{CMSdetector}.}
			\label{fig:tracker}
	\end{figure}
	\\
	The innermost tracking system is a pixel detector consisting of three layers in the barrel region with a distance to the beam axis of $4.4\;\text{cm}$, $7.3\;\text{cm}$ and $10.2\;\text{cm}$, respectively. Additionally, two endcap discs are installed. Each of the $66$ million silicon pixels extends to $100\;\text{\textmu m} \times 150\;\text{\textmu m}$. With a thought through arrangement of pixels a resolution of $10\;\text{\textmu m}$ in the $r-\phi$-plane and $20\;\text{\textmu m}$ in $z$-direction is reached. Outside the pixel detector a silicon strip tracker is built. It is further divided into four sub systems: an inner barrel (TIB) with four layers, an outer barrel (TOB) with six layers, two inner endcaps (TID) consisting of three discs and two outer endcaps (TEC).

\subsection{Calorimeters}
	To define the type of a particle, one has to measure not only momentum but also energy. Therefore, outside the tracker a colorimetry system with two types of calorimeters is installed. Additionally, showers seen in the calorimeters are used as seed for track finding algorithms and particle flow (section \ref{sec:pf}). The underlying principle of this sub detector is to absorb all of the energy a particle carries and convert it to a measurable quantity. Energy deposited in the calorimeter is then transformed into a light signal using scintillating materials. The intensity of the light signal is proportional to the energy of the incoming particle. The material also has to be dense enough to stop an incoming particle entirely and transform all its energy into the light signal. As a measure for calorimeter thickness, the radiation length $X_O$ for electrons is defined as the path where an electron looses $\frac{1}{e}$ of its initial energy. Analogously, a nuclear interaction length $\lambda_n$ for hadrons is introduced, taking strong interaction into account. Of course the value of both, $X_0$ and $\lambda_n$, depends on the used material. Finally, the light is collected and measured with photo multipliers and can be converted into an energy value. Measurements from calorimeters typically get more precise with increasing energy. Therefore, they provide a good alternative method in cases where the measurement of the tracking system gets worse. The typical resolution of calorimeters is shown in Eq. \ref{eq:calo}.
	\begin{equation}
	\frac{\sigma_E}{E\left[\text{GeV}\right]} = \frac{a}{\sqrt{E\left[\text{GeV}\right]}} \oplus b \oplus \frac{c}{E\left[\text{GeV}\right]}
	\label{eq:calo}
	\end{equation}
	In this formula $\frac{\sigma_E}{E}$ denotes the relative energy resolution depending on the energy $E$, a statistical term $a$, a constant term $b$ - including systematic errors or shower leakage - and a noise term $c$, containing electronics and pile-up noise. At high energies the $b$-term is dominant but can be reduced by adequate calibration. 
\subsubsection{Electromagnetic Calorimeter}
	The inner part of the calorimetry system, the electromagnetic calorimeter (ECAL), is supposed to measure the energy of electrons and photons in a range up to $|\eta| = 3.0$. It consists of \chem{PbWO_4} crystals and is designed to deliver a very good spatial resolution. The crystals are $230\;\text{mm}$ long, which corresponds to $25.8$ times the radiation length of an electron, and have a face area of $22\;\text{mm} \times 22\;\text{mm}$. With this, photons, which are not seen in the tracker, can be reconstructed with a high precision. Even the direction of photons can be resolved from shower reconstruction. This design choice was driven by the decay of a Higgs boson into two photons ($H\rightarrow \gamma \gamma$) where the Higgs boson mass can be reconstructed from the four-momenta of the photon pair. A performance study presented in \cite{EGammaPerformance} names a resolution of less than $2\%$ for $45\;\text{GeV}$ electrons from $Z$ boson decays in the barrel region.

\subsubsection{Hadronic Calorimeter}
	Since hadrons have a larger absorption length they mostly pass the electromagnetic calorimeter and are absorbed and measured in the hadronic calorimeter (HCAL). Two alternating layers of different materials are used. One is to absorb the energy of an incoming particle and consists of brass plates. The other one is made out of a plastic scintillator, that acts as active material. The barrel part of the HCAL begins in a distance of $r=1.77\;\text{m}$ to the interaction point and reaches up to the solenoid at $r=2.95\;\text{m}$ covering a range of $|\eta| < 1.3$. The thickness corresponds to about ten times the hadronic interaction length. To cover a range of $1.3 < |\eta| < 3$, endcaps are installed. Additionally, forward calorimeters at a distance of $11.1\;\text{m}$ in $z$-direction to the interaction point are used to cover the range of $3 < |\eta| < 5$. To account for and measure leakage, a further outer calorimeter is installed outside the solenoid. 
	%todo GENAUIGKEIT: referenzwert wie bei E= XX ist Aufloesung YY

 
\subsection{Muon System}
\label{sec:muonsystem}
	Because of the possibility to reconstruct muons with very high precision, they play an important role in the design of CMS. Thus, a whole system is installed just to identify and measure muons. Muons at typical energies at LHC are minimal ionizing particles and are therefore not absorbed in the calorimeters. Thus, it is possible to use an additional tracking detector for muons at the very outside of the detector. As a muon tracker various versions of gaseous detectors are installed. In the barrel part, drift tubes and resistive plate chambers are embedded in the iron return yoke of the magnet where a magnetic field of about $2\;\text{T}$ is present. In addition, two endcaps with resistive plate chambers in combination with cathode strip chambers are installed. The design choice is due to the expected flux of muons in the different regions of the detector. While muons occur rarely in the barrel part, they are predicted to appear in high rates in the endcaps.
	%todo Genauigkeit

\subsection{Trigger}
\label{sec:trigger}
	At the interaction points, proton bunches are brought to collision every $25\;\text{ns}$. At the design luminosity, per crossing about $20$ interactions take place. This adds up to $8 \cdot 10^8$ interactions per second.	It is impossible to store that amount of data with this rate and the total required storage capacity is not feasible either. Additionally, it would need a tremendous number of computing cores to analyse this data. Therefore, a trigger system is installed selecting only interesting events to reduce the amount of data. For the computation system it is required to reduce the rate by a factor of the order $10^6$ to a few $100\;\text{Hz}$. This is achieved with two trigger stages: a hardware based Level-1 (L1) Trigger\cite{L1} and a software based High-Level Trigger (HLT)\cite{HLT}. The L1 Trigger is designed to deliver an output of about $30\;\text{kHz}$, the remaining reduction is done by the HLT.	To decide which events are worth storing and which will be neglected different criteria are defined. Fast components of the detector like calorimeter and muon system are read out and deliver information to the trigger system. Then the trigger makes decisions based on this information and stores for example events with high energy muons or jets. CMS then provides a trigger menu where data streams are ordered by the type of object that fired the trigger.

	
\chapter{Reconstruction of Objects}
\label{ch:Reco}
	From all kinds of information provided by the detector systems has to be interpreted as physical objects in order to analyse the recorded data. Therefore algorithms are run to link the information from the detector to usable objects like muons, electrons or jets. Objects are defined by their detection in different detector systems. Paths of all detectable objects through the detector are sketched in Fig. \ref{fig:CMS_reco}.
	\begin{figure}[tb]
		\centering
		\includegraphics [width=.8\textwidth]{../Images/CMS_Slice_white.png}
		\caption{Slice through the CMS detector looking in direction of the beam pipe. Tracks of different particles on their way through the layers of the detector are displayed. From left to right, you can see the tracker, calorimeters, the superconducting solenoid and the muon system. Taken from \cite{CMSslicewhite}.}
		\label{fig:CMS_reco}
	\end{figure} 

\section{Particle Flow}
\label{sec:pf}
	CMS uses a special algorithm called particle flow \cite{particleflow} to combine information from tracker and calorimeters and thus reconstruct objects to a high precision. Tracks from the inner tracker are extrapolated into the calorimeters. If a shower fits to the track, information from these two sub detectors are combined. Since only charged particles will interact with the tracking system, showers from neutral hadrons and photons cannot be associated with a track.
\section{Electrons}
	Electrons are reconstructed in the tracker and electromagnetic calorimeter. Due to the magnetic field, electrons will take a curved path through the tracking system. They are then stopped in the ECAL where their energy is measured.
\section{Muons}
	Muons have a very low probability to interact with the Calorimeters of CMS. Because of that, muons are identified by hits in the muon chambers. Combining informations from the muon system and the inner tracker leads to a very precise measurement of muons in the CMS detector. Therefore muons are very reliable objects in CMS data analyses. An object is announced a muon if %todo KRITERIEN MUON
\section{Jets}
	Jets are objects, used to reconstruct quarks. Because of confinement quarks cannot exist isolated but hadronize. This results in a particle shower consisting of hadrons. To reconstruct the initial quark one needs to sum up all particles from the final shower. To combine the tracks in a well defined way jet algorithms are used. Two important requirements for an jet algorithm are to be infrared and collinear safe. The first criterion means, a jet should not change when including or excluding soft radiation. Collinear safety addresses collinear splitting of particles in a jet which should also not change the jet. All presented jet algorithms in this thesis fulfil these requirements.
\subsection{Anti-$k_T$ and Cambridge/Aachen Jet Algorithms}
	In CMS the common way to cluster jets from the detected particles is to use iterative jet algorithms. Thus particles are clustered step by step until an abort criterion is reached. The most common group of iterative jet algorithms are $k_T$-like algorithms. As an input, a list of objects reconstructed in the detector is given. The Algorithm then calculates two quantities $d_{ij}$ (Eq. \ref{eq:dij}) and $d_{iB}$ (Eq. \ref{eq:iB}) for each pair of objects $i$ and $j$:
	\begin{equation}
	d_{ij} = min (k_{T,i}^{n}, k_{T,j}^{n})  \frac{\Delta R_{ij}^2}{R^2}
	\label{eq:dij}
	\end{equation}
	\begin{equation}
	d_{iB} = k_{T,i}^{-n}.
	\label{eq:iB}
	\end{equation}
	Here, $d_{ij}$ describes a effective distance between two objects $i$ and $j$, $d_{iB}$ is a distance measure from object to beam axis. The variable $k_T$ is the transverse momentum of an object, $\Delta R_{ij}$ denotes the distance between objects $i$ and $j$ and $R$ is a constant parameter that defines the radius of the resulting jet. Changing the exponent $n$ influences the order of clustering. If $n=-2$, the algorithm is called Anti-$k_T$ \cite{antikt} and particles with high transverse momenta are clustered first. Using the Cambridge/Aachen \cite{CA1, CA2} algorithm, and therefore choosing $n=0$ will not weight the objects by momenta and provides a pure geometrical measure. When $d_{ij}$ is smaller than $d_{iB}$ both objects $i$ and $j$ are combined and both quantities are calculated again. At some point, $d_{iB}$ will be smaller than any $d_{ij}$, then object $i$ is called a jet and is removed from the list of objects. This procedure is repeated until the list of objects is empty. With the anti-$k_T$ jet algorithm, particles with large transverse momenta are clustered first because $d_{ij}$ (Eq. \ref{eq:dij}) is easily smaller then $d{iB}$ for large $p_T$. The resulting jets are very circular. 
	%todo plot mit AK shape.
	%todo KT auch erwähnen?

\subsection{HOTVR Jet Algorithm}
	Another approach to cluster jets is the 'heavy object tagger with variable R' (HOTVR) \cite{hotvr}. This algorithm does not use a constant radius parameter $R$ but a $p_T$ dependent effective $R=R_\text{eff}(p_T)$ (see Eq. \ref{eq:HOTVR}). Thus the $R_\text{eff}$ decreases with increasing $p_T$ leading to smaller jets when the decay products are expected to be more close because of the Lorentz boost. The $p_T$ dependence is scaled with a parameter $\rho$ with a default value of $600\;\text{GeV}$. Furthermore, upper and lower boundaries for the jet radius can be set. The default values are $R_\text{min} = 0.1$ and $R_\text{max} = 1.5$. 	
	\begin{equation}
	\label{eq:HOTVR}
	  R_\text{eff} =
	   \begin{cases}
	     R_\text{min} & \text{for } \rho / p_T < R_\text{min} \\
	     R_\text{max} & \text{for } \rho / p_T > R_\text{max} \\
	     \rho / p_T & \text{else}  
	   \end{cases}
	\end{equation}
	
	\noindent This effective $R$ is then used with the equations of the Anti-$k_T$ algorithm described earlier (see Eq. \ref{eq:dij} - \ref{eq:iB}). Additionally, a mass-jump criterion is used.
	%todo mass jump Kriterium, auch mit Formel
	

\subsection{XCone Jet Algorithm}
\label{sec:xcone}
	XCone \cite{xcone} is an exclusive jet algorithm, returning conical jets. It is well suited for analysis where the final state and therefore the expected number of jets is known since it returns a fixed number of jets. Thus, a physical final state has direct influence on the jet finding. \\
	Starting with a fixed number of jet axes $N$ the algorithm calculates the direction of these axes by minimizing the N-jettiness variable. N-jettiness is a measure for how N-jet-like an event looks. The definition is shown in Eq. \ref{njettines}.
	\begin{equation}
	\tilde{\tau}_N = \sum_i \min\{\rho_\text{jet}(p_i, n_1), \dots, \rho_\text{jet}(p_i, n_N), \rho_\text{beam}(p_i)\}
	\label{njettines}
	\end{equation}
	Once the minimizing process converges, all particles inside a radius $R$ from a jet axis are added to one jet. Here, it is important to mention that XCone handles overlapping jets differently than common iterative jet algorithms. HOTVR and the Anti-$k_T$ algorithm remove already clustered objects from the input list. Thus, one of two nearby jets is crescent shaped while the other one is circular. XCone however, will assign every object to the closest jet resulting in a straight border between two jets. This feature is shown in Fig. \ref{fig:XCone_overlap}. 
	\begin{figure}[tb]
		\centering
		\includegraphics [width=.6\textwidth]{../Plots/XCone_Overlap.png}
		\caption{The separation of nearby jets clustered with the XCone jet algorithm in a $t\bar{t}$ event, requiring six jets with a radius of $0.5$, is depicted. The area of jets is shown in the $\phi$-$y$-plane. Here $y$ indicates the rapidity. Taken from \cite{xcone}.}
		\label{fig:XCone_overlap}
	\end{figure} 
	A more general advantage of XCone is that the N-jettiness variable is often used to do theory calculations of particle physics events, the XCone algorithm is easier to include in these calculations.
	
\subsection{Jet Energy Corrections}
\label{sec:jec}
	Jets found by clustering sequences are furthermore corrected to account for non linearities in the detector response and differences between data and simulation because of modelling \cite{JEC}. The approach is to factorize different sources of variance and scale momentum vectors of jets with a factor addressing each source. Thus, four factors are introduced which then result in one final $p_T$ and $\eta$ dependent correction factor as shown in Eq. \ref{eq:jec} and \ref{eq:cjec}.
	\begin{equation}
	p^{\text{corrected}} = C_{\text{JEC}} \cdot p^{\text{raw}}
	\label{eq:jec}
	\end{equation}
	where
	\begin{equation}
	C_{\text{JEC}} = C_{\text{offset}} \cdot C_{\text{MC}} \cdot C_{\text{rel}} \cdot C_{\text{abs}}
	\label{eq:cjec}
	\end{equation}
	Firstly, a factor is applied to address the additional energy in a jet because of pile-up ($C_{\text{offset}}$), which is not taken care of in simulation. The factor is constructed $p_T$ dependent to subtract a constant energy from a jet in data. Secondly, reconstructed jets are corrected to match the generated jet momentum. This is done by calculating a factor $R=\frac{p_T^{\text{rec}}}{p_T^{\text{gen}}}$ in different $p_T$ and $\eta$ regions and apply this to reconstructed jets as $C_{\text{MC}} = \frac{1}{<R>}$, where $<R>$ indicates a mean in the given region. Next, a relative correction $C_{\text{rel}}$ is applied which accounts for non linearities in the detector response. Therefore, a $\eta$ dependent correction factor is used. Lastly, an absolute factor $\cdot C_{\text{abs}}$ derived from data in $Z$ + jets and $\gamma$ + jets events to fit the absolute energy scale.

\subsection{b-Jets}
\label{sec:btag}
	In this analysis jets originating from a bottom quark are identified to reduce background. To identify a jet as an b-jet the "Combined Secondary Vertex" (CSV) algorithm is used. Since b-hadrons have a large lifetime of $1.5\;\text{ps}$ they travel about $450\;\text{\textmu m}$ in the detector before decaying. This leads to a secondary vertex at the spatial point where the hadron decays which can be reconstructed in the tracking system. Additionally the composition of hadrons in a b-jet is different from other jets.
	%todo ALLES AUßER secondary vertex NOCHMAL NACHLESEN 	
	%todo working points beschreiben
	%todo Quelle
	These properties are taken advantage of in the CSV algorithm. 
	
\section{$\cancel{E}_T$ and $S_T$}
	With the information of mentioned objects two important variables are defined. The missing transverse energy $\cancel{E}_T$ is defined to estimate the energy carried away by particles which leave the experiment undetected. Summing up all transverse momenta of each particle in the final state returns the $p_T$ of the system in the initial state which is $p_T = 0$ at LHC. Thus the transverse energy of all undetected particles is defined as the absolute value of the negative sum over all transverse momenta of detected objects (Eq. \ref{eq:MET}). Since every object, independent from it's $\eta$, has to be taken into account for this variable, $\cancel{E}_T$ is independent from the selection applied. The missing energy is due to neutrinos which can not be detected with CMS because of their low probability to interact with the detector material or a new physics state. 
	
	\begin{equation}
		\cancel{E}_T = \left| - \sum_{\textrm{leptons, jets}}^{} \vec{p}_T \right|
		\label{eq:MET}
	\end{equation}
	
	\noindent Another important variable to describe an event is $S_T$. As represented in Eq. \ref{eq:ST} is defined as the scalar sum of all transverse momenta of reconstructed objects plus the missing transverse energy mentioned above. Different from $\cancel{E}_T$ only objects surviving the selection are considered. $S_T$ gives an estimate how much energy one event contains and is therefore used to distinguish between low energy QCD and high energy processes. 
	
	\begin{equation}
		S_T = \left( \sum_{\textrm{leptons, jets}}^{} \vert \vec{p}_T \vert \right) + \cancel{E}_T
		\label{eq:ST}
	\end{equation}
	
	\noindent Additionally, a definition for $S_T$ which just takes leptonic activity into account is used in this analysis. It is similarly defined and referred to as $S_T^{\text{lep}}$ (see Eq. \ref{eq:STlep}).
	
	\begin{equation}
		S_T^\text{lep} = \left( \sum_{\textrm{leptons}}^{} \vert \vec{p}_T \vert \right) + \cancel{E}_T
		\label{eq:STlep}
	\end{equation}	

\chapter{Data and Monte-Carlo Simulation}
	In almost every particle physics analysis data is compared to simulations. Physicists have to rely on simulations because it is not possible to purely calculate the outcome one sees in the detector. In this section the used data set is described as well as the production of a Monte-Carlo Simulation of a given process.
	\section{Data}
	This Thesis analyses data recorded by the CMS detector in 2016 at a centre-of-mass energy of $13\;\text{TeV}$. The size of the data set corresponds to an integrated luminosity of ???. 
	\section{Event Generator and Parton Shower}
	\section{Monte-Carlo samples}
	%todo MC hat auch GEN!
	Additionally MC samples listed in table \ref{MC_Tab} are processed. The most important simulation for this analysis is of course the $t\bar{t}$ sample. The main background processes are $W$+jets and single-top production.
	
	\begin{landscape}
	\begin{table}
	\centering
	 \begin{tabular}{|l|l|r|l|r|}
	 	\hline
	 	Process & Sample & Cross Section [pb] & MC Generator & Number of Events \\
	 	\hline
	 	$t\bar{t}$ & $0 < M_{t\bar{t}} < 700$ & 831.76 & & \\
	 	$t\bar{t}$ & $700 < M_{t\bar{t}} < 1000$ & 76.605 & & \\
	 	$t\bar{t}$ & $1000 < M_{t\bar{t}} < \infty$ & 20.578 & & \\
	 	\hline
		Single Top &t-channel  & & & \\
		Single Top & t-channel (anti top) & & & \\
		Single Top & tW-channel & & & \\
		Single Top & tW-channel (anti top) & & & \\
		Single Top & s-channel & & & \\
		$W$+jets & $100 < S_T < 200$ & & & \\
	 	$W$+jets & $200 < S_T < 400$ & & & \\
	 	$W$+jets & $400 < S_T < 600$ & & & \\
	 	$W$+jets & $600 < S_T < 800$ & & & \\
	 	$W$+jets & $800 < S_T < 1200$ & & & \\
	 	$W$+jets & $1200 < S_T < 2500$ & & & \\
	 	$W$+jets & $2500 < S_T < \infty$ & & & \\
	 	$Z$+jets & & & & \\
	 	QCD (muon enriched) & & & & \\
	 	QCD (EM enriched) & & & & \\
	 	$WW$& & & & \\
	 	$WZ$& & & & \\
	 	$ZZ$& & & & \\
	 	 \hline
	 \end{tabular}
	\caption{Summary of data sets and MC samples used in this analysis. Assumed cross section, MC generator and number of events are displayed for each sample.}
	\label{MC_Tab}	
	\end{table}
	\end{landscape}
% !TEX root = Master_Thesis.tex
\chapter{Analysis}
\label{ch:Ana}
	This chapter covers the analysis performed in this thesis. The general strategy and the goals of this analysis are given in section~\ref{sec:strategy}. A suitable jet clustering algorithm is studied and chosen on particle level in section~\ref{sec:jet_studies}. Here, also the selection on particle level is discussed. In section~\ref{sec:selection}, details of the selection on reconstruction level are presented. Furthermore, the final measurement phase space including a jet energy correction for XCone jets is discussed, resulting in a jet mass distribution which is then unfolded. The unfolding procedure is presented in section~\ref{sec:unfolding}.
	
\section{Analysis Strategy}
\label{sec:strategy}
	This analysis aims at highly boosted $t\bar{t}$ final states where all decay products from the fully hadronic top quark decay merge into a single jet. For this measurement the lepton + jets channel is used, where the event is tagged with the lepton, suppressing backgrounds. The measurement of the jet mass is then performed on the hadronically decaying top quark. The detailed selection is presented in section~\ref{sec:selection}. In the required phase space the distribution of the jet mass reconstructing a top quark decaying into quarks ($t\rightarrow b q \bar{q}'$) is measured. Afterwards an unfolding is performed by using the TUnfold \cite{tunfold} software package. The goal is to compare data unfolded to particle level with first principle calculations. Studies on particle level are presented in section~\ref{sec:jet_studies} to find a jet algorithm that fits this purpose.
	

\section{Jet Studies on Particle Level}
\label{sec:jet_studies}
	For this analysis, it is crucial to choose a suitable jet algorithm and cone size. The previously mentioned analysis at $8\;\text{TeV}$ \cite{torben_paper} uses Cambridge-Aachen jets with the radius parameter set to $R=1.2$ for its measurement. This large radius was chosen to compensate for low statistics in the boosted $t\bar{t}$ regime at $8\;\text{TeV}$. Since the cross section of $t\bar{t}$ production is much higher at a center-of-mass energy of $13\;\text{TeV}$ a smaller cone is expected to be applicable for this analysis. Additionally, jets clustered with Anti-$k_T$, HOTVR and XCone algorithms are studied and optimized in sections~\ref{sec:AK}, \ref{sec:HOTVR2} as well as \ref{sec:XCone_strat} and finally compared in section~\ref{sec:jet_comp} to find the algorithm most suitable for this analysis. The goal is to select a jet algorithm which returns jets in which all decay products of a hadronically decaying top quark are merged. In this case, the jet mass $M_\text{jet}$ is sensitive to the top quark mass $M_\text{top}$. The jet mass distribution should return a sharp peak at the top quark mass of around $173\;\text{GeV}$ to be able to extract the top quark mass. Jet studies are performed with a $t\bar{t}$ simulation using the information of MC simulations at particle level. The detailed selection is described in the following.

\subsection{Selection on Particle Level}
\label{sec:GenSel}
	All studies on particle level are performed with a simulated $t\bar{t}$ sample. Several selection criteria are used to select boosted top quark decays in the lepton + jets channel. Since the selection is applied on particle level, also particle level information is used. The selection reads:
	\begin{itemize}
	\item exactly one prompt electron or muon from the $W$ boson decay,
	\item veto on additional leptons,
	\item $p_T^{\text{1st jet}} > 400\;\text{GeV}$,
	\item $p_T^{\text{2nd jet}} > 200\;\text{GeV}$,
	\item Veto on additional jets with $p_T > 200\;\text{GeV}$,
	\item $\Delta R (\text{lepton, 2nd jet}) < \text{jet radius}$ and
	\item $M^{\text{1st jet}} > M^{\text{2nd jet + lepton}}$.
	\end{itemize}
	The first jet refers to the leading jet in $p_T$ of the respective jet clustering algorithm. It is expected to be originating from the hadronically decaying top quark, the second one is expected to contain the products of the leptonically decaying top quark. The purpose of the $p_T$ thresholds is to select boosted top decays. The veto on additional jets is set to select $t\bar{t}$ events where only one jet per top quark decay is found. Note, that the veto is not present for studies of the XCone algorithm since it will always return exactly two jets in this set up. A cut on the distance between lepton and second leading jet in $p_T$ enriches the sample in boosted topologies where the lepton is close to or inside the jet of the leptonically decaying top quark. This criterion is also obsolete for XCone, because of the selected clustering sequence (see section~\ref{sec:XCone_strat}). To suppress events where not all decay products of the hadronically decaying top quark end up in the jet with the highest transverse momentum, a mass criterion $M^{\text{1st jet}} > M^{\text{2nd jet + lepton}}$ is set. The criterion includes the assumption that the mass of the jet on the leptonic side is lower because the neutrino cannot be reconstructed. This selection is applied to every jet algorithm output to be able to compare these different approaches. All shown distributions are scaled to match a integrated luminosity of $37.76\;\text{fb}^{-1}$.

\FloatBarrier %draw figures of previous section before the new one starts	
\subsection{Studies with Anti-$k_T$}
\label{sec:AK}	
	Since the top quark decay should be reconstructed with one jet, all decay products need to end up inside the defined jet cone. Choosing different cone sizes has various effects. When the cone is small, not all decay products may end up in the jet and the jet mass is reconstructed smaller than the top mass. If the cone size is large, the probability to include additional radiation and pile-up grows and the resulting jet mass is reconstructed too high.	As a starting point Anti-$k_T$ jets with a radius of $0.8$ are selected since this is the CMS internal standard to reconstruct top quark jets. AK jets with a radius of $0.8$ and $1.2$ are presented in Fig.~\ref{fig:GEN_AK08} and \ref{fig:GEN_AK12}, respectively, to study the influence of the cone size. Additionally, a matching to generated particles is performed. If all three decay products of the top quark are clustered into the jet, the jet is called 'matched'. One can see that the distribution for AK8 jets features a shoulder of values smaller than the top quark mass. This is due to the higher fraction of 'not matched' events. In this case, these are events where not every decay product ends up in the jet. AK12 jets on the other hand often return a mass higher than the top quark mass which is due to underlying event. A larger cone size has a higher probability of including particles not originating from the top quark decay. Furthermore, even the peak position for AK12 jets is reconstructed above the top quark mass. It is also worth mentioning that with a larger cone size more events survive the selection criteria because a large jet sums up more particles and the jet thus acquires higher transverse momentum. Because of the better resolution in the peak region and a smaller tail, AK8 fits the purpose of this analysis well and is further used. To reduce additional energy clustered into the jet further, grooming algorithms like soft drop are used. A comparison of AK8 jets with and without soft drop is depicted in Fig.~\ref{fig:GEN_AK08sd}. With soft drop applied, the $W$ peak at $80\;\text{GeV}$ can be clearly identified. Additionally, jet masses above the top quark mass are suppressed. The distribution of AK8 jets with soft drop (Fig.~\ref{fig:GEN_AK08sd1}) will be compared to other clustering methods in section~\ref{sec:jet_comp}.

	\begin{figure}[tb]
		\begin{subfigure}{.5\textwidth}
	    \centering
		\includegraphics [width=\textwidth]{../Plots/GenStudies/AK08_matching}
		\caption{}
		\label{fig:GEN_AK08}
		\end{subfigure}
		\begin{subfigure}{.5\textwidth}
		\centering
		\includegraphics [width=\textwidth]{../Plots/GenStudies/AK12_matching}
		\caption{}
		\label{fig:GEN_AK12}
		\end{subfigure}
		\caption{Comparison of jet mass distributions of AK8 (a) and AK12 (b) jets. A smaller cone size (a) leads to a lower reconstructed mass while a large cone (b) returns higher masses. The fraction of 'matched' and 'not matched' events is shown in the histograms.}
	\end{figure}
	
	\begin{figure}[tb]
	    \centering
		\includegraphics [width=.5\textwidth]{../Plots/GenStudies/AK08softdrop_matching}
		\caption{Jet mass distribution of AK8 jets with soft drop. The influence of the grooming is especially visible between the $W$ boson mass and the top quark mass in comparison to Fig.~\ref{fig:GEN_AK08}.}
		\label{fig:GEN_AK08sd1}
	\end{figure}
	
\FloatBarrier %draw figures of previous section before the new one starts			
\subsection{Studies with HOTVR}
\label{sec:HOTVR2}
	Another approach to find an appropriate cone size is to use not a constant, but $p_T$ dependent radius parameter. Since decay products are Lorentz boosted with high momentum, the higher the $p_T$, the smaller cone size is necessary to contain all decay products. Thus, HOTVR (see section~\ref{sec:HOTVR}) directly addresses this. The default settings set the effective radius to $R_\text{eff} = \frac{600\;\text{GeV}}{p_T}$, which corresponds to a maximum radius of $1.5$ with the used selection of jets with $p_T > 400\;\text{GeV}$. The result is visible in Fig.~\ref{fig:GEN_HOTVR}. Due to the large cone size for the lowest possible jet momentum, the distribution shows a large tail. In comparison to the AK12 distribution, the peak is narrower while HOTVR does not feature a shoulder like AK8 does. The parameter to tune the HOTVR clustering is $\rho$. By lowering $\rho$, the effective radius shrinks. Figure~\ref{fig:GEN_HOTVRrho} shows a comparison of the jet mass for different $\rho$. The value of $\rho$ is decreased in $100\;\text{GeV}$ steps to a value of $300\;\text{GeV}$ which corresponds to a radius of $0.75$ for jets with a transverse momentum of $400\;\text{GeV}$. A behaviour similar to decreasing the radius of Anti-$k_T$ jets is visible. Increasing $\rho$ returns more events where all decay products end up in the jet cone, but a jet also includes more additional particles leading to a higher mass. Since the setting $\rho = 400\;\text{GeV}$ returns the highest fraction of events with a mass around the top quark mass, it will act as representative for HOTVR in the comparison presented in section~\ref{sec:jet_comp}.

	\begin{figure}[h]
		\centering
		\includegraphics [width=.6\textwidth]{../Plots/GenStudies/HOTVR_matching}
		\caption{Jet mass distribution of jets clustered with the HOTVR algorithm. Here, default values of the clustering procedure are used. The large tail can be explained with large jet radii for jets with a transverse momentum around $400\;\text{GeV}$.}
		\label{fig:GEN_HOTVR}
	\end{figure}	
	
	\begin{figure}[tb]
		\begin{subfigure}{.5\textwidth}
	    \centering
		\includegraphics [width=\textwidth]{../Plots/GenStudies/HOTVRrho600_matching}
		\caption{}
		\label{fig:GEN_HOTVR600}
		\end{subfigure}
		\begin{subfigure}{.5\textwidth}
		\centering
		\includegraphics [width=\textwidth]{../Plots/GenStudies/HOTVRrho500_matching}
		\caption{}
		\label{fig:GEN_HOTVR500}
		\end{subfigure}
		\begin{subfigure}{.5\textwidth}
	    \centering
		\includegraphics [width=\textwidth]{../Plots/GenStudies/HOTVRrho400_matching}
		\caption{}
		\label{fig:GEN_HOTVR400}
		\end{subfigure}
		\begin{subfigure}{.5\textwidth}
		\centering
		\includegraphics [width=\textwidth]{../Plots/GenStudies/HOTVRrho300_matching}
		\caption{}
		\label{fig:GEN_HOTVR300}
		\end{subfigure}		
		\caption{Study of the influence of the HOTVR $\rho$ parameter on the jet mass. While $\rho$ is decreasing from (a) to (d) in $100\;\text{GeV}$ steps, the effective cone radius $R_\text{eff}$ decreases accordingly.}
		\label{fig:GEN_HOTVRrho}
	\end{figure}

\FloatBarrier %draw figures of previous section before the new one starts		
\subsection{Studies with XCone}
\label{sec:XCone_strat}
	The XCone jet algorithm described in section~\ref{sec:xcone} has already been tested resolving $t\bar{t}$ decays in studies for hadronically decaying top quark pairs, presented in a paper from Thaler and Wilkason \cite{xconetop}. In the called publication, the XCone algorithm is tuned to the $t\bar{t}$ final state, expecting six jets. Analogously to \cite{xconetop}, an approach to reconstruct the top quark decays with a strategy using two clustering steps is chosen in this analysis (see Fig.~\ref{fig:JetDisplay}). Firstly, XCone is required to find exactly two jets with a large radius ensuring that all decay products of the top quark end up in the jet. Thus, every particle from the top quark decay should be clustered into one of the jets. The goal of this first step is to separate the two top quarks into independent jets and suppress the influence of additional radiation. Now, the jets are identified if they contain the decay products of the hadronically decaying or leptonically decaying top quark via a distance measure to the lepton $\Delta R (\text{lepton, jet})$. To check if the categorization works, the distance $\Delta R$ between the hadronically decaying top quark on particle level and the selected jet is calculated. The distribution is shown in Fig.~\ref{fig:XCone_dR}. According to this plot, almost every jet is categorized correctly. A small fraction of events at $\Delta R = \pi$ indicates the events where the leptonically decaying top quark is reconstructed with the wrong jet. These events will be included into the unmatched fraction. The jet shape in the $\eta$-$\phi$-plane and all particles in the event are shown in Fig.~\ref{fig:JetDisplay1}. After that, the jets are further divided into smaller subjets. Since one expects only two visible components on the leptonic side, only two subjets are required, while the other jet contains three. This strategy will be referred to as '$2+5$' and is depicted in Fig.~\ref{fig:JetDisplay2}. The subjets are then combined to form a final jet that is used from now on.	
	\begin{figure}[tb]
		\centering
		\includegraphics [width=.5\textwidth]{../Plots/GenStudies/XCone_dR_GEN_R20}
		\caption{Check of categorisation of XCone jets. The distance between generated hadronically decaying top quark and jet identified as containing its decay products is shown. The distribution is expected to peak at low values if the categorisation works, which is the case.}
		\label{fig:XCone_dR}
	\end{figure} 
	\begin{figure}[tb]
		\begin{subfigure}{.5\textwidth}
	    \centering
		\includegraphics [width=\textwidth]{../Plots/JetDisplayR15/xcone_incjets_event04}
		\caption{}
		\label{fig:JetDisplay1}
		\end{subfigure}
		\begin{subfigure}{.5\textwidth}
	    \centering
		\includegraphics [width=\textwidth]{../Plots/JetDisplayR15/xcone_subjets_event04}
		\caption{}
		\label{fig:JetDisplay2}
		\end{subfigure}
		\caption{Display of the jet area from XCone jets clustered with the '$2+5$' approach after the first (a) and the second step (b) . The grey dots show particles identified by the PF algorithm, red circles indicate decay products from the hadronically decaying top quark where the bottom quark is additionally marked. Decay products from the leptonically decaying top quark are marked as follows: the black circle with star illustrates the bottom quark, the black circle with plus sign shows the lepton and the white circle marks the neutrino.}
		\label{fig:JetDisplay}
	\end{figure}
	Finally, the cone sizes have to be chosen. Here, $R_1$ denotes the radius parameter of the first clustering step, while $R_2$ measures the radii of the subjets. The subjets are set to a radius of $R_2=0.4$ to be comparable to the CMS standard for subjets, which are AK4 jets. Studies of the jet mass are made to determine the most suitable radius for the first clustering step. Figure~\ref{fig:JetDisplayR} shows an effect of larger cone sizes in the first clustering step. While the large smaller jet in Fig.~\ref{fig:JetDisplayR1} identifies the top quark decay products correctly, the large $R_1$ of jets in Fig.~\ref{fig:JetDisplayR2} include additional particles and misidentify them. Since all decay products of the top quark decay are yet included into the other two subjets, the jet mass will be reconstructed too high. On the other hand, a small cone may not include all decay products. This effect can also be seen in the jet mass distributions in Fig.~\ref{fig:XConeR1}. Based on these studies, the radius parameter is chosen to be $R=1.2$ because of less events in the shoulder around the $W$ mass and a smaller tail.\\
	To perform the clustering with data, an easier method is tested, where both fat jets contain three subjets, referred to as '$2+6$'. This approach is necessary because the lepton used to identify the leptonic jet is not fully defined at the stage where jet clustering is performed. The final jets are analogously categorized after a proper lepton identification into a jet originating from the hadronically and the leptonically decaying top quark. A comparison between the two methods (see Fig.~\ref{fig:GEN_XCone_comp}) shows that both return almost the same distribution. Thus, the '$2+6$' method is chosen to represent the XCone result.
	
	\begin{figure}[tb]
		\begin{subfigure}{.5\textwidth}
	    \centering
		\includegraphics [width=\textwidth]{../Plots/JetDisplayR10/xcone_subjets_event09}
		\caption{}
		\label{fig:JetDisplayR1}
		\end{subfigure}
		\begin{subfigure}{.5\textwidth}
	    \centering
		\includegraphics [width=\textwidth]{../Plots/JetDisplayR20/xcone_subjets_event09}
		\caption{}
		\label{fig:JetDisplayR2}
		\end{subfigure}
		\caption{Jet area comparison between a small (a) and a large $R_1$ (b). The small cone only barely contains all decay products while the larger cone leads to misidentification of particles not belonging to the top quark decay.}
		\label{fig:JetDisplayR}
	\end{figure}	
		
	\begin{figure}[tb]
		\begin{subfigure}{.5\textwidth}
  		\centering
		\includegraphics [width=\textwidth]{../Plots/GenStudies/XCone_GEN_R10}
		\caption{}
		\end{subfigure}
		\begin{subfigure}{.5\textwidth}
  		\centering
		\includegraphics [width=\textwidth]{../Plots/GenStudies/XCone_GEN_R12}
		\caption{}
		\end{subfigure}
		\begin{subfigure}{.5\textwidth}
  		\centering
		\includegraphics [width=\textwidth]{../Plots/GenStudies/XCone_GEN_R15}
		\caption{}
		\end{subfigure}
		\begin{subfigure}{.5\textwidth}
  		\centering
		\includegraphics [width=\textwidth]{../Plots/GenStudies/XCone_GEN_R20}
		\caption{}
		\end{subfigure}
						
		\caption{Jet mass distributions for different $R_1$. The smaller $R_1$ is, the more jets are reconstructed at the $W$ mass.}
		\label{fig:XConeR1}
	\end{figure}	
	
 	\begin{figure}[tb]
 		\begin{subfigure}{.5\textwidth}
  		\centering
 		\includegraphics [width=\textwidth]{../Plots/GenStudies/XCone23_matching}
 		\label{fig:GEN_XCone23}
 		\caption{}
 		\end{subfigure}
 		\begin{subfigure}{.5\textwidth}
  		\centering
 		\includegraphics [width=\textwidth]{../Plots/GenStudies/XCone33_matching}
 		\label{fig:GEN_XCone33}
 		\caption{}
 		\end{subfigure}
 		\caption{Comparison of the jet mass distribution of XCone jets clustered with the '$2+5$' (a) and the '$2+6$' (b) method. Both graphs show a very similar shape.}
 		\label{fig:GEN_XCone_comp}
 	\end{figure}
 	
\FloatBarrier %draw figures of previous section before the new one starts 	
\subsection{Comparing Jet Algorithms}
\label{sec:jet_comp}
	In this section, the resulting jet mass distributions of different jet clustering algorithms discussed above are compared. Figure~\ref{fig:Jet_Comp} shows all jet mass distributions examined above. All three clustering methods return a good resolution in the peak region while the maximum sits at the top quark mass. A more detailed look, how the jet algorithms distribute over different jet mass regions is presented in Tab. \ref{jet_numbers}. The Anti-$k_T$ algorithm with a cone radius of $R=0.8$ and soft drop applied reconstructs many events with a low mass, with the pronounced peak at the $W$ boson mass. This effect is explained by the small cone size. If only the decay products of the $W$ boson, but not the $b$ quark is clustered, a mass around the $W$ boson mass is expected. With a variable sized cone in HOTVR, this effect is reduced but more events with a mass larger than $200\;\text{GeV}$ are observed. XCone on the other hand combines the good aspects of Anti-$k_T$ and HOTVR. It returns a jet mass distribution with a narrow peak at the top quark mass and a low fraction of masses reconstructed too low or high. Note, that the performed matching is much more stringent for XCone jets. While the top quark decay products only have to be merged into the large jet for AK and HOTVR, a matching to the subjets is performed with XCone. However, XCone still returns a similar matched fraction than the AK and HOTVR jet algorithms.  Furthermore, much more jets are reconstructed correctly, therefore survive the selection criteria and lead to much higher event yield. This is explained by a softer selection of jets since an exclusive jet clustering algorithm will always return exactly two jets and a veto on additional jets is redundant. Because of the advantages of XCone, namely good resolution, peak position sensitive to the top quark mass and very high statistics, the measurement will be performed using XCone jets clustered with the '$2+6$' method.
	
 	\begin{figure}[tb]
 		\begin{subfigure}{.5\textwidth}
  		\centering
 		\includegraphics [width=\textwidth]{../Plots/GenStudies/AK08softdrop_matching}
 		\label{fig:Jet_Comp_ak}
 		\caption{}
 		\end{subfigure}
 		\begin{subfigure}{.5\textwidth}
  		\centering
 		\includegraphics [width=\textwidth]{../Plots/GenStudies/HOTVRrho400_matching}
 		\label{fig:Jet_Comp_HOTVR}
 		\caption{}
 		\end{subfigure}
 		\begin{subfigure}{.5\textwidth}
  		\centering
 		\includegraphics [width=\textwidth]{../Plots/GenStudies/XCone33_matching}
 		\label{fig:Jet_Comp_XCone}
 		\caption{}
 		\end{subfigure} 		
 		\caption{Jet mass distributions of jets clustered with Anti-$k_T$ (a), HOTVR (b) and XCone (c). While all distributions show a good resolution in the peak region, AK shows a large fraction of masses reconstructed too low and XCone returns the largest statistics.}
 		\label{fig:Jet_Comp}
 	\end{figure}	
	
	\begin{table}
	\centering
	\begin{tabular}{l c c c }
	 & AK8 & HOTVR & XCone \\
	\hline
	\hline	 
	above $200\;\text{GeV}$ & $8.0\%$ & $23.6\%$ & $19.0\%$ \\ 
	below $150\;\text{GeV}$ & $31.0\%$ & $17.8\%$ & $17.7\%$ \\ 	
	$150\;\text{GeV} < M_\text{jet} < 200\;\text{GeV}$  & $61.0\%$ & $58.6\%$ & $63.3\%$ \\ 	
	not matched             & $36.3\%$ & $32.7\%$ & $35.5\%$ \\ 	
	\end{tabular}
	\caption{Overview of event yield fractions in different jet mass regions for the three compared jet clustering algorithms.}
	\label{jet_numbers}	
	\end{table}
	
\section{Studies on Reconstruction Level}
\label{sec:selection}
	Since the measurement should be performed with data, a similar jet mass distribution has to be achieved on reconstruction level. A selection is applied to simulation and data to obtain a data sample consisting of mostly $t\bar{t}$ events in the lepton+jets channel. The selection can be divided into two steps. Firstly, a baseline selection is used to suppress background processes, section~\ref{sec:PreSel}. Secondly, the final phase space is defined in section~\ref{sec:FinalSel} to select $t\bar{t}$ events with boosted top quarks. This is crucial for this analysis because the goal is to reconstruct the top quark with one jet. This can only be done if all of its decay products merge into one jet.

\subsection{Baseline Selection}
\label{sec:PreSel}
	In the lepton+jets channel of the $t\bar{t}$ process one expects to find exactly one muon or electron, two small jets from the hadronically decaying $W$ boson, two b-jets and missing transverse energy since the neutrino cannot be detected. This baseline selection is designed to remove non-$t\bar{t}$ events. Since this analysis focuses on the muon channel, the selection reads:
	\begin{itemize}
	\item single muon trigger (combination of "HLT\_Mu50\_v*" and "HLT\_TkMu50\_v*") with $p_T > 50\;\text{GeV}$ threshold,
	\item exactly one tight muon with $p_T > 55\;\text{GeV}$ and $|\eta| < 2.4$,
	\item veto on additional leptons,
	\item two-dimensional muon isolation criterion: \\ $\Delta R(\text{lepton, next AK4 jet}) > 0.4$ or $p_T^{\text{rel}}(\text{lepton, next AK4 jet}) > 40\;\text{GeV}$ \footnote{$p_T^{\text{rel}}(a,b) = \frac{|\vec{p_a} \times \vec{p_b}|}{|\vec{p_b}|}$},
	\item at least two AK4 jets with $p_T > 50\;\text{GeV}$ and $|\eta| < 2.4$,
	\item $\cancel{E}_T > 50\;\text{GeV}$ and
	\item at least one tight b-tag.
	\end{itemize}
	Because the dataset corresponding to the single muon trigger is used in this analysis, the trigger criteria have to be fulfilled in simulation as well. An additional cut on the muon $p_T$ above $53\;\text{GeV}$ is recommended \cite{MuonID} for this trigger to reach the plateau of trigger efficiency. Furthermore, a scale factor is applied to simulation to account for efficiency differences in data and simulation. Since only one lepton is expected in the lepton+jets channel of $t\bar{t}$, a veto on additional leptons is used to suppresses diboson events. Because no isolation criterium is applied for the muons, a two-dimensional cut is applied to reject QCD events. A window in the $\Delta R$-$p_T^{\text{rel}}$-plane is cut out where the majority of QCD that survive the lepton criteria accumulates. A display of the two-dimensional cut is presented in Fig.~\ref{fig:2D}. \\
	The requirements to find two AK4 jets with at least $50\;\text{GeV}$, missing transverse energy of at least $50\;\text{GeV}$ and a b-tag to preferably select $t\bar{t}$ events. For b-tagging a scale factor is applied to match efficiency in data and simulation. After applying the selection the remaining events contain about $80\%$ $t\bar{t}$, the main remaining backgrounds are $W+$jets and Single-Top production. 
 	\begin{figure}[tb]
 		\begin{subfigure}{.5\textwidth}
  		\centering
 		\includegraphics [width=\textwidth]{../Plots/TwoD_QCD}
 		\caption{}
 		\end{subfigure}
 		\begin{subfigure}{.5\textwidth}
  		\centering
 		\includegraphics [width=\textwidth]{../Plots/TwoD_TTbar}
 		\caption{}
 		\end{subfigure}
 		\caption{Distribution in the $\Delta R$-$p_T^{\text{rel}}$-plane for QCD (a) and $t\bar{t}$ events (b). Both distributions are normalised to unity. The window affected by the 2D cut is surrounded by red lines.}
 		\label{fig:2D}
 	\end{figure}
 	After applying the baseline selection, a difference in total number of events between simulation and data  as well as a $p_T$ dependent trend is visible (see Fig.~\ref{fig:PreSeljet}), probably coming from a mismodelled top quark $p_T$ distribution. The difference in $p_T$ spectra of the top quark has been observed in several publications, for example in a $t\bar{t}$ differential cross section measurement from CMS \cite{ttreweight}. An observed ratio between data and simulation in shown in Fig.~\ref{fig:topreweight}. To test this assumption a re-weighting of the top quark $p_T$ spectrum in $t\bar{t}$ simulation is applied according to Fig.~\ref{fig:topreweight}. The resulting histograms are presented in Fig.~\ref{fig:PreSel_reweight}. After this procedure, MC and data are well in agreement except for the AK4 jet $p_T$ in high energy regions. This is expected to not be well corrected by the re-weighting since it covers the range from $400\;\text{GeV}$ upwards with only one constant value. This effect is also visible in the measurement phase space discussed below and is one of the simulation properties that could be validated with this analysis.
 	 		
 	\begin{figure}[tb]
  		\centering
 		\includegraphics [width=.65\textwidth, trim=0 0 0 0, clip]{../Plots/../Plots/PreSel/08_bTag_jets/pt_jet_log.pdf}
 		\caption{Transverse momenta of all AK4 jets in the event. A trend in the ratio between data and simulation is observed.}
 		\label{fig:PreSeljet}
 	\end{figure}
 	
  	\begin{figure}[tb]
   		\centering
  		\includegraphics [width=.5\textwidth]{../Plots/top_reweight}
  		\caption{Ratio between data and simulation in differential top quark pair cross section in dependence on the transverse momentum of top quarks. Taken from \cite{topreweight}.}
  		\label{fig:topreweight}
  	\end{figure}	
 	
 	\begin{figure}[tb]
 		\centering
 		\begin{subfigure}{.45\textwidth}
  		\centering
		\includegraphics [width=\textwidth, trim=0 0 5.5cm 0, clip]{../Plots/PreSel_legend/ttbar_reweight_Muon/number_lin.pdf}
 		\caption{}
 		\end{subfigure}
 		\begin{subfigure}{.45\textwidth}
  		\centering
 		\includegraphics [width=\textwidth, trim=0 0 5.5cm 0, clip]{../Plots/PreSel/ttbar_reweight_Muon/pt_1_log.pdf}
 		\caption{}
 		\end{subfigure} 		
 		\begin{subfigure}{.45\textwidth}
  		\centering
 		\includegraphics [width=\textwidth, trim=0 0 5.5cm 0, clip]{../Plots/PreSel_legend/ttbar_reweight_Jets/number_lin.pdf}
 		\caption{}
 		\end{subfigure}
 		\begin{subfigure}{.45\textwidth}
  		\centering
 		\includegraphics [width=\textwidth, trim=0 0 5.5cm 0, clip]{../Plots/PreSel/ttbar_reweight_Jets/pt_jet_log.pdf}
 		\caption{}
 		\end{subfigure}
		\begin{subfigure}{.45\textwidth}
  		\centering
 		\includegraphics [width=\textwidth, trim=0 0 5.5cm 0, clip]{../Plots/PreSel_legend/ttbar_reweight_Event/BTAG_T_lin.pdf}
 		\caption{}
 		\end{subfigure} 	
  		\begin{subfigure}{.45\textwidth}
   		\centering
  		\includegraphics [width=\textwidth, trim=0 0 5.5cm 0, clip]{../Plots/PreSel/ttbar_reweight_Event/MET_log.pdf}
  		\caption{}
  		\end{subfigure}  	 			
 		\caption{Control distributions after applying a reweighting of the top quark $p_T$ in $t\bar{t}$ simulation. Displayed are number of muons (a), $p_T$ spectrum of muons (b), number of AK4 jets (c), $p_T$ distribution of all AK4 jets (d), number of b-tagged AK4 jets (e) and the spectrum of the missing transverse energy (f). }
 		\label{fig:PreSel_reweight}
 	\end{figure}	

\FloatBarrier %draw figures of previous section before the new one starts
\subsection{Measurement Phase Space}
\label{sec:FinalSel}
	The measurement phase space on reconstruction level is defined analogously to the particle level selection. Boosted topologies are selected by requiring the leading jet to surpass a cut on $p_T > 400\;\text{GeV}$. In addition, the mass of the leading jet is expected to be higher than the second jet mass if all top quark decay products are reconstructed correctly. Thus, on top of the baseline selection presented above, following criteria are checked:
	\begin{itemize}
	\item subjets $p_T > 30\;\text{GeV}$,
	\item $p_T^{\text{1st jet}} > 400\;\text{GeV}$ and
	\item $M^{\text{1st jet}} > M^{\text{2nd jet + lepton}}$.
	\end{itemize}
	In this phase space, the selection criteria refer to XCone jets clustered with the '$2+6$' method.	In addition to the definition on particle level, only subjets with a $p_T$ larger than $30\;\text{GeV}$ are considered to form the final jet. This requirement is set so suppress subjets which contain mostly objects from pile-up and not any decay product of the top quark. Figure~\ref{fig:MJet_raw1} shows the jet mass distribution after these jet requirements. Similar as on particle level, XCone returns a peak at the expected bin of the top quark mass. Thus, the reconstruction with XCone jets works very well. As seen after the baseline selection, simulation exceeds the number of events in data. In Fig.~\ref{fig:MJet_raw2}, $t\bar{t}$ simulation is scaled with a constant factor to match the total events in data and simulation. A good agreement between data and MC is visible. Of course, further procedures like unfolding will use the unscaled version as input. In addition, another variant of jet clustering with XCone is presented in Fig.~\ref{fig:MJet_raw3}. Here, the large XCone jet with radius $R=1.2$ is put into the soft drop algorithm. Albeit the grooming of soft drop, the '2+6' method returns a much sharper mass peak. While the turn on of the soft drop mass is comparable with '2+6', there are many jets with a mass reconstructed too high due to large contributions from pile-up.
 	\begin{figure}[tb]
 		\begin{subfigure}{.5\textwidth}
  		\centering
 		\includegraphics [width=\textwidth, trim=0 0 3cm 0, clip]{../Plots/PostSel/XCone_raw/M_jet1__lin.pdf}
 		\caption{}
 		\label{fig:MJet_raw1}
 		\end{subfigure}
 		\begin{subfigure}{.5\textwidth}
 		\centering
		\begin{tikzpicture}
		 \node[anchor=south west,inner sep=0] (image) at (0,0)
		 {\includegraphics[width=\textwidth, trim=0 0 3cm 0, clip]{../Plots/PostSel/XCone_raw_SF/M_jet1__lin.pdf}};
		 \node[align=left,font=\tiny] at (2.2, 4.2) {$t\bar{t}$ scaled};
		\end{tikzpicture} 
 		\caption{}
 		\label{fig:MJet_raw2}
 		\end{subfigure}
 		\begin{subfigure}{.5\textwidth}
 		\centering
		\begin{tikzpicture}
		 \node[anchor=south west,inner sep=0] (image) at (0,0)
		 {\includegraphics[width=\textwidth, trim=0 0 3cm 0, clip]{../Plots/PostSel/XCone_raw_SF/SoftdropMass_had_lin.pdf}};
		 \node[align=left,font=\tiny] at (2.2, 4.2) {$t\bar{t}$ scaled};
		\end{tikzpicture} 
 		\caption{}
 		\label{fig:MJet_raw3}
 		\end{subfigure} 		
 		\caption{Jet mass distribution of XCone jets after applying the measurement phase space requirements. Additional to the raw output (a) a distribution where $t\bar{t}$ simulation is scaled to data is presented (b). Histogram (c) shows the soft drop mass of the large XCone jet ($R=1.2$) in comparison. It shows that a grooming via XCone subjet finding is much more effective than soft drop in this particular case.}
 		\label{fig:MJet_raw}
 	\end{figure}
	
\FloatBarrier %draw figures of previous section before the new one starts	
\subsection{Jet Energy Corrections for XCone Jets} 
	The normal procedure in CMS analyses is to apply jet energy corrections (see section~\ref{sec:jec}) to every jet collection used. Those jet energy corrections (JEC) have been derived by dedicated CMS groups and are dependent on the jet algorithm used to cluster jets. Since the XCone algorithm is not a standard jet finding procedure in CMS, there are no valid corrections available. Because the final jet measured in this analysis is a combination of subjets, only the subjets are corrected. A XCone jet with applied jet energy corrections refers then to a jet put together from corrected subjets. The first attempt to correct XCone jets is to use the AK4 jet corrections, since the jet shape should be very similar to XCone subjets with $R=0.4$ as they were used in this analysis. Figure~\ref{fig:MJet_jec} shows a comparison between jets with and without JEC applied. 
 	\begin{figure}[tb]
 		\begin{subfigure}{.5\textwidth}
 		\centering
		\begin{tikzpicture}
		 \node[anchor=south west,inner sep=0] (image) at (0,0)
		 {\includegraphics[width=\textwidth, trim=0.5cm 0 4cm 0, clip]{../Plots/PostSel/XCone_raw_SF/M_jet1__lin.pdf}};
		 \node[align=left,font=\tiny] at (2.2, 4.2) {$t\bar{t}$ scaled};
		\end{tikzpicture} 
 		\caption{}
 		\label{fig:MJet_jec1}
 		\end{subfigure}
 		\begin{subfigure}{.5\textwidth}
 		\centering
		\begin{tikzpicture}
		 \node[anchor=south west,inner sep=0] (image) at (0,0)
		 {\includegraphics[width=\textwidth, trim=0.5cm 0 4cm 0, clip]{../Plots/PostSel/XCone_jec_SF/M_jet1__lin.pdf}};
		 \node[align=left,font=\tiny] at (2.2, 4.2) {$t\bar{t}$ scaled};
		\end{tikzpicture} 
 		\caption{}
 		\label{fig:MJet_jec2}
 		\end{subfigure}
 		\caption{Jet mass distribution of XCone jets before (a) and after applying AK4 jet energy corrections (b). A shift in jet masses to higher values is visible.}
 		\label{fig:MJet_jec}
 	\end{figure}	
 	An obvious shift in jet mass is visible, resulting in a peak position above the top quark mass. Fitting a Gaussian to the peak, one is able to obtain the peak position. With the peak at $172.2\;\text{GeV}$ with raw jets and a peak at $182.8\;\text{GeV}$ with corrected jets, the mass is reconstructed about $10\;\text{GeV}$ above the top quark mass when applying AK4 corrections. Since the data-simulation agreement is well in both cases and one can not optimize on the top quark mass in this analysis, different measures have to be defined to validate the jet energy correction. An attempt is made to test the subjets energy by a well known mass scale. Therefore, all three jet mass combinations $M_{ij}$ of two of the three subjets on the hadronic side are calculated. It is expected that the combination with the lowest jet mass should be proportional to the $W$ mass which is very precisely measured to $80.4\;\text{GeV}$ \cite{Wmass}. As shown in Fig.~\ref{fig:Wmass} the jet mass peak is reconstructed too high. Fitting with a Gaussian in the peak region returns values of $79.5\;\text{GeV}$ and $84.3\;\text{GeV}$ for raw and jets with AK4 correction, respectively. Since the raw subjets in Fig.~\ref{fig:Wmass1} show a agreement with the measured value of the $W$ boson mass, applying jet energy correction from AK4 jets to XCone subjets is not valid.
  	\begin{figure}[tb]
  		\begin{subfigure}{.5\textwidth}
  		\centering
 		\begin{tikzpicture}
 		 \node[anchor=south west,inner sep=0] (image) at (0,0)
 		 {\includegraphics[width=\textwidth, trim=0 0 3cm 0, clip]{../Plots/PostSel/XCone_raw_subjets_SF/min_mass_Wjet_zoom_lin.pdf}};
 		 \node[align=left,font=\tiny] at (2.2, 4.2) {$t\bar{t}$ scaled};
 		\end{tikzpicture} 
  		\caption{}
  		\label{fig:Wmass1}
  		\end{subfigure}
  		\begin{subfigure}{.5\textwidth}
  		\centering
 		\begin{tikzpicture}
 		 \node[anchor=south west,inner sep=0] (image) at (0,0)
 		 {\includegraphics[width=\textwidth, trim=0 0 3cm 0, clip]{../Plots/PostSel/XCone_jec_subjets_SF/min_mass_Wjet_zoom_lin.pdf}};
 		 \node[align=left,font=\tiny] at (2.2, 4.2) {$t\bar{t}$ scaled};
 		\end{tikzpicture} 
  		\caption{}
  		\label{fig:Wmass2}
  		\end{subfigure}
  		\caption{Lowest jet mass combination $\min(M_{ij})$ without (a) and with (b) JEC applied. The position of the peak is expected to match the $W$ boson mass of $80.4\;\text{GeV}$. The shift after applying JEC puts $\min(M_{ij})$ to higher values.} 
  		\label{fig:Wmass}
  	\end{figure}	
	In order to be able to correct XCone jets for response non linearities and pile-up effects, a correction factor on top of AK4 corrections is derived for XCone jets. For this, only events from $t\bar{t}$ simulation are used. Furthermore, only the subjets from the jet belonging to the hadronically decaying top quark are considered. A matching to generator jets is executed and the fraction $r=\frac{p_T^{\text{rec}}}{p_T^{\text{gen}}}$ calculated. This is done in different $p_T$ and $\eta$ regions. The mean $r$ in an $\eta$ region is then filled in a two dimensional histogram representing the $p_T$-$\eta$-plane (see Fig.~\ref{fig:Correction}). The bin boundaries are chosen to obtain enough statistics in each bin to suppress uncertainties (validated by the RMS\footnote{root mean square} width, see appendix Fig.~\ref{fig:A_rms}). 
		\begin{figure}[tb]
			\centering
			\includegraphics [width=.7\textwidth]{../Plots/Correction/Mean_numbers}
			\caption{Mean values of $r=\frac{p_T^{\text{rec}}}{p_T^{\text{gen}}}$ in the $p_T$-$\eta$ plane.}
			\label{fig:Correction}
		\end{figure}	
	The correction factor applied to every XCone jet is now $f = \frac{1}{r}$. In every $\eta$ bin a polynomial function is fitted to get a factor $f(p_T)$ to get a smooth transition between the different regions. An example of the fit is shown in Fig.~\ref{fig:Correction_fit} (all fit functions can be found in the appendix in Fig.~\ref{fig:A_fits}).
	\begin{figure}[tb]
		\centering
		\includegraphics [width=.5\textwidth]{../Plots/Correction/Fits_example}
		\caption{Example of fit function. The correction factors are derived from Fig.~\ref{fig:Correction} and then fitted with a polynomial function of degree $2$.}
		\label{fig:Correction_fit}
	\end{figure}
	Now every subjet is corrected with a different $p_T$ dependent function according to its direction in $\eta$. Note that because of the loose ends of the fit, the correction factor is set constant for $p_T > 425\;\text{GeV}$. To verify this correction, the minimum jet mass from two subjets is again compared with the $W$ boson mass in Fig.~\ref{fig:Wmass_cor}. The peak position is again obtained with a Gaussian and is with its $80.6\;\text{GeV}$ in agreement with the $W$ boson mass of $M_W = 80.4\;\text{GeV}$.
  	\begin{figure}[tb]
  		\centering
 		\begin{tikzpicture}
 		 \node[anchor=south west,inner sep=0] (image) at (0,0)
 		 {\includegraphics[width=.7\textwidth, trim=0 0 3cm 0, clip]{../Plots/PostSel/XCone_cor_subjets_SF/min_mass_Wjet_zoom_lin.pdf}};
 		 \node[align=left,font=\small] at (3.3, 6.5) {$t\bar{t}$ scaled};
 		\end{tikzpicture} 
   		\caption{Lowest jet mass combination $\min(M_{ij})$ after AK4 jet energy correction and the correction for XCone. The peak position is in agreement with the value of the $W$ boson mass of $80.4\;\text{GeV}$.} 
  		\label{fig:Wmass_cor}
  	\end{figure}	  
  	Furthermore, the relative $p_T$ deviation between reconstruction level jets and particle level jets is calculated. A graph showing the mean and width (here calculated as RMS) of $\frac{p_T^\text{rec} - p_T^\text{gen}}{p_T^\text{gen}}$ in bins of $p_T^\text{rec}$ is presented in Fig.~\ref{fig:Reso}. While the width of the distribution (Fig.~\ref{fig:Reso2}) is not much influenced by the different jet energy correction stages, the mean shows a large improvement after the additional correction. The raw XCone jets and the ones corrected with AK4 corrections show a dependence on $p_T$ where the corrected jets are flat in this variable. Additionally, the corrected jets show a constant mean around zero which indicates a well performing correction. Figure~\ref{fig:Reso3} shows the mass resolution of the final jet. All but the first bin show a resolution below $10\%$, indicating a good reconstruction. The first bin shows a large value because of insufficient statistics for masses below $100\;\text{GeV}$. 
  	
  	\begin{figure}[tb]
  		\centering
  		\begin{subfigure}{.6\textwidth}
   		\centering
  		\includegraphics [width=\textwidth]{../Plots/Resolution_Subjets/pt_mean_rec_after}
  		\caption{}
  		\label{fig:Reso1}
  		\end{subfigure}
  		\begin{subfigure}{.6\textwidth}
   		\centering
  		\includegraphics [width=\textwidth]{../Plots/Resolution_Subjets/pt_rms_rec_after}
  		\caption{}
  		\label{fig:Reso2}
  		\end{subfigure}
  		\begin{subfigure}{.6\textwidth}
   		\centering
  		\includegraphics [width=\textwidth]{../Plots/Resolution_Mass/mass_rms_massbin.pdf}
 		\caption{}
 		\label{fig:Reso3}
 		\end{subfigure}
  		\caption{Mean (a) and width (b) of the relative deviation in jet $p_T$ between particle and reconstruction level. The mean stays almost constant after applying the customised correction. Furthermore, the width is rather independent on the correction. Figure (c) shows the width of the relative deviation in mass. Despite the first bin, the resolution lies below $0.1$.}
  		\label{fig:Reso}
  	\end{figure}
  	
  	
\FloatBarrier %draw figures of previous section before the new one starts	
\subsection{Final Jet Mass Distribution}
	After the events passed the baseline selection, the measurement phase space requirement and the newly derived jet energy correction, the final jet mass distribution is presented in Fig.~\ref{fig:MJet_final}. Data and Monte Carlo prediction agree very well after scaling $t\bar{t}$ simulation with a normalisation factor of $0.75$. The dependence on pile-up is studied in Fig.~\ref{fig:Mass_PU}. The orange band, consisting the majority of events, shows an almost flat distribution over the number of primary vertices per event. A slight trend, that high pile-up leads to higher masses is observed. However, this effect is not dominant since the red line, indicating a top quark mass of $173\;\text{GeV}$ is still in agreement with the region where jets accumulate. A jet mass distribution from the $8\;\text{TeV}$ analysis is shown in Fig.~\ref{fig:Torben_MJet} to compare both results. Cambridge/Aachen jets with a radius parameter of $1.2$ are used, selecting events with one jet with $p_T > 500\;\text{GeV}$. A drastic improvement in peak resolution and statistics is visible in the XCone distribution, promising higher statistics, a finer binning and less influence from pile-up when unfolding the data in the $13\;\text{TeV}$ analysis
	\begin{figure}[h]
  		\centering
 		\begin{tikzpicture}
 		 \node[anchor=south west,inner sep=0] (image) at (0,0)
 		 {\includegraphics[width=.8\textwidth, trim=0 0 3cm 0, clip]{../Plots/PostSel/XCone_cor_SF/M_jet1__lin.pdf}};
 		 \node[align=left,font=\small] at (3.5, 7.0) {$t\bar{t}$ scaled};
 		\end{tikzpicture} 
  		\caption{Jet mass distribution of the corrected jets clustered with XCone. This distribution will be used as input in the unfolding.} 
  		\label{fig:MJet_final}
  	\end{figure}
  		
	\begin{figure}[tb]
		\centering
		\includegraphics [width=.8\textwidth]{../Plots/Mass_PileUp_2D.pdf}
		\caption{Dependence of the jet mass on the number of primary vertices in $t\bar{t}$ simulation. A slight trend is observed that higher pile-up generates higher jet masses. Every column is normalized to unity. The red line is drawn at the top quark mass of $173\;\text{GeV}$ as a reference.}
		\label{fig:Mass_PU}
	\end{figure}
	  	
	\begin{figure}[tb]
		\centering
		\includegraphics [width=.6\textwidth]{../Plots/Torben/Torben_data_paper.pdf}
		\caption{Jet mass distribution from a similar analysis at $8\;\text{TeV}$. Besides the different dataset, Cambridge/Aachen jets with a radius parameter of $1.2$ are used. In this case, only jets with $p_T > 500\;\text{GeV}$ are shown. Taken from \cite{torben_paper}.}
		\label{fig:Torben_MJet}
	\end{figure}

	
\section{Unfolding}
\label{sec:unfolding}
	Most analyses at the LHC measure distributions in appropriate variables and then compare the obtained results in data with simulations. In this method the MC samples also include detector effects. What one measures in this case is the real distribution on particle level folded with a detector response simulated in detail for each event. Studying the difference in MC between particle level and reconstruction level, it is possible to calculate the probabilities that a measured value in a bin $y_i$ is originating from bin $x_i$ on particle level. A visualisation of this problem is drawn in Fig.~\ref{fig:Unfolding}.	
	\begin{figure}[tb]
		\centering
		\includegraphics [width=.6\textwidth]{../Images/Unfolding.png}
		\caption{Schematic view of an unfolding procedure. The goal is to unfold a measured distribution $\mathbf{y}$ to obtain a true distribution $\mathbf{x}$ without detector effects. Taken from \cite{tunfold}.}
		\label{fig:Unfolding}
	\end{figure}
	The matrix $\mathbf{A}$ describing migrations from bins in $\mathbf{x}$ to bins in $\mathbf{y}$ can be calculated using simulations where both distributions $\mathbf{x}$ and $\mathbf{y}$ are known. Then, the inverted migration matrix can be used to obtain an estimate for data on particle level. The following equation denotes the basic problem one has to solve
	\begin{equation}
	\tilde{y}_i = \sum_{j=1}^{m} A_{ij}\tilde{x}_j, 1 \leq i \leq n,
	\label{eq:unfold}
	\end{equation}
	where $m$ and $n$ are the number of bins of the true and measured distributions, respectively. The tilde marks the statistical mean of $\mathbf{x}$ and $\mathbf{y}$. Here, one is interested in a distribution $x_j$ instead of $\tilde{x}_j$ making this problem non-trivial. If one only replaces $\tilde{y}_i \rightarrow y_i$ and $\tilde{x}_j \rightarrow x_j$, and solves for $x_j$ by inverting the matrix $\mathbf{A}$, statistical fluctuations of $\mathbf{y}$ would be amplified. Thus, fluctuations have to be damped with a regularisation.
	
\subsection{Regularised Unfolding with TUnfold}
	The TUnfold software package \cite{tunfold} provides a framework for regularised unfolding procedures in high energy physics. The Lagrangian implemented in TUnfold that is minimised is given by
	\begin{eqnarray}
	\label{eq:unfold_lagrange}
	\mathcal{L}(x,\lambda) &=& \mathcal{L}_1 + \mathcal{L}_2 + \mathcal{L}_3 
	\\ \nonumber \text{with}
	\\ 
	\label{eq:unfold_lagrange1}
	\mathcal{L}_1 &=& (\mathbf{y} - \mathbf{Ax})^\intercal \mathbf{V_{yy}}^{-1} (\mathbf{y} - \mathbf{Ax}),
	\\
	\label{eq:unfold_lagrange2}
	\mathcal{L}_2 &=& \tau^2 (\mathbf{x} - f_b \mathbf{x}_0)^\intercal (\mathbf{L}^\intercal \mathbf{L}) (\mathbf{x} - f_b \mathbf{x}_0),
	\\
	\label{eq:unfold_lagrange3}
	\mathcal{L}_3 &=& \lambda (Y-\mathbf{e}^\intercal \mathbf{x}) \ \text{with} \ Y=\sum_{i} y_i \ \text{and} \ e_j = \sum_{i}A_{ij}.
	\end{eqnarray}
	The first term $\mathcal{L}_1$ contains a standard least square minimisation where $\mathbf{V_{yy}}$ is the covariance matrix describing uncertainties. Secondly a regularisation with strength $\tau^2$ is used. A bias vector can be introduced using a factor $f_b$ and a vector $\mathbf{x}_0$ to suppress deviations of $\mathbf{x}$ from $f_b\mathbf{x}_0$. Additionally, three choices for the matrix $\mathbf{L}$ can be made to either regularise the absolute value, first or second derivative of $\mathbf{x}$. The final term $\mathcal{L}_3$ expresses an optional area constraint, checking differences in event counts between input and output. Since the free parameter $\tau$ defines the regularisation strength, a suitable value has to be found. This is achieved by an L-Curve scan, which finds the point of largest curvature in a graph. Two variables $L_x$ and $L_y$ are defined as follows:
	\begin{eqnarray}
	L_x  &=& \log \left(  \mathcal{L}_1 \right), \\
	L_y  &=& \log \left(  \frac{\mathcal{L}_2}{\tau^2} \right).
	\end{eqnarray}
	Now, a graph is constructed in the $L_x$-$L_y$-plane for many values of $\tau$. The point of largest curvature then indicates the point, where $\mathcal{L}_1$ and $\mathcal{L}_2$ equally influence the unfolding. The value of $\tau$ then corresponds to this point. 
	
\subsection{Migration Matrix}
\label{sec:migrations}
	To perform an unfolding, a migration matrix has to be filled. A matrix entry $A_{ij}$ contains the number of events that is generated in bin $i$ and reconstructed in bin $j$. Thus, a projection on one dimension returns either the jet mass distribution on particle or reconstruction level. Only events from the $t\bar{t}$ samples are considered to construct the matrix. If an event passes the particle level selection (see section~\ref{sec:GenSel}) and reconstruction level selection (see sections~\ref{sec:PreSel} and \ref{sec:FinalSel}) it is filled in the according bin. To account for events that pass only one selection, underflow bins are used, while jet masses above $400\;\text{GeV}$ are filled into overflow bins. Furthermore care has to be taken due to different weights because corrections like trigger and b-tagging scale factors as well as pile-up re-weighting are only applied on reconstruction level. Thus, the weight applied to an event is split into a generator level weight and a reconstruction level weight,
	\begin{equation}
	w_\text{event} = w_\text{gen} \cdot w_\text{reco}.
	\label{eq:weight}
	\end{equation}
	Since events without reconstruction level information only carry a generator weight, events are filled into underflow bins with a compensation weight to match the total number of events.
	\begin{equation}
	w_\text{comp} = w_\text{gen} \cdot (1 - w_\text{reco}).
	\label{eq:weight2}
	\end{equation}		
	Figure~\ref{fig:Migration} shows the migration matrix where every bin is scaled to the total event count in the whole column, not considering underflow bins. Thus, an entry $A_{ij}$ gives the probability that an event generated in bin $i$ is reconstructed in bin $j$. A clear diagonal is observed, showing the good performance of the mass reconstruction for XCone jets.
	
	\begin{figure}[tb]
		\centering
		\includegraphics [width=.6\textwidth]{../Plots/Unfolding/Data/Migration_prob.pdf}
		\caption{Display of the migration matrix. Each bin is scaled to the total event count in a column.}
		\label{fig:Migration}
	\end{figure}
	
	
\subsection{Closure Tests}
	Various tests of the unfolding are performed in this analysis based on simulated events. Firstly, a MC sample is unfolded with itself. Since the L-curve scan does not work when using statistically dependent samples, a $\tau$ value is set manually. Nevertheless, the output distribution of TUnfold is expected to exactly match the particle level distribution. The comparison presented in Fig.~\ref{fig:Unfolding_same} shows an exact accordance and proves a correct filling of the migration matrix. Another approach to test the unfolding is to split the MC sample into two statistically independent parts. For this test, $10\%$ are randomly selected and saved as pseudo data, while the other $90\%$ are used to construct the migration matrix. The splitting is performed four times, with every pseudo dataset containing different events. The unfolded pseudo data is compared in Fig.~\ref{fig:Unfolding_split1234} to its distribution on particle level, referred to as truth. In all bins except the second and third, unfolded distribution and truth agree well. The disagreement in the second and third bin could indicate a bias in the unfolding. \\
	Therefore, a further test is performed to check the unfolding for model dependence. Again, the MC sample is split, but now the renormalisation and factorisation scales are varied in the pseudo dataset. Figure~\ref{fig:Unfolding_split_scale} shows the unfolded distribution of pseudo data with both scales varied up and down. Here, biases above (scaled up) and below (scaled down) the particle level distributions are visible, especially in the second and third bins. Thus, the current unfolding setup is biased by differently modelled inputs. This can be explained with the reconstruction selection efficiency. Since only about $15\%$ of events that passed the particle level selection also pass the reconstruction level selection, the unfolding is strongly influenced by the underflow bin of the migration matrix. As the $8\;\text{TeV}$ analysis showed  \cite{torben_paper}, this effect can be compensated by include migrations from lower $p_T$ thresholds into the matrix. For this analysis in the current status, a model uncertainty is calculated from the difference of the unfolded distribution and its truth in every bin. Here, the maximum difference of both scale variations is considered and then included in the unfolding of data.
	
	\begin{figure}[tb]
		\centering
		\includegraphics [width=.55\textwidth]{../Plots/Unfolding/MC_Same/Unfold.pdf}
		\caption{Unfolding of a MC sample with itself. An exact agreement between unfolded distribution and its truth is observed, validating a correctly filled migration matrix.}
		\label{fig:Unfolding_same}
	\end{figure}	
	
	\begin{figure}[tb]
		\centering
		\includegraphics [width=.55\textwidth]{../Plots/Unfolding/MC_Split_all/Unfold.pdf}
		\caption{A unfolding with a MC sample split into $10\%$ pseudo data and $90\%$ to fill the migration matrix is performed four times with different pseudo datasets. Statistical uncertainties in the truth level distribution are represented by the line thickness.}
		\label{fig:Unfolding_split1234}
	\end{figure}		
	
	\begin{figure}[tb]
		\begin{subfigure}{.5\textwidth}
		\centering
		\includegraphics [width=\textwidth]{../Plots/Unfolding/MC_Split_up/Unfold.pdf}
		\caption{}
		\label{fig:Unfolding_split_up}
		\end{subfigure}
		\begin{subfigure}{.5\textwidth}
		\centering
		\includegraphics [width=\textwidth]{../Plots/Unfolding/MC_Split_down/Unfold.pdf}
		\caption{}
		\label{fig:Unfolding_split_down}
		\end{subfigure}		
		\caption{Unfolding with statistically independent MC samples where the renormalisation and factorisation scales in the pseudo data are scaled up (a) and down (b). The unfolded distributions are compared to the truth distributions where a bias is observed in both graphs.}
		\label{fig:Unfolding_split_scale}
	\end{figure}	
	
\FloatBarrier %draw figures of previous section before the new one starts	
\subsection{Unfolding of Data}
	Despite a preliminary migration matrix, a first test to unfold data is presented. Prior to the unfolding, background processes estimated from simulation are subtracted within the TUnfold software package. The unfolded distribution is depicted in comparison with the particle level distribution of the simulation in Fig.~\ref{fig:Unfolding_data}. Besides statistical uncertainties, indicated by the inner bars, the model uncertainty derived above is quadratically added to obtain a total uncertainty depicted by the outer error bars. Within the large uncertainties, the unfolded data agrees with the simulation. A more elaborate migration matrix is needed to damp these uncertainties which will be included in the future. The correlation matrix obtained from statistical uncertainties of the unfolded data is shown in Fig.~\ref{fig:Correlations}. Nearby bins are anti-correlated with correlation coefficients between $-0.6$ and $1$. The largest positive correlations observed in off-diagonal bins have values below $0.6$. This indicates an appropriate strength of the regularisation. The value for the regularisation strength $\tau$ is derived via a L-curve scan and was set to $\tau = 0.11$. Figure~\ref{fig:lcurve} shows the L-curve scan in the $L_x$-$L_y$ plane, where the point of largest curvature is highlighted.
	
	\begin{figure}[h]
		\begin{subfigure}{.5\textwidth}
		\centering
		\includegraphics [width=\textwidth]{../Plots/Unfolding/Data/Unfold.pdf}
		\caption{}
		\label{fig:Unfolding_data}
		\end{subfigure}		
		\begin{subfigure}{.5\textwidth}
		\centering
		\includegraphics [width=\textwidth]{../Plots/Unfolding/Data/Correlations.pdf}
		\caption{}
		\label{fig:Correlations}		
		\end{subfigure}	
		\caption{Unfolded data compared to the particle level distribution in simulation (a). The inner error bars show statistical uncertainties while the outer errors include the model uncertainties derived from scale variations. The correlation matrix, obtained from statistical uncertainties is displayed in (b).}
	\end{figure}
	\begin{figure}[h]
		\centering
		\includegraphics [width=.5\textwidth]{../Plots/Unfolding/Data/Lcurve.pdf}
		\caption{A display of the L-curve scan to derive the most suitable value for $\tau$. The point of largest curvature in the $L_x$-$L_y$ curve is highlighted by a red dot.}
		\label{fig:lcurve}
	\end{figure}	
	
\section{Summary of Results}
\label{sec:results}
	After a baseline selection to obtain a sample enriched by $t\bar{t}$ production, XCone jets are clustered to reconstruct top quark decays. The measurement phase space, selecting highly boosted top quarks is defined via two criteria:
	\begin{itemize}
	\item $p_T^{\text{1st jet}} > 400\;\text{GeV}$ and
	\item $M^{\text{1st jet}} > M^{\text{2nd jet + lepton}}$.
	\end{itemize}
	The fully hadronic top quark decay is reconstructed in a single jet. Figure~\ref{fig:Result1} shows the obtained jet mass distribution, which is sensitive to the top quark mass and resistant against pile-up while obtaining an excellent resolution and high statistics. \\
	Based on this distribution, an unfolding is performed, using simulation information to fill a migration matrix. The unfolding procedure is tested and validated with pseudo data and finally applied to data. Figure~\ref{fig:Result2} shows the unfolded distribution. Within the large uncertainties, unfolded data and the distribution on particle level from simulation show a good agreement. Further improvements are expected with a more elaborate phase space in the migration matrix and a closer look into the bins size of the measurement. This promises a measurement with much improved statistical and systematic uncertainties compared to the $8\;\text{TeV}$ result, and thus much higher sensitivity to the top quark mass.
	
	\begin{figure}[tb]
		\begin{subfigure}{.5\textwidth}
		\centering
		\includegraphics [width=1.05\textwidth, trim=0 0 5cm 0, clip]{../Plots/PostSel/XCone_cor_SF/M_jet1__lin.pdf}
 		\caption{}
		\label{fig:Result1}		
		\end{subfigure}		
		\begin{subfigure}{.5\textwidth}
		\centering
		\includegraphics [width=.95\textwidth,  trim=0 0.4cm 0 0, clip]{../Plots/Unfolding/Data/Unfold.pdf}
		\caption{}
		\label{fig:Result2}
		\end{subfigure}		
		\caption{As a result of this analysis, the unfolded data distribution (a) and the reconstructed jet mass (b) are shown. The very well performing XCone clustering algorithm reconstructs the top quark decay with high precision. Thus, the jet mass is sensitive to the top quark mass. The unfolded distribution agrees with the prediction by simulation but shows large model uncertainties.}
	\end{figure}

\chapter{Summary and Outlook}

	The analysis performing a measurement of the jet mass distribution in boosted top quark decays has been presented. Studies of various jet clustering algorithms were performed to find the most suitable jet definition. The influence of the jet radius on Anti-$k_T$ jets as well as jet mass distributions obtained by using HOTVR jets were examined and finally compared to a novel clustering method, namely XCone. The exclusive approach of XCone is used to define a specialised jet finding method for this analysis. This method proofed to be very successful in reconstructing hadronically decaying top quarks, returning a jet mass distribution with a narrow peak at the top quark mass. Together with high statistics and a large fraction of jets containing all decay, the jet mass distribution on particle level acts as very promising input for the unfolding procedure. \\
	On reconstruction level, baseline criteria are defined to select a preferably pure $t\bar{t}$ sample. Additionally, a further selection is applied defining a measurement phase space with boosted top quark decays. The jet mass distribution after passing every selection criterium shows is nice features on reconstruction level as well. Furthermore, also pile-up is not effecting the distribution much, indicating a well suited input for an unfolding procedure. In addition, jet energy corrections for XCone jets were derived. Based on AK4 corrections, that are expected behave similarly to XCone subjets with the same radius, a correction factor was obtained. The $p_T$ and $\eta$ dependent correction was found via a matching from reconstructed jets to jets on particle level in simulation. After validation with the well known $W$ boson mass, the new jet energy corrections are applied to every measured jet.\\	
	A first unfolding approach was tested with Monte Carlo samples. Unfolding of a simulation with itself confirmed a correctly filled migration matrix. Though, a check with independent samples indicated a modelling dependent bias. It expected to solve this issue with a further extended migration matrix, including migrations from lower jet $p_T$ regions. In addition, the binning of the matrix is not examined yet. Here, purity and stability are useful and recommended measures to define a well suited binning scheme. With simulations from different generators, a model dependency can be validated and better understood. Furthermore, the electron + jets channel can be added to this analysis to increase statistics and systematic uncertainties have to be covered to present a result. \\
	After obtaining a well understood unfolded distribution, a top quark mass can be extracted. First tests with simulations of different mass points but especially comparisons with theory calculations are planned in the future of this analysis. Calculations are not yet available for LHC but the use of XCone jets will certainly be preferred by theorist because of a clustering based on N-jettiness.

\appendix
\chapter{Jet Energy Corrections for XCone}
	After applying AK4 jet energy correction to the XCone subjets, an additional correction is derived to account for slight differences in XCone and Anti-$k_T$ jets. While figure \ref{fig:A_mean} shows the ratios between jet $p_T$ on reconstruction and particle level in the $\eta$-$p_T$-plane, Fig. \ref{fig:A_rms} shows the belonging RMS width. Since the width shows small values, each bin contains enough statistics to derive a correction factor. In every $\eta$ bin a $p_T$ dependent function is fitted to obtain a smooth correction. All separate fits are displayed in Fig. \ref{fig:A_fits}. 

	\begin{figure}[h]
		\centering
		\includegraphics [width=.9\textwidth]{../Plots/Correction/Mean_numbers}
		\caption{Mean of the ratio between jet $p_T$ on reconstruction and particle level.}
		\label{fig:A_mean}
	\end{figure}
	
	\begin{figure}[h]
		\centering
		\includegraphics [width=.9\textwidth]{../Plots/Correction/RMS_numbers}
		\caption{RMS values of $R=\frac{p_T^{\text{rec}}}{p_T^{\text{gen}}}$ in the $p_T$-$\eta$ plane.}
		\label{fig:A_rms}
	\end{figure}	
	
	\begin{figure}[h]
		\centering
		\includegraphics [width=.98\textwidth]{../Plots/Correction/Fits}
		\caption{Fit functions for XCone jet energy corrections in every $\eta$ region.}
		\label{fig:A_fits}
	\end{figure}


\pagestyle{plain}
% ===========================================================
% ======================== Literaturverzeichnis =============
% ===========================================================
% Eintr�ge erscheinen nur, wenn sie auch zitiert werden!!!
%\printbibliography
\bibliography{MasterBib}

%todo in bib style formatieren. z.B. darf "CMS Collaboration" nicht C. Collaboration werden
%----------------------------------------------------------------
%------------------Unterschrift----------------------------------
%----------------------------------------------------------------
\selectlanguage{ngerman}
\cleardoublepage
%todo einverstanden/nicht einverstanden??
\noindent Hiermit best�tige ich, dass die vorliegende Arbeit von mir selbstst"andig verfasst wurde und ich keine anderen als die angegebenen Hilfsmittel -- insbesondere keine im Quellenverzeichnis nicht benannten Internet-Quellen -- benutzt habe und die Arbeit von mir vorher nicht einem anderen Pr�fungsverfahren eingereicht wurde. Die eingereichte schriftliche Fassung entspricht der auf dem elektronischen Speichermedium. Ich bin damit einverstanden, dass die Masterarbeit ver�ffentlicht wird.\\
\newline
\newline
\newline
\begin{tabular*}{\textwidth}{lll}
	\hline
	Ort, Datum & \hspace{10.4cm} & Unterschrift  \\
\end{tabular*}

\cleardoublepage
%----------------------------------------------------------------
%---------------------Danksagung---------------------------------
%----------------------------------------------------------------
\section*{Danksagung}
Zun"achst m"ochte ich Prof. Dr. Johannes Haller daf"ur danken, dass ich die M"oglichkeit hatte, meine Masterarbeit in seiner Forschungsgruppe "uber dieses interessante Thema zu schreiben. Desweiteren bedanke ich mich f"ur die vielen R"ucksprachen und Denkanst"o"se, die meine Arbeit stets bereichert haben. Prof. Dr. Peter Schleper m"ochte ich daf"ur danken, dass er das Zweitgutachten meiner Arbeit "ubernehnommen hat. Ein besonderer Dank gilt Dr. Roman Kogler, der viel Zeit in die Betreuung meiner Analyse, zahlreiche Diskussionen "uber Jets und die Korrektur meiner Arbeit gesteckt hat. Au"serdem danke ich allen Mitgliedern der Arbeitsgruppe und insbesondere meinen B"uronachbarinnen und -nachbarn f"ur die Menge an gekl"arten Fragen, eine nette Atmosph"are und erhellende Gespr"ache in und abseits der Teilchenphysik. Zuletzt m"ochte ich meiner Familie und vor allem Inga danken, die mich immer unterst"utzt haben. 

\end{document}  
%----------------------------------------------------------------
%----------------------------------------------------------------
%----------------------------------------------------------------
