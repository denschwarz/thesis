\chapter{Measurement of the Top Quark Mass}
\label{ch:Measure}
	The top quark mass is an essential parameter to check the Standard Model of particle physics for consistency. For instance, the top quark has, due to its high mass, a large coupling to the Higgs boson. Therefore, it has to be included in correction terms for the Higgs boson mass. This leads to a relation of the masses of top quark, Higgs boson and electro-weak bosons that can be verified with a well known top quark mass, as performed in \cite{ewfit}. In this chapter conventional mass measurement methods are presented and compared to the measurement this analysis aims at. Additionally, previous results from CMS as well as its possible improvements are discussed. 
	
\section{Conventional Mass Measurements}
	Measuring the top quark mass is performed by reconstructing and combining the decay products to calculate back to the initial top quark. Usually a template  fit is used, meaning a comparison to Monte Carlo samples based on various top quark masses. Therefore, a selection is applied to data and simulation with the goal to select preferably $t\bar{t}$ events. Finally the MC samples with different $m_\text{top}^{MC}$ including a detector simulation are fitted to data. The most precise mass measurement of this kind is achieved combining results of the CMS, ATLAS, CDF, and D0 collaborations. The result of this world average is a value of $173.34 \pm 0.27 (\text{stat}) \pm 0.71 (\text{sys})\;\text{GeV}$ \cite{topmass_combination}. Performing these kinds of measurements, one relies on a correct simulation of $t\bar{t}$ events and is essentially measuring the top quark mass of a given simulation. This mass can then not easily be related to a mass parameter in a Lagrange density and is therefore referred to as the Monte Carlo mass $m_\text{top}^{MC}$. In addition, not all effects of QCD can be calculated \cite{nonperturbative} in simulation. This is why a cut-off \cite{cutoff} scale is defined which directly influences the mass associated with a given simulation (see section \ref{sec:Simulation}). Because of this difficulty, this analysis aims to provide a mass measurement without relying on ambiguities of the mass parameter in Monte Carlo simulation.

\section{Measurement in Boosted Decays}
	The present analysis aims at a mass measurement independent of the mass parameter in simulations. Therefore, it provides a well defined top quark mass measurement which is capable of validating current measurements as well as simulations. The approach is to obtain a distribution that can be calculated and thus be directly compared to theory. For lepton collisions it is shown in \cite{eejetmass} that a jet mass distribution in can be calculated and its peak is sensitive to the top quark mass. Mentioned calculations are done dividing the event in two hemispheres containing one top quark decay each. The jet mass is then the invariant mass of the sum of all particles in one hemisphere. Since events at LHC contain much more objects not belonging to the $t\bar{t}$ system, jets with a finite cone size are defined. To reconstruct a top quark using a single jet, all decay products have to fit into the selected cone size. Therefore, Lorentz boosted top quarks are selected to obtain small distances between the decay products. The jet mass $m_\text{jet}$ in this analysis is then defined as the invariant mass of the jet four-vector. To obtain the vector, all four-vectors of particles clustered into a jet a summarised. In conclusion the jet mass reads:
	\begin{equation}
	m_\text{jet} = \sqrt{\left( \sum_{i} p_i \right) \cdot \left( \sum_{i} p_i \right)},
	\end{equation}
	where $i$ runs over all constituents of a jet and $p_i$ indicates the four-vector of the $i$-th constituent. After selecting suitable events, an unfolding is performed to obtain a jet mass distribution comparable to particle level as well as theory calculations. 
	
	
	%todo ref: proposal to use boosted regime
	%todo ref: jet mass calculated
	%todo scet analysis


\section{Previous Results}
	A measurement of the jet mass in highly boosted $t\bar{t}$ events \cite{torben_paper} has already been performed by CMS with the $8\;\text{TeV}$ dataset corresponding to an integrated luminosity of $19.7\;\text{fb}^{-1}$. The resulting differential cross section measurement and a summary of the uncertainties are shown in Fig. \ref{fig:Torben1} and Fig. \ref{fig:Torben2}, respectively. In the mentioned publication, Cambridge/Aachen jets with a radius parameter of $R=1.2$ were used to reconstruct top quark decays, proving the capability of the method but not reaching high precision. Two main points a subsequent analysis on $13\;\text{TeV}$ could largely improve are statistical uncertainties and effects due to pile-up. First, the statistic is now very high in the selected phase space on $8\;\text{TeV}$. Since the $t\bar{t}$ cross section increases by a factor of about three in this region with a centre-of-mass energy of $13\;\text{TeV}$, this uncertainty is expected to decrease. This also influences the choice of jet algorithm and cone size. While in the $8\;\text{TeV}$ analysis large cones are crucial to obtain enough statistics, smaller cone sizes should be possible on $13\;\text{TeV}$. The second limitation is pile-up. Because of the large cone size, pile-up largely influences the measurement, visible as shift of the peak position to values above the top quark mass. Again, a smaller cone on $13\;\text{TeV}$ is expected to decrease this effect.

	\begin{figure}[tb]
		\begin{subfigure}{.5\textwidth}
	    \centering
		\includegraphics [width=\textwidth]{../Plots/Torben/Torben_result_paper}
		\caption{}
		\label{fig:Torben1}
		\end{subfigure}
		\begin{subfigure}{.5\textwidth}
		\centering
		\includegraphics [width=\textwidth]{../Plots/Torben/Torben_error_paper}
		\caption{}
		\label{fig:Torben2}
		\end{subfigure}
		\caption{Unfolded cross section measurement (a) and display of all uncertainties (b) from a similar analysis performed at $8\;\text{TeV}$. Taken from \cite{torben_paper}.}
		\label{fig:Torben}
	\end{figure}	
	
