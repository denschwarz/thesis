% !TEX root = Master_Thesis.tex
\chapter{Measurement of the Top Quark Mass}
\label{ch:Measure}
	The top quark mass is an essential parameter to check the Standard Model of particle physics for consistency. For instance, the top quark has, due to its high mass, a large coupling to the Higgs boson. Therefore, it has to be included in correction terms for the Higgs boson mass. This makes it possible to relate the masses of Higgs boson, $W$ boson and top quark and thus validate measured values, as performed in \cite{ewfit}. In this chapter conventional mass measurement methods are presented and compared to the measurement this analysis aims at. Additionally, previous results from CMS as well as its possible improvements are discussed. 
	
\section{Conventional Mass Measurements}
	The most precise measurements of the top quark mass are performed by reconstructing and combining the top quark decay products to obtain the invariant mass of the top quark. Usually a template  fit to the distributions of the mass is used, meaning a comparison to Monte Carlo simulations obtained by using different values of the top quark mass $m_\text{top}^{MC}$ of the event generator. A selection is applied to data and simulation with the goal to select $t\bar{t}$ events. The MC samples with different values of $m_\text{top}^{MC}$, including detector simulation, are then fitted to data. The most precise mass measurement of this kind is achieved by combining results of the CMS, ATLAS, CDF, and D0 collaborations. The result of this world average is a value of $173.34 \pm 0.27 (\text{stat}) \pm 0.71 (\text{sys})\;\text{GeV}$ \cite{topmass_combination}. Performing these kinds of measurements, one relies on a correct simulation of $t\bar{t}$ events and is essentially measuring the top quark mass of a given simulation. This mass can not easily be related to a mass parameter in a Lagrange density and is therefore referred to as the Monte Carlo mass $m_\text{top}^{MC}$. In addition, not all effects of QCD can be calculated \cite{nonperturbative} in simulation. This is why a cut-off scale \cite{cutoff} is defined which directly influences the mass associated with a given simulation (see section~\ref{sec:Simulation}). Because of this difficulty, this analysis aims to provide a measurement which can help to resolve ambiguities of the mass parameter in Monte Carlo simulation.

\section{Measurement in Boosted Decays}
	The present analysis aims at a measurement of the top quark jet mass, which is independent of $m_\text{top}^{MC}$ used in simulations. The idea is to obtain a distribution with high sensitivity to the value of $m_\text{top}$, which can be calculated from first principle and thus be directly compared to analytical calculations. For lepton collisions it is shown in \cite{eejetmass} that a jet mass distribution can be calculated and its peak is sensitive to the top quark mass. These calculations are performed dividing the event in two hemispheres containing one top quark decay each. The jet mass is then the invariant mass of the sum of all particles in one hemisphere.\\
	Since events at the LHC contain many more objects not belonging to the $t\bar{t}$ system, jets with a finite cone size are defined. All decay products have to fit into the selected cone size, to reconstruct a top quark using a single jet. Therefore, Lorentz boosted top quarks are selected to obtain small distances between the decay products. The jet mass $m_\text{jet}$ in this analysis is then defined as the invariant mass of the jet four-vector,
	\begin{equation}
	m_\text{jet} = \sqrt{\left( \sum_{i} p_i \right) \cdot \left( \sum_{i} p_i \right)},
	\end{equation}
	where $i$ runs over all constituents of a jet and $p_i$ indicates the four-vector of the $i$-th constituent. After selecting suitable events, an unfolding is performed to obtain a jet mass distribution comparable to particle level as well as analytical calculations. 

\section{Previous Results}
	A measurement of the jet mass in highly boosted $t\bar{t}$ events \cite{torben_paper} has already been performed by CMS with the $8\;\text{TeV}$ dataset corresponding to an integrated luminosity of $19.7\;\text{fb}^{-1}$. The resulting differential cross section measurement and a summary of the uncertainties are shown in Fig.~\ref{fig:Torben1} and Fig.~\ref{fig:Torben2}, respectively. In this analysis, Cambridge/Aachen jets with a radius parameter of $R=1.2$ were used to reconstruct top quark decays, proving the capability of the method but not reaching high precision. Two main points were identified, which a subsequent analysis on $13\;\text{TeV}$ could largely improve: statistical uncertainties and effects due to pile-up. First, statistics are very high in the selected phase space on $13\;\text{TeV}$. Since the $t\bar{t}$ cross section increases by a factor of about three and about a factor of ten in the highly boosted regime with a centre-of-mass energy of $13\;\text{TeV}$, the statistical uncertainty is expected to decrease. This also influences the choice of jet algorithm and cone size. While in the $8\;\text{TeV}$ analysis large cones are crucial to obtain enough statistics, smaller cone sizes are possible on $13\;\text{TeV}$. The second limitation is pile-up. Because of the large cone size, pile-up largely influences the measurement, visible as a shift of the peak position in reconstructed events to values above the top quark mass. Again, a smaller cone size on $13\;\text{TeV}$ is expected to decrease this effect. In addition, various jet clustering algorithms can be studied to find the most suitable method to reconstruct top quark decays inside one jet.

	\begin{figure}[tb]
		\begin{subfigure}{.5\textwidth}
	    \centering
		\includegraphics [width=\textwidth]{../Plots/Torben/Torben_result_paper}
		\caption{}
		\label{fig:Torben1}
		\end{subfigure}
		\begin{subfigure}{.5\textwidth}
		\centering
		\includegraphics [width=\textwidth]{../Plots/Torben/Torben_error_paper}
		\caption{}
		\label{fig:Torben2}
		\end{subfigure}
		\caption{Unfolded cross section measurement (a) and display of all uncertainties (b) from a similar analysis performed at $8\;\text{TeV}$. Taken from \cite{torben_paper}.}
		\label{fig:Torben}
	\end{figure}	
	
